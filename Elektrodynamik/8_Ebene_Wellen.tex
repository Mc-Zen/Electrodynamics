% !TeX root = Theo_III.tex



\chapter{Ebene elektromagnetische Wellen}


Das Ziel dieses Kapitels ist die Beschreibung von elektromagnetischen Wellen, was eine wesentliche Errungenschaft der Maxwellschen Theorie darstellt.
Außerdem soll die dynamische Theorie und Frequenzabhängigkeit der komplexen Dielektrizität $\varepsilon\in\mathbb{C}$ erläutert werden\footnote{Natürlich ist auch die Permeabilität $\mu$ im Grunde frequenzabhängig und komplex, allerdings ist für die meisten Materialien $\mu(\omega)\approx 1$. Insbesondere in der Optik wird diese Näherung fast immer angewandt. }.

Als Ausgangspunkt dienen die Maxwell-Gleichungen in Materie, wobei wir freie Ladungen und Ströme vernachlässigen wollen\footnote{In diesem und dem folgenden Kapitel wird anstelle der kovarianten Vierer-Notation wieder die alte Vektorschreibweise verwendet.}:
\begin{align}
    \nabla\cdot\vec E=0,\quad \nabla\cdot\vec B=0, \quad \nabla\times \vec E=-\frac{\partial\vec B}{\partial t}, \quad \nabla\times \vec B=\mu\varepsilon \frac{\partial\vec E}{\partial t}
\end{align}
Zur Vereinfachung sollen die Materialgrößen ortsunabhängig sein, $\varepsilon\neq\varepsilon(\vec r),\mu\neq\mu(\vec r)$.




\section{Ebene Wellen im nichtleitenden, homogenen Medium}

Die Wellengleichungen für nichtleitende homogene Medien erhalten wir durch Anwenden der Rotation auf $\nabla\times \vec E$ und $\nabla\times\vec B$:
\begin{align*}
    \nabla\times\nabla\times\vec E            & = -\frac{\partial}{\partial t}\nabla\times \vec B =-\mu\varepsilon \frac{\partial^2}{\partial t^2}\vec E \\
    \nabla(\nabla\cdot \vec E)-\nabla^2\vec E & = -\mu\varepsilon \frac{\partial^2}{\partial t^2}\vec E.
\end{align*}
Analog wird die Wellengleichung für das magnetische Feld hergeleitet. Zusammengefasst ist
\begin{align}
    \label{eq:wellengleichungen}
    \boxed{\begin{aligned}
                   \left(\nabla^2-\frac{1}{c^2}\frac{\partial^2}{\partial t^2}\right) \vec E(\vec r,t) & =0 \\
                   \left(\nabla^2-\frac{1}{c^2}\frac{\partial^2}{\partial t^2}\right) \vec B(\vec r,t) & =0
               \end{aligned}}
\end{align}
mit der Ausbreitungsgeschwindigkeit
\begin{align*}
    c=\frac{1}{\sqrt{\varepsilon\mu}} = \frac{c_0}{\sqrt{\varepsilon_r\mu_r}}
\end{align*}
der elektromagnetischen Wellen in Materie und Vakuumlichtgeschwindigkeit
\begin{align*}
    c_0 = \frac{1}{\sqrt{\varepsilon_0\mu_0}}.
\end{align*}
Die Ausbreitungsgeschwindigkeit von Licht im Vakuum ist also gleich der von elektromagnetischen Wellen und die Vermutung liegt nahe,
dass es sich bei Licht selbst um eine elektromagnetische Welle handelt, was von Heinrich Hertz 1886 auch bestätigt wurde.

Zur Lösung der Wellengleichungen \eqref{eq:wellengleichungen} machen wir den Lösungsansatz einer ebenen (komplexen\footnote{Natürlich existieren in der Realität keine komplexen Wellen, aber ihre Verwendung macht viele Rechnungen deutlich komfortabler.
    Es gilt, dass wenn eine komplexe Funktion eine Differentialgleichung erfüllt, dass dann auch Realteil und Imaginärteil für sich diese Differentialgleichung erfüllen.
    Um also die physikalische Welle zu erhalten, muss lediglich der Realteil $\real(\vec E),\real(\vec B)$ gebildet werden. }) Welle,
\begin{align}
    \label{eq:ansatz_ebene_welle}
    \begin{pmatrix}
        \vec E(\vec r,t) \\
        \vec B(\vec r,t)
    \end{pmatrix}
    =
    \begin{pmatrix}
        \vec E_0 \\
        \vec B_0
    \end{pmatrix}
    e^{i(\vec k\cdot \vec r-\omega t)}.
\end{align}
Dabei ist $\omega=2\pi/T$ die Kreisfrequenz und $T$ die Periodendauer.
Ferner ist $\vec k=k \vec{\hat{k}}$ der Wellenvektor, der in Ausbreitungsrichtung zeigt und dessen Betrag $k=|k|=2\pi/\lambda$ auch als Wellenzahl bezeichnet wird, welche umgekehrt zu der Wellenlänge $\lambda$ ist.

Einsetzen des Ansatzes \eqref{eq:ansatz_ebene_welle} in die Wellengleichungen \eqref{eq:wellengleichungen} liefert
\begin{align*}
    \left(-k^2+\frac{1}{c^2}\omega^2\right) \begin{pmatrix} \vec E \\ \vec B \end{pmatrix} = 0,
\end{align*}
woraus die bekannte Dispersionsrelation folgt:
\begin{align}
    \label{eq:dispersionsrelation}
    \omega^2 = c^2k^2
\end{align}
Man erkennt, dass es sich bei der Ausbreitungsgeschwindigkeit $c$ auch um die Phasengeschwindigkeit von elektromagnetischen Wellen handelt, die ja als $\omega/k$ definiert ist.
Dieser Zusammenhang lässt sich gut veranschaulichen, indem die Welle mithilfe von \eqref{eq:dispersionsrelation} zu (zur Vereinfachung eindimensional)
\begin{align*}
    \begin{pmatrix}
        \vec E(\vec r,t) \\
        \vec B(\vec r,t)
    \end{pmatrix}
    =
    \begin{pmatrix}
        \vec E_0 \\
        \vec B_0
    \end{pmatrix}
    e^{ik(z \mp ct)} =
    \begin{pmatrix}
        \vec E(z\mp ct) \\
        \vec B(z\mp ct)
    \end{pmatrix}
\end{align*}
umgeschrieben wird. Orte gleicher Phase (also bei konstantem Argument $z\mp ct$) wandern bei fortschreitender Zeit $t$ mit der Geschwindigkeit $c$ in (oder entgegen der) Ausbreitungsrichtung $\hat{\vec k}$.


Das Verhältnis von Vakuumlichtgeschwindigkeit und Ausbreitungsgeschwindigkeit in einem Medium wird als Brechungsindex definiert:
\begin{align}
    \label{eq:definition_brechungsindex}
    n=\frac{c_0}{c} \implication n=\sqrt{\varepsilon_r\mu_r}
\end{align}
Meistens (vor allem in der Optik) kann man $\mu_r$ auf $1$ setzen und erhält.
\begin{align*}
    n = \sqrt{\varepsilon_r}
\end{align*}
Natürlich treten (unendlich ausgedehnte) ebene Wellen in der Realität nicht auf, aber sie stellen häufig eine gute Näherung dar.
Außerdem lässt sich aus der Überlagerung von vielen ebenen Wellen ein Wellenpaket konstruieren, das praktisch eine endliche Ausdehnung hat.
Aufgrund der Dispersion zerläuft allerdings ein solches Wellenpaket in Medien mit $\varepsilon\neq 1$ und $\mu\neq 1$, weil Wellen mit verschiedenen Frequenzen auch unterschiedliche Ausbreitungsgeschwindigkeiten haben, $c=c(\omega)$.

Obwohl hier ohne Beweis aufgeführt, löst auch eine Kugelwelle
\begin{align*}
    \begin{pmatrix} \vec E \\ \vec B \end{pmatrix} \propto \frac{1}{r}e^{i(kr-\omega t)}
\end{align*}
die Wellengleichungen \eqref{eq:wellengleichungen}.




\subsection{Dämpfung}

Findet in einem Medium Absorption statt, wird die elektromagnetische Welle gedämpft.
Dies lässt sich durch eine komplexe Dielektrizität und einen komplexen Brechungsindex
\begin{align*}
    \bar{n} = \sqrt{\varepsilon_r} = n + i\kappa
\end{align*}
beschreiben, bei dem der Imaginärteil $\kappa$ die Absorption beschreibt.
Unter Verwendung von der Dispersionsrelation \eqref{eq:dispersionsrelation} und der Definition des Brechungsindexes \eqref{eq:definition_brechungsindex} erhält man so
\begin{align*}
    k=\frac{\omega}{c_0} \bar n = \frac{\omega}{c_0}(n+i\kappa).
\end{align*}
Setzt man diesen Ausdruck in die eindimensionale ebene Welle ein,
\begin{align*}
    \begin{pmatrix} \vec E \\ \vec B \end{pmatrix} = \begin{pmatrix} \vec E_0 \\ \vec B_0 \end{pmatrix} e^{i(kz-\omega t)} =\begin{pmatrix} \vec E_0 \\ \vec B_0 \end{pmatrix} e^{i\left(\frac{\omega}{c_0}(n+i\kappa) z-\omega t\right)}
    =\begin{pmatrix} \vec E_0 \\ \vec B_0 \end{pmatrix} e^{i\frac{\omega}{c_0}n \left( z- \frac{c_0}{n} t\right)}e^{-\frac{\omega\kappa}{c_0}z}
\end{align*}
erhält man einen Anteil mit realer Exponentialfunktion
\begin{align*}
    e^{-\frac{z}{\xi}}, \quad \xi = \frac{c_0}{\omega\kappa},
\end{align*}
der eine exponentielle Dämpfung mit Absorptionslänge $\xi$ kennzeichnet (siehe \Abbref{fig:daempfung_em_welle}).
Diese ist gerade antiproportional zu $\kappa$ \textendash{} also bestimmt der Imaginärteil des komplexen Brechungsindexes $\bar n$ die Dämpfung.

\begin{figure}[htb]
    \centering
    \tfigDaempfungEMWelle
    \caption{Gedämpfte Welle mit Einhüllender $\pm e^{-z/\xi}$ (grün). Die Absorptionslänge $\xi$ liegt dort, wo die Amplitude der Welle auf $1/e$ abgesunken ist. }
    \label{fig:daempfung_em_welle}
\end{figure}


\subsection{Polarisation}

Die Polarisation einer elektromagnetischen Welle gibt die Schwingungsrichtung des Feldes an und ist nicht zu verwechseln mit der dielektrischen Polarisation, welche das elektrische Dipolmoment beschreibt.

Aus den Maxwellgleichungen folgt, dass das elektrische und das magnetische Feld senkrecht zur Ausbreitungsrichtung schwingen:
\begin{align*}
    \left. \begin{aligned} \nabla\cdot \vec E &=0 \\ \nabla\cdot \vec B &= 0\end{aligned} \right\} \implication \left. \begin{aligned} \vec k\cdot\vec E_0&=0\equivalence \vec E_0\perp\vec k \\ \vec k\cdot\vec B_0&=0\equivalence\vec B_0\perp\vec k\end{aligned}\right\}
\end{align*}
Das gilt allerdings nur für isotrope Medien, allgemein gilt nur $\vec D_0 \perp\vec k$, aber $\vec E_0 \not\perp\vec k$ in anisotropen Medien, da $\vec D=\varepsilon\vec E \nparallel\vec E$, weil $\varepsilon$ ein Tensor ist.

Die Ebene, die den Wellenvektor $k$ und das elektrische bzw. magnetische Feld enthält, wird als Polarisationsebene bezeichnet.
Die folgenden Ausführungen werden sich auf das elektrische Feld beschränken.
Das magnetische Feld verhält sich ganz analog, ist aber stets orthogonal zum elektrischen.

Es gibt verschiedene Arten von Polarisation:
\begin{itemize}
    \item \textbf{Lineare Polarisation:} Das elektrische Feld schwingt entlang einer gleichbleibenden Achse \textendash{} die Polarisationsebene ist also konstant, wie in Abbildung \Abbref{fig:lineare_polarisation} dargestellt. Das elektrische Feld ist demnach stets in die gleiche Richtung $\vec e$ gerichtet:
          \begin{align*}
              \vec E_0 = E_0 \vec e.
          \end{align*}
          Aus der Maxwell-Gleichung folgt für einen Ansatz wie \eqref{eq:ansatz_ebene_welle}, dass
          \begin{align*}
              \nabla\times\vec E                                         & = -\partial_t\vec B                                   \\
              i\vec k \times \vec E_0 e^{i(\vec k\cdot \vec r-\omega t)} & = i\omega\vec B_0 e^{i(\vec k\cdot \vec r-\omega t)},\end{align*}
          also
          \begin{align}
              \label{eq:b_feld_aus_e_feld}
              \boxed{\vec B_0 = \frac{1}{\omega}\vec k\times \vec E_0.}
          \end{align}
          \begin{figure}[htb]
              \centering
              \tfigLinearePolarisation
              \caption{Lineare Polarisation einer ebenen Welle: Die Feldvektoren des elektrischen und des magnetischen Feldes stehen orthogonal zum Wellenvektor $\vec k$, welcher die Ausbreitungsgeschwindigkeit kennzeichnet. Die Schwingungsebene des elektrischen und des magnetischen Feldes ist für lineare Polarisation jeweils konstant. }
              \label{fig:lineare_polarisation}
          \end{figure}


    \item \textbf{Zirkulare Polarisation:} Die Schwingungsebene rotiert mit der Winkelgeschwindigkeit $\omega$ und die $\vec k$-Achse (siehe \Abbref{fig:zirkulare_polarisation}). Die Rotation des $\vec E_0$-Vektors lässt sich folgendermaßen beschreiben:
          \begin{align*}
              \vec E_0 = E_0(\vec e_1\pm i\vec e_2)
          \end{align*}
          Dabei wird eine um \SI{90}{\degree} phasenverschobene Welle erzeugt, die entlang $\vec e_2$ polarisiert ist.
          Die beiden jeweils linear polarisierten Felder werden mit gleicher Stärke überlagert, sodass sich insgesamt der Realteil des Vektors $\vec E_0$ im Kreis bewegt.
          \begin{figure}[htb]
              \centering
              \tfigZirkularePolarisation
              \caption{Zirkulare (links) und elliptische (rechts) Polarisation einer ebenen Welle: Der elektrische Feldvektor dreht sich auf einer Kreisbahn bzw. einer elliptischen Bahn mit der Kreisfrequenz $\omega$. }
              \label{fig:zirkulare_polarisation}
          \end{figure}


    \item \textbf{Elliptische Polarisation:} Die zirkulare Polarisation ist ein Sonderfall der elliptischen Polarisation, bei der die beiden überlagerten Felder eine unterschiedliche Stärke haben:
          \begin{align*}
              \vec E_0 = E_{01}\vec e_1 \pm i E_{02}\vec e_2
          \end{align*}


          % \item \textbf{Elliptische Polarisation:} Die zirkulare Polarisation ist ein Sonderfall der elliptischen Polarisation, bei der die beiden überlagerten Felder eine unterschiedliche Stärke haben:
          % \begin{align*}
          %     \vec E_0 = E_{01}\vec e_1 \pm i E_{02}\vec e_2
          % \end{align*}
\end{itemize}