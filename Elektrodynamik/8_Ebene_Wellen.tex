% !TeX root = Theo_III.tex



\chapter{Ebene elektromagnetische Wellen}


Das Ziel dieses Kapitels ist die Beschreibung von elektromagnetischen Wellen, was eine wesentliche Errungenschaft der Maxwellschen Theorie darstellt.
Außerdem soll die dynamische Theorie und Frequenzabhängigkeit der komplexen Dielektrizität $\varepsilon\in\mathbb{C}$ erläutert werden\footnote{Natürlich ist auch die Permeabilität $\mu$ im Grunde frequenzabhängig und komplex, allerdings ist für die meisten Materialien $\mu(\omega)\approx 1$. Insbesondere in der Optik wird diese Näherung fast immer angewandt. }.

Als Ausgangspunkt dienen die Maxwell-Gleichungen in Materie, wobei wir freie Ladungen und Ströme vernachlässigen wollen\footnote{In diesem und dem folgenden Kapitel wird anstelle der kovarianten Vierer-Notation wieder die alte Vektorschreibweise verwendet.}:
\begin{align}
    \nabla\cdot\vec E=0,\quad \nabla\cdot\vec B=0, \quad \nabla\times \vec E=-\frac{\partial\vec B}{\partial t}, \quad \nabla\times \vec B=\mu\varepsilon \frac{\partial\vec E}{\partial t}
\end{align}
Zur Vereinfachung sollen die Materialgrößen ortsunabhängig sein, $\varepsilon\neq\varepsilon(\vec r),\mu\neq\mu(\vec r)$.




\section{Ebene Wellen im nichtleitenden, homogenen Medium}

Die Wellengleichungen für nichtleitende homogene Medien erhalten wir durch Anwenden der Rotation auf $\nabla\times \vec E$ und $\nabla\times\vec B$:
\begin{align*}
    \nabla\times\nabla\times\vec E            & = -\frac{\partial}{\partial t}\nabla\times \vec B =-\mu\varepsilon \frac{\partial^2}{\partial t^2}\vec E \\
    \nabla(\nabla\cdot \vec E)-\nabla^2\vec E & = -\mu\varepsilon \frac{\partial^2}{\partial t^2}\vec E.
\end{align*}
Analog wird die Wellengleichung für das magnetische Feld hergeleitet. Zusammengefasst ist
\begin{align}
    \label{eq:wellengleichungen}
    \boxed{\begin{aligned}
                   \left(\nabla^2-\frac{1}{c^2}\frac{\partial^2}{\partial t^2}\right) \vec E(\vec r,t) & =0 \\
                   \left(\nabla^2-\frac{1}{c^2}\frac{\partial^2}{\partial t^2}\right) \vec B(\vec r,t) & =0
               \end{aligned}}
\end{align}
mit der Ausbreitungsgeschwindigkeit
\begin{align*}
    c=\frac{1}{\sqrt{\varepsilon\mu}} = \frac{c_0}{\sqrt{\varepsilon_r\mu_r}}
\end{align*}
der elektromagnetischen Wellen in Materie und Vakuumlichtgeschwindigkeit
\begin{align*}
    c_0 = \frac{1}{\sqrt{\varepsilon_0\mu_0}}.
\end{align*}
Die Ausbreitungsgeschwindigkeit von Licht im Vakuum ist also gleich der von elektromagnetischen Wellen und die Vermutung liegt nahe,
dass es sich bei Licht selbst um eine elektromagnetische Welle handelt, was von Heinrich Hertz 1886 auch bestätigt wurde.



\subsection{Ungedämpfte ebene Welle}

Zur Lösung der Wellengleichungen \eqref{eq:wellengleichungen} machen wir den Lösungsansatz einer ebenen (komplexen\footnote{Natürlich existieren in der Realität keine komplexen Wellen, aber ihre Verwendung macht viele Rechnungen deutlich komfortabler.
    Es gilt, dass wenn eine komplexe Funktion eine Differentialgleichung erfüllt, dass dann auch Realteil und Imaginärteil für sich diese Differentialgleichung erfüllen.
    Um also die physikalische Welle zu erhalten, muss lediglich der Realteil $\real(\vec E),\real(\vec B)$ gebildet werden. }) Welle,
\begin{align}
    \label{eq:ansatz_ebene_welle}
    \begin{pmatrix}
        \vec E(\vec r,t) \\
        \vec B(\vec r,t)
    \end{pmatrix}
    =
    \begin{pmatrix}
        \vec E_0 \\
        \vec B_0
    \end{pmatrix}
    e^{i(\vec k\cdot \vec r-\omega t)}.
\end{align}
Dabei ist $\omega=2\pi/T$ die Kreisfrequenz und $T$ die Periodendauer.
Ferner ist $\vec k=k \vec{\hat{k}}$ der Wellenvektor, der in Ausbreitungsrichtung zeigt und dessen Betrag $k=|k|=2\pi/\lambda$ auch als Wellenzahl bezeichnet wird, welche umgekehrt zu der Wellenlänge $\lambda$ ist.

Einsetzen des Ansatzes \eqref{eq:ansatz_ebene_welle} in die Wellengleichungen \eqref{eq:wellengleichungen} liefert
\begin{align*}
    \left(-k^2+\frac{1}{c^2}\omega^2\right) \begin{pmatrix} \vec E \\ \vec B \end{pmatrix} = 0,
\end{align*}
woraus die bekannte Dispersionsrelation folgt:
\begin{align}
    \label{eq:dispersionsrelation}
    \omega^2 = c^2k^2
\end{align}
Man erkennt, dass es sich bei der Ausbreitungsgeschwindigkeit $c$ auch um die Phasengeschwindigkeit von elektromagnetischen Wellen handelt, die ja als $\omega/k$ definiert ist.
Dieser Zusammenhang lässt sich gut veranschaulichen, indem die Welle mithilfe von \eqref{eq:dispersionsrelation} zu (zur Vereinfachung eindimensional)
\begin{align*}
    \begin{pmatrix}
        \vec E(\vec r,t) \\
        \vec B(\vec r,t)
    \end{pmatrix}
    =
    \begin{pmatrix}
        \vec E_0 \\
        \vec B_0
    \end{pmatrix}
    e^{ik(z \mp ct)} =
    \begin{pmatrix}
        \vec E(z\mp ct) \\
        \vec B(z\mp ct)
    \end{pmatrix}
\end{align*}
umgeschrieben wird. Orte gleicher Phase (also bei konstantem Argument $z\mp ct$) wandern bei fortschreitender Zeit $t$ mit der Geschwindigkeit $c$ in (oder entgegen der) Ausbreitungsrichtung $\hat{\vec k}$.


Das Verhältnis von Vakuumlichtgeschwindigkeit und Ausbreitungsgeschwindigkeit in einem Medium wird als Brechungsindex definiert:
\begin{align}
    \label{eq:definition_brechungsindex}
    n=\frac{c_0}{c} \implication n=\sqrt{\varepsilon_r\mu_r}
\end{align}
Meistens (vor allem in der Optik) kann man $\mu_r$ auf $1$ setzen und erhält.
\begin{align*}
    n = \sqrt{\varepsilon_r}
\end{align*}
Natürlich treten (unendlich ausgedehnte) ebene Wellen in der Realität nicht auf, aber sie stellen häufig eine gute Näherung dar.
Außerdem lässt sich aus der Überlagerung von vielen ebenen Wellen ein Wellenpaket konstruieren, das praktisch eine endliche Ausdehnung hat.
Aufgrund der Dispersion zerläuft allerdings ein solches Wellenpaket in Medien mit $\varepsilon\neq 1$ und $\mu\neq 1$, weil Wellen mit verschiedenen Frequenzen auch unterschiedliche Ausbreitungsgeschwindigkeiten haben, $c=c(\omega)$.

Obwohl hier ohne Beweis aufgeführt, löst auch eine Kugelwelle
\begin{align*}
    \begin{pmatrix} \vec E \\ \vec B \end{pmatrix} \propto \frac{1}{r}e^{i(kr-\omega t)}
\end{align*}
die Wellengleichungen \eqref{eq:wellengleichungen}.




\subsection{Dämpfung}

Findet in einem Medium Absorption statt, wird die elektromagnetische Welle gedämpft.
Dies lässt sich durch eine komplexe Dielektrizität und einen komplexen Brechungsindex
\begin{align*}
    \bar{n} = \sqrt{\varepsilon_r} = n + i\kappa
\end{align*}
beschreiben, bei dem der Imaginärteil $\kappa$ die Absorption beschreibt.
Unter Verwendung von der Dispersionsrelation \eqref{eq:dispersionsrelation} und der Definition des Brechungsindexes \eqref{eq:definition_brechungsindex} erhält man so
\begin{align*}
    k=\frac{\omega}{c_0} \bar n = \frac{\omega}{c_0}(n+i\kappa).
\end{align*}
Setzt man diesen Ausdruck in die eindimensionale ebene Welle ein,
\begin{align*}
    \begin{pmatrix} \vec E \\ \vec B \end{pmatrix} = \begin{pmatrix} \vec E_0 \\ \vec B_0 \end{pmatrix} e^{i(kz-\omega t)} =\begin{pmatrix} \vec E_0 \\ \vec B_0 \end{pmatrix} e^{i\left(\frac{\omega}{c_0}(n+i\kappa) z-\omega t\right)}
    =\begin{pmatrix} \vec E_0 \\ \vec B_0 \end{pmatrix} e^{i\frac{\omega}{c_0}n \left( z- \frac{c_0}{n} t\right)}e^{-\frac{\omega\kappa}{c_0}z}
\end{align*}
erhält man einen Anteil mit realer Exponentialfunktion
\begin{align*}
    e^{-\frac{z}{\xi}}, \quad \xi = \frac{c_0}{\omega\kappa},
\end{align*}
der eine exponentielle Dämpfung mit Absorptionslänge $\xi$ kennzeichnet (siehe \Abbref{fig:daempfung_em_welle}).
Diese ist gerade antiproportional zu $\kappa$ \textendash{} also bestimmt der Imaginärteil des komplexen Brechungsindexes $\bar n$ die Dämpfung.

\begin{figure}[htb]
    \centering
    \tfigDampeningElectromagneticWave
    \caption{Gedämpfte Welle mit Einhüllender $\pm e^{-z/\xi}$ (grün). Die Absorptionslänge $\xi$ liegt dort, wo die Amplitude der Welle auf $1/e$ abgesunken ist. }
    \label{fig:daempfung_em_welle}
\end{figure}


\subsection{Polarisation}

Die Polarisation einer elektromagnetischen Welle gibt die Schwingungsrichtung des Feldes an und ist nicht zu verwechseln mit der dielektrischen Polarisation, welche das elektrische Dipolmoment beschreibt.

Aus den Maxwellgleichungen folgt, dass das elektrische und das magnetische Feld senkrecht zur Ausbreitungsrichtung schwingen:
\begin{align*}
    \left. \begin{aligned} \nabla\cdot \vec E &=0 \\ \nabla\cdot \vec B &= 0\end{aligned} \right\} \implication \left. \begin{aligned} \vec k\cdot\vec E_0&=0\equivalence \vec E_0\perp\vec k \\ \vec k\cdot\vec B_0&=0\equivalence\vec B_0\perp\vec k\end{aligned}\right\}
\end{align*}
Das gilt allerdings nur für isotrope Medien, allgemein gilt nur $\vec D_0 \perp\vec k$, aber $\vec E_0 \not\perp\vec k$ in anisotropen Medien, da $\vec D=\varepsilon\vec E \nparallel\vec E$, weil $\varepsilon$ ein Tensor ist.

Die Ebene, die den Wellenvektor $k$ und das elektrische bzw. magnetische Feld enthält, wird als Polarisationsebene bezeichnet.
Die folgenden Ausführungen werden sich auf das elektrische Feld beschränken.
Das magnetische Feld verhält sich ganz analog, ist aber stets orthogonal zum elektrischen.

Es gibt verschiedene Arten von Polarisation:
\begin{itemize}
    \item \textbf{Lineare Polarisation:} Das elektrische Feld schwingt entlang einer gleichbleibenden Achse \textendash{} die Polarisationsebene ist also konstant, wie in Abbildung \Abbref{fig:lineare_polarisation} dargestellt. Das elektrische Feld ist demnach stets in die gleiche Richtung $\vec e$ gerichtet:
          \begin{align*}
              \vec E_0 = E_0 \vec e.
          \end{align*}
          Aus der Maxwell-Gleichung folgt für einen Ansatz wie \eqref{eq:ansatz_ebene_welle}, dass
          \begin{align*}
              \nabla\times\vec E                                         & = -\partial_t\vec B                                   \\
              i\vec k \times \vec E_0 e^{i(\vec k\cdot \vec r-\omega t)} & = i\omega\vec B_0 e^{i(\vec k\cdot \vec r-\omega t)},\end{align*}
          also
          \begin{align}
              \label{eq:b_feld_aus_e_feld}
              \boxed{\vec B_0 = \frac{1}{\omega}\vec k\times \vec E_0.}
          \end{align}
          \begin{figure}[htb]
              \centering
              \tfigLinearPolarisation
              \caption{Lineare Polarisation einer ebenen Welle: Die Feldvektoren des elektrischen und des magnetischen Feldes stehen orthogonal zum Wellenvektor $\vec k$, welcher die Ausbreitungsgeschwindigkeit kennzeichnet. Die Schwingungsebene des elektrischen und des magnetischen Feldes ist für lineare Polarisation jeweils konstant. }
              \label{fig:lineare_polarisation}
          \end{figure}


    \item \textbf{Zirkulare Polarisation:} Die Schwingungsebene rotiert mit der Winkelgeschwindigkeit $\omega$ und die $\vec k$-Achse (siehe \Abbref{fig:zirkulare_polarisation}). Die Rotation des $\vec E_0$-Vektors lässt sich folgendermaßen beschreiben:
          \begin{align*}
              \vec E_0 = E_0(\vec e_1\pm i\vec e_2)
          \end{align*}
          Dabei wird eine um \SI{90}{\degree} phasenverschobene Welle erzeugt, die entlang $\vec e_2$ polarisiert ist.
          Die beiden jeweils linear polarisierten Felder werden mit gleicher Stärke überlagert, sodass sich insgesamt der Realteil des Vektors $\vec E_0$ im Kreis bewegt.
          \begin{figure}[htb]
              \centering
              \tfigCircularPolarisation
              \caption{Zirkulare (links) und elliptische (rechts) Polarisation einer ebenen Welle: Der elektrische Feldvektor dreht sich auf einer Kreisbahn bzw. einer elliptischen Bahn mit der Kreisfrequenz $\omega$. }
              \label{fig:zirkulare_polarisation}
          \end{figure}


    \item \textbf{Elliptische Polarisation:} Die zirkulare Polarisation ist ein Sonderfall der elliptischen Polarisation, bei der die beiden überlagerten Felder eine unterschiedliche Stärke haben:
          \begin{align*}
              \vec E_0 = E_{01}\vec e_1 \pm i E_{02}\vec e_2
          \end{align*}

\end{itemize}




\subsection{Energie- und Impulsdichte}

Die mittlere Energiedichte ergibt sich aus der bereits bekannten Formel \eqref{eq:energiedichte} für die Energiedichte
\begin{align}
    \left\langle u\right\rangle & = \frac{1}{T} \int_0^T \frac{1}{2}\left(\vec E_\mathrm{R}\cdot \vec D_\mathrm{R}+\vec H_\mathrm{R}\cdot\vec B_\mathrm{R}\right)\diff t \nonumber \\
    \label{eq:mittlere_energiedichte}
    \left\langle u\right\rangle & = \frac{\varepsilon}{2}\left|\vec E_0\right|^2 = \frac{1}{2\mu}\left|\vec B_0\right|^2
\end{align}
mit reellen Feldern $\vec E_\mathrm{R}=\real(\vec E)$ usw.

Außerdem ist der mittlere Poynting-Vektor gegeben durch
\begin{align}
    \left\langle\vec S\right\rangle & = \frac{1}{t}\int_0^T\real(\vec E\times\vec H)\diff t \nonumber                                                   \\
    \label{eq:mittlerer_poyntingvektor}
    \left\langle\vec S\right\rangle & = \frac{1}{2} \sqrt{\frac{\varepsilon}{\mu}} \vec E_0^2 \hat{\vec k}= \left\langle u\right\rangle c \hat{\vec k},
\end{align}
die Energie wird folglich mit der Geschwindigkeit $c$ entlang $\vec k$ transportiert.

Die Impulsdichte einer elektromagnetischen kann im Vakuum mit
\begin{align}
    \label{eq:mittlere_impulsdichte}
    \left\langle\vec p_\mathrm{em}\right\rangle = \frac{1}{c^2}\left\langle\vec S\right\rangle = \frac{1}{c_0} \left\langle u\right\rangle \hat{\vec k} = \frac{1}{\omega} \left\langle u\right\rangle \vec k
\end{align}
bestimmt werden.

Für einen klassischen elektromagnetischen Puls ist der Photonenzahl $N$ nicht scharf. Ein Puls besteht im Mittel aus
\begin{align*}
    \left\langle N\right\rangle = \frac{1}{\hbar\omega} \int \left\langle u\right\rangle \diffa[3]{\vec r}
\end{align*}
Photonen. Einsetzen in \eqref{eq:mittlere_impulsdichte} liefert
\begin{align*}
    \vec p = \int\diffa[3]{\vec r}\left\langle\vec p_\mathrm{em}\right\rangle = \left\langle N\right\rangle \hbar\vec k
\end{align*}
für den Gesamtimpuls eines Lichtpulses.




\section{Reflexion, Transmission und Brechung}

Ziel dieses Kapitels ist es, den Übergang einer ebenen elektromagnetischen Welle an einer Trennfläche von zwei Medien zu beschreiben.

Betrachte eine ebene Welle, die an einer Grenzfläche reflektiert und gebrochen wird, wie in \Abbref{fig:reflection_refraction_plane_wave} abgebildet.
Die Grenzfläche mit Normalenvektor $\vec n$ trennt zwei Medien mit unterschiedlichen Permittivitäten $\varepsilon,\varepsilon'$ und Permeabilitätem $\mu,\mu'$ und somit verschiedenen Brechungsindizes $n$ und $n'$.
Die vom ersten Medium aus unter dem Winkel $\varphi$ auf die Grenzfläche einfallende Welle (1) hat den Wellenvektor $\vec k$.

Die Ebene, die den Wellenvektor $\vec k$ und den Normalenvektor $\vec n$ der Grenzfläche enthält, wird als Einfallsebene bezeichnet und der Tangentialvektor $\vec t$ der Grenzfläche wird so gewählt, dass er in der Einfallsebene liegt.
Wir haben bereits gesehen, dass beliebig polarisiertes Licht in zwei linear polarisierte Anteile zerlegt werden kann. Den Anteil, für den die Schwingungsrichtung des elektrischen Feldes in der Einfallsebene liegt, nennt man parallel bzw. p-polarisierten Anteil und den anderen senkrecht bzw. s-polarisiert.
Zur Vereinfachung wird in diesem Kapitel ohne Beschränkung der Allgemeinheit die $xz$-Ebene als Einfallsebene gewählt.

Ein Teil der Welle wird unter einem Winkel $\varphi'$ in das zweite Medium transmittiert. Der Wellenvektor der transmittierten Welle (2) soll $\vec k'$ genannt werden.
Der Rest wird unter einem Winkel $\psi$ und mit einem neuen Wellenvektor $\vec k''$ in das erste Medium zurückreflektiert (3).

\begin{figure}[htb]
    \centering
    \tfigReflectionRefractionPlaneWave
    \caption{Eine ebene Welle, die unter einem Winkel $\varphi$ auf eine Grenzfläche mit Normalenvektor $\vec n$ zwischen zwei Medien mit verschiedenen Brechungsindizes $n$ und $n'$ fällt,
        wird zum Teil unter dem Winkel $\psi$ in das erste Medium zurückreflektiert und der Rest wird unter dem Winkel $\varphi'$ in das zweite Medium transmittiert. Mit $\vec t$ wird außerdem der Tangentialvektor auf der Grenzfläche bezeichnet, der in der Einfallsebene (jene Ebene, die den Wellenvektor $k$ der einfallende Welle und das Lot enthält) liegt. }
    \label{fig:reflection_refraction_plane_wave}
\end{figure}

Alle Wellen sollen durch ebene Wellen beschrieben werden. Für die einfallende Welle gilt allgemein
\begin{align*}
    \vec E =\vec E_0 e^{i(\vec k\cdot\vec r-\omega t)}, \quad \vec B =\frac{n}{c_0}\hat{\vec k}\times\vec E, \quad\omega = ck=\frac{c_0}{n}k.
\end{align*}
Die transmittierte Welle lässt sich als
\begin{align*}
    \vec E' =\vec E_0' e^{i(\vec k'\cdot\vec r-\omega t)}, \quad \vec B'  =\frac{n'}{c_0}\hat{\vec k}'\times\vec E', \quad \omega = c'k'=\frac{c_0}{n'}k'
\end{align*}
schreiben und die reflektierte als (Achtung, gleicher Brechungsindex, wie für die einfallende, da das Medium das Gleiche ist)
\begin{align*}
    \vec E'' =\vec E_0'' e^{i(\vec k''\cdot\vec r-\omega t)}, \quad \vec B''  =\frac{n}{c_0}\hat{\vec k}''\times\vec E'', \quad k''  =k.
\end{align*}
Zur Lösung des Problems ist die Betrachtung der Randbedingungen bei $z=0$ nötig.
Diese gehen aus den vier Maxwellgleichungen hervor.
Zum einen müssen die Normalkomponeneten $\vec D$ und $\vec B$ stetig sein,
\begin{align}
    \label{eq:ebene_welle_RB_stetige_normalkomponenten}
    \begin{aligned}
        \nabla\cdot \vec D & =0 &  & \implication & \left(\vec D+\vec D''-\vec D'\right)\cdot \vec n & =0 \\
        \nabla\cdot \vec B & =0 &  & \implication & \left(\vec B+\vec B''-\vec B'\right)\cdot \vec n & =0
    \end{aligned}
\end{align}
und zum anderen müssen die Tangentialkomponenten von $\vec E$ und $\vec H$ stetig sein,
\begin{align}
    \label{eq:ebene_welle_RB_stetige_tangentialkomponenten}
    \begin{aligned}
        \nabla\times \vec E & = -\frac{1}{c}\frac{\partial\vec B}{\partial t} &  & \implication & \left(\vec E+\vec E''-\vec E'\right)\cdot \vec t & =0  \\
        \nabla\times \vec H & =\frac{\partial\vec D}{\partial t}              &  & \implication & \left(\vec H+\vec H''-\vec H'\right)\cdot \vec t & =0.
    \end{aligned}
\end{align}
Diese Bedingungen müssen für alle $x,t$ bei $z=0$ erfüllt sein ($y$ ist bereits 0, da als Einfallsebene die $xz$-Ebene gewählt wurde). Alle Wellen müssen demnach identische Phasen haben,
\begin{align}
    \label{eq:ebene_welle_identische_phasen}
    \boxed{\left.\vec k\cdot\vec r\right|_{z=0}=\left.\vec k'\cdot\vec r\right|_{z=0}=\left.\vec k''\cdot\vec r\right|_{z=0}.}
\end{align}
Daraus folgt, dass $\vec k$, $\vec k'$ und $\vec k''$ alle in einer Ebene liegen müssen, denn
\begin{align*}
    \vec k\cdot \vec e_y=0 \implication \vec k'\cdot \vec e_z=\vec k''\cdot \vec e_y=0.
\end{align*}
Außerdem lässt sich das Snelliussche Brechungsgesetz ableiten, das die Relation zwischen den Winkeln von einfallender und transmittierter Welle angibt.
Durch Projektion der Gleichung \eqref{eq:ebene_welle_identische_phasen} auf die $z=0$-Ebene erhält man
\begin{align*}
    k \sin\varphi = k' \sin\varphi' = k''\sin\psi,
\end{align*}
also
\begin{align}
    \label{eq:snelliussches_brechungsgesetz}
    \boxed{\frac{\sin\varphi}{\sin\varphi'}=\frac{k'}{k}=\frac{n'}{n}.}
\end{align}
Beim Übergang zum optisch dichteren Medium ($n'>n$) erfolgt die Brechung zum Lot hin, $\varphi'<\varphi$, beim Übergang zum opisch dünneren Medium vom Lot weg.
Ferner folgt wegen $k=k''$, dass Einfallswinkel und Ausfallswinkel (der reflektierten Welle) gleich sind:
\begin{align*}
    \varphi=\psi.
\end{align*}



\subsubsection{Die Fresnelschen Formeln}

Aus den Randbedingungen \eqref{eq:ebene_welle_RB_stetige_normalkomponenten} und \eqref{eq:ebene_welle_RB_stetige_tangentialkomponenten} können außerdem die sogenannten Fresnelschen Formeln hergeleitet werden, die die Reflektivität und Transmittivität abhängig vom Winkel angeben.
Dazu werden die elektrischen Feldvektoren aufgespaltet in
\begin{itemize}
    \item den s-polarisierten Teil, $\vec E_{0,\perp},\vec E_{0,\perp}',\vec E_{0,\perp}''$ (orthogonal zur Einfallsebene) mit
          \begin{align}
              \label{eq:fresnel_senkrecht}
              t_\perp = \frac{E_{0,\perp}'}{E_{0,\perp}}=\frac{2n\cos\varphi}{n\cos\varphi+n'\frac{\mu}{\mu'}\cos\varphi'},\quad r_\perp = \frac{E_{0,\perp}''}{E_{0,\perp}}=\frac{n\cos\varphi-n'\frac{\mu}{\mu'}\cos\varphi}{n\cos\varphi+n'\frac{\mu}{\mu'}\cos\varphi'}
          \end{align}
          Im optischen Bereich ist i.d.R. $\mu\approx\mu'\approx 1$.
    \item und den p-polarisierten Teil $\vec E_{0,\parallel},\vec E_{0,\parallel}',\vec E_{0,\parallel}''$ (in der Einfallsebene). Es gilt
          \begin{align}
              \label{eq:fresnel_parallel}
              t_\parallel = \frac{E_{0,\parallel}'}{E_{0,\parallel}}=\frac{2n\cos\varphi}{n'\frac{\mu}{\mu'}\cos\varphi+n\cos\varphi'},\quad r_\parallel\frac{E_{0,\parallel}''}{E_{0,\parallel}}=\frac{n'\frac{\mu}{\mu'}\cos\varphi-n\cos\varphi}{n'\frac{\mu}{\mu'}\cos\varphi+n\cos\varphi'}
          \end{align}
\end{itemize}

Der sogenannte Reflexionsgrad $R$ gibt das Verhältnis aus einfallender und reflektierter Intensität an und ergibt sich aus dem Betragsquadrat der Reflektivität,
\begin{align*}
    R=\left| \frac{E_0'}{E_0} \right|.
\end{align*}
Der Transmissionsgrad $T$ gibt entsprechend das Verhältnis aus einfallender und transmittierter Intensität an. Es gilt
\begin{align*}
    T=1-R.
\end{align*}
Die Reflektivitäten und Transmittivitäten, sowie der Reflexions- und Transmissionsgrad des senkrechten und des parallelen Anteils sind in \Abbref{fig:fresnel_equations}
über den Einfallswinkel $\varphi$ aufgetragen. Die Fälle für $n<n'$ und $n>n'$ unterscheiden sich.

Zuletzt ist für senkrechtes Licht $\varphi=\varphi'=\psi=0$ und damit nach den Fresnelschen Gleichungen
\begin{align*}
    \frac{E_{0,\perp}''}{E_{0,\perp}}=\frac{E_{0}''}{E_{0}} = \frac{n-n'}{n+n'} < 0
\end{align*}
für $n'>n$. Da das Verhältnis von einfallendem und reflektiertem Feld kleiner als $0$ ist, findet ein Vorzeichenwechsel des elektrischen Felds statt.
Dies entspricht einem Phasensprung um $\pi$ bzw. \SI{180}{\degree} bei der Reflexion am optisch dichteren Medium.
% Wird am optisch dünneren Medium reflektiert, also $n'<n$, dann findet der Phasensprung

\begin{figure}[H]
    \centering
    \tfigFresnelEquations
    \caption{Darstellung der Fresnelschen Formeln: Die Amplitudenkoeffizienten $r$ und $t$ sowie der Reflexionsgrad $R$ und der Transmissionsgrad $T$ sind über den Einfallswinkel $\varphi$ aufgetragen, oben für die Reflexion am optisch dichteren Medium und unten am optisch dünneren Medium und jeweils für den senkrecht und den parallel polarisierten Anteil. Bei der Reflexion am dichteren Medium sinkt der Reflexionsgrad für den parallelen Anteil am Brewsterwinkel auf 0 ab und bei Reflexion am dünneren Medium steigt der Reflexionsgrad auf 1, wenn der Winkel der Totalreflexion erreicht wird. Für die Funktionsgraphen wurde $n=1$ und $n'=\num{1,8}$ gewählt. }
    \label{fig:fresnel_equations}
\end{figure}


\subsubsection{Der Brewsterwinkel}

Wie in \Abbref{fig:fresnel_equations} zu sehen, fällt die Amplitude des p-polarisierten reflektierten Strahls für einen bestimmten Winkel (welcher von den Brechungsindizes abhängt) auf $0$ ab.
Dieser Winkel wird als Brewsterwinkel $\varphi_\mathrm{B}$ bezeichnet. Aus der Bedingung, dass das reflektierte Feld verschwindet, $E_{0,\parallel}''=0$ folgt mit Gleichung \eqref{eq:fresnel_parallel}, dass
\begin{align*}
    n' \cos\varphi_\mathrm{B} = n \cos\varphi'
\end{align*}
und näherungsweise
\begin{align}
    \label{eq:brewsterwinkel}
    \varphi_\mathrm{B} = \arctan\frac{n'}{n}.
\end{align}
Zum Beispiel liegt für den Übergang von Luft ($n\approx 1$) zu Glas $n'\approx \num{1,5}$ der Brewsterwinkel bei $\varphi_\mathrm{B}=\SI{56}{\degree}$.
Da unter dem Brewsterwinkel reflektiertes Licht keine parallele Komponente aufweist, ist der reflektierte Strahl vollständig s-polarisiert, was eine Anwendung in Polarisatoren ermöglicht.

\begin{figure}[htb]
    \centering
    \tfigBrewsterAngle
    \caption{Brewsterwinkel: Die Reflexion von unter dem Brewsterwinkel auf ein optisch dichteres Medium einfallende Wellen ist vollständig s-polarisiert. }
    \label{fig:brewsterwinkel}
\end{figure}

\subsubsection{Die Totalreflexion}

Beim Übergang zu einem optisch dünneren Medium ($n'<n$) wird wie bereits erwähnt der transmittierte Strahl vom Lot weg gebrochen, der Transmissionswinkel $\varphi'$ ist also größer als der Einfallswinkel $\varphi$.
Vergrößert man den Einfallswinkel, erreicht der Transmissionswinkel schließlich \SI{90}{\degree} und die Welle wird totalreflektiert.
Dann ist
\begin{align}
    \label{eq:totalreflexion}
    \varphi_\mathrm{T} = \arcsin\frac{n'}{n}.
\end{align}
Für noch größere Einfallswinkel $\varphi>\varphi_\mathrm{T}$ ist
\begin{align*}
    \sin\varphi'=\frac{n}{n'}\sin\varphi>1
\end{align*}
und
\begin{align*}
    \cos\varphi'=\sqrt{1-\sin^2\varphi'}=i\sqrt{\left(\frac{n}{n'}\right)^2\sin^2\varphi-1}=i\sqrt{\frac{\sin^2\varphi}{\sin^2\varphi_\mathrm{T}}-1}.
\end{align*}
Einsetzen in den Ansatz der ebenen Welle liegert dann
\begin{align*}
    e^{i\vec k'\cdot \vec r} = e^{ik' \sin\varphi' x+ik'\cos\varphi' z}  = e^{ik' \sin\varphi' x}e^{ik'\cos\varphi' z} = e^{ik' \sin\varphi' x}e^{-\frac{z}{\delta}}
\end{align*}
mit Eindringtiefe
\begin{align*}
    \delta=\frac{1}{k'} \frac{\sin\varphi_\mathrm{T}}{\sqrt{\sin^2\varphi-\sin^2\varphi_\mathrm{T}}}.
\end{align*}
Der total reflektierte Strahl dringt also exponentiel gedämpft in das optisch dünnere Medium ein. Man spricht von der evaneszenten Welle bzw. dem evaneszenten Feld.

Dieser Effekt findet Anwendung in Wellenleitern (z.B. Glasfaser), in der Röntgenoptik (Reflexion unter streifendem Einfall) und in der Mikroskopie,
wo am zu vermessenden Objekt seitlich ein Strahl gestreut wird und aus der seitlichen Verschiebung des totalreflektierten Strahls der Abstand gemessen werden kann.




\section{Dynamische Theorie der Dielektrizitätskonstanten}

In dispersiven Materialien ist die Dielektrizitätskonstante abhängig von der Frequenz,
\begin{align*}
    \varepsilon=\varepsilon(\omega).
\end{align*}
Dies ist in gewissen Maße bei allen Medien außer Vakuum der Fall. Allerdings können Medien in bestimmten Frequenzbereichen praktisch nicht-dispersiv sein.

Die Frequenzabhängigkeit der Dielektrizitätskonstante hat einige wichtige Auswirkungen:
\begin{itemize}
    \item Die Phasengeschwindigkeit $c$ ist von der Frequenzabhängig, sodass Wellenpakete, die aus verschiedenen Frequenzkomponenten bestehen, im Raum zerlaufen:
          \begin{align*}
              c(\omega) = \frac{c_0}{n(\omega)} = \frac{c_0}{\sqrt{\varepsilon_r(\omega)}}
          \end{align*}
    \item Es wird sich zeigen, dass der Imaginärteil $\imag\varepsilon(\omega)$ ungleich 0 ist, was eine Absorption von Energie durch das Medium zur Folge hat (siehe Kapitel \ref{sec:satz_von_poynting}).
    \item Insgesamt ist $\varepsilon(\omega)$ komplex, $\varepsilon(\omega) = \varepsilon'(\omega) + i\varepsilon''(\omega)$ und es gibt einen formalen Zusammenhang zwischen $\varepsilon'(\omega)$ und $\varepsilon''(\omega)$. Dieser Zusammenhang ist als Kramers-Kronig-Relation bekannt und wird in dem Abschnitt \ref{sec:kramers_kronig_relation} behandelt.
\end{itemize}

Es gibt verschiedene Theoriemodelle, mithilfe derer die Dielektrizitätskonstante quanitativ beschrieben wird.
In Abschnitt \ref{sec:lorentzmodell_gebundene_elektronen} wird das Lorentzmodell für gebundene Elektronen disktutiert und in Abschnitt \ref{sec:freie_elektronen} die Dielektrizitätskonstante bei freien Elektronen behandelt.


Zunächst soll eine grobe Einordnung des Problems erfolgen.
In Kapitel \ref{sec:Makroskopische_Gleichungen_der_Elektrostatik} haben wir die dielektrische Verschiebung $\vec D$ auf verschiedene Weisen beschrieben:
\begin{align*}
    \vec D= \varepsilon \vec E=\varepsilon_0 \vec E+\vec P=\varepsilon_0 \underbrace{(1+\chi)}_{\varepsilon _r }\vec E
\end{align*}
Die Suszeptibilität $\chi$ hat ihren Ursprung im atomaren oder molekularen Dipolmoment $\vec p=\varepsilon_0 \alpha\vec E$ und bewirkt eine makroskopische Polarisation
$\vec P=N\vec p = \varepsilon_0 N\alpha\vec E = \varepsilon\chi\vec E$. Hier beschreibt $N$ die Dipoldichte und $\alpha$ die Polarisierbarkeit.

Wir haben ferner gesehen, dass die Suszeptibilität in dichten Systemen durch
\begin{align*}
    \chi =\frac{N\alpha}{1-\frac{1}{3}N\alpha}
\end{align*}
gegeben ist, weil $\vec E_\mathrm{lokal}\neq\vec E$.

Es gibt verschiedene Beiträge zu der Polarisierbarkeit, die durch unterschiedliche Mechanismen zustande kommen.
\begin{enumerate}
    \item Elektronische Dipolmomente bewirken eine Verschiebungspolarisation.
    \item Ähnlich funktionieren ionische Dipolmomente, die aber durch die Ladungstrennung in Ionen entsteht.
    \item Die Orientierungspolarisation ist die Polarisation, die durch die Ausrichtung von molekularen, statistisch verteilten Dipolen entsteht.
\end{enumerate}

Bei welchen Frequenzen diese Beiträge eine Rolle spielen ist in \Abbref{fig:polarisierbarkeit_ueber_spektrum} grob schematisch skizziert.
Die Idee dabei ist, dass verschiedene Beträge charakteristische Frequenzen haben mit
\begin{align*}
    \omega_\mathrm{O}<\omega_\mathrm{I}<\omega_\mathrm{e}
\end{align*}
($\omega_\mathrm{O}$ für die Orientierungspolarisation, $\omega_\mathrm{I}$ für ionische Dipolmomente und $\omega_\mathrm{e}$ für elektronische Dipolmomente).
Für viel größere Frequenzen als die höchste charakteristische Frequenz sinkt die Polarisierbarkeit auf $0$.

\begin{figure}[htb]
    \centering
    \tfigPolarizabilityOverSpectrum
    \caption{Polarisierbarkeit skizzenhaft aufgetragen über das elektromagnetische Spektrum von Radiowellen bis zum UV-Bereich. }
    \label{fig:polarisierbarkeit_ueber_spektrum}
\end{figure}



\subsection{Lorentzmodell für gebundene Elektronen\label{sec:lorentzmodell_gebundene_elektronen}}

Das Lorentzmodell beschreibt die mit den auftreffenden elektrischen Feldern wechselwirkenden Elektronen als klassische harmonische Oszillatoren.
Das Elektron erfährt eine lineare, rücktreibende Anziehungskraft durch den Atomkern $\vec F_\mathrm{A} = -m\omega_0^2 \vec r$ mit Eigenfrequenz $\omega_0$ (mit einer Feder vergleichbar),
eine von der Geschwindigkeit abhängige Reibungskraft $\vec F_\mathrm{R}=-m\gamma\dot{\vec r}$ mit Dämpfungskonstante $\gamma$ und zuletzt die externe Kraft, die durch ein äußeres, mit Frequenz $\omega$ oszillierendes elektrisches Feld gegeben ist.
Diese äußere Kraft schreiben wir als $-e\vec E_0(\omega)e^{-i\omega t}$

Insgesamt gilt also
\begin{align*}
    \vec F_\mathrm{ges} & =\vec F_\mathrm{A} + \vec F_\mathrm{R} + \vec F_\mathrm{ext}               \\
    m\ddot{\vec r}      & = -m\omega_0^2 \vec r-m\gamma\dot{\vec r} -e\vec E_0(\omega)e^{-i\omega t}
\end{align*}
bzw.
\begin{align*}
    m\left(\ddot{\vec r}+\gamma\dot{\vec r} +\omega_0^2 \vec r\right) & = -e\vec E_0(\omega)e^{-i\omega t}.
\end{align*}
Diese inhomogene Differentialgleichung zweiten Grades wird mit einem Exponentialansatz $\vec r(t)=\vec r(\omega)e^{-i\omega t}$ gelöst:
\begin{align*}
    \vec r(t) = -\frac{e}{m} \frac{1}{\omega_0^2-\omega^2-i\gamma\omega} \vec E_0(\omega)e^{-i\omega t}
\end{align*}
Für das Dipolmoment $\vec p(t)=\vec p(\omega)e^{-i\omega t} = -e\vec r(t)$ folgt
\begin{align*}
    \vec p(\omega) = e\vec r(\omega)= -\frac{e^2}{m} \frac{1}{\omega_0^2-\omega^2-i\gamma\omega} \vec E_0(\omega) = \varepsilon_0 \alpha(\omega)\vec E_0(\omega).
\end{align*}
Im Allgemeinen gibt es pro Atom viele Elektronen mit Eigenfrequenzen $\omega_i$ und Dämpfungen $\gamma_i$, was sich durch die Überlagerung von mehreren harmonischen Oszillatoren der Stärke $f_i$ modellieren lässt\footnote{Die Oszillatorstärken $f_i$ addieren sich bei Atomen zu der Kernladungszahl $Z$ auf.}.
Die makroskopische Polarisation ist dann (mit Anzahl $N$ der Atome)
\begin{align*}
    P = \sum_i p_i = \frac{Ne^2}{m}\sum_i  \frac{f_i}{\omega_i^2-\omega^2-i\gamma_i\omega} \vec E_0(\omega)e^{-i\omega t}.
\end{align*}
Da aber $\vec P=\varepsilon_0 \chi \vec E$, ist
\begin{align*}
    \chi = \frac{Ne^2}{\varepsilon_0 m} \sum_i \frac{f_i}{\omega_i^2-\omega^2-i\gamma_i\omega}
\end{align*}
und
\begin{align*}
    \varepsilon_r = 1+\chi = 1+ \frac{Ne^2}{\varepsilon_0 m} \sum_i \frac{f_i}{\omega_i^2-\omega^2-i\gamma_i\omega}.
\end{align*}
Eine Zerlegung in Real- und Imaginärteil liefert
\begin{align*}
    \varepsilon_r'(\omega)  & = 1 + \frac{Ne^2}{\varepsilon_0 m} \sum_i f_i\frac{\omega_i^2-\omega^2}{\omega_i^2-\omega^2-i\gamma_i\omega} \\
    \varepsilon_r''(\omega) & = \frac{Ne^2}{\varepsilon_0 m} \sum_i f_i\frac{ \gamma_i\omega}{\omega_i^2-\omega^2-i\gamma_i\omega} .
\end{align*}
Für nur einen einzigen Oszillator beschreibt deren Verlauf eine typische Resonanzkurve, wie in \Abbref{fig:dielectric_constant_real_and_imaginary_one_resonance} dargestellt. 


\begin{figure}[htb]
    \centering
    \tfigDielectricConstantRealPart
    \tfigDielectricConstantImaginaryPart
    \caption{<caption>}
    \label{fig:dielectric_constant_real_and_imaginary_one_resonance}
\end{figure}







\subsection{Beitrag freier Elektronen\label{sec:freie_elektronen}}
\subsection{Kramers Kronig Relationen\label{sec:kramers_kronig_relation}}
