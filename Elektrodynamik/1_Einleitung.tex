% !TEX root = Theo_III.tex

\chapter{Einleitung\label{einleitung}}

\section{Geschichte}

\begin{itemize}
	\item \textbf{1785} \textendash{} Charles Augustin de Coulomb: Entdeckung des Coulombsches Gesetzes.

	\item \textbf{1800} \textendash{} Alessandro Volta: Erfindung der erstern Batterie, die Voltasche Säule.

	\item \textbf{1820} \textendash{}  Hans Christian \O{}rsted: Das \O{}rstedsche Gesetz beschreibt, dass elektrische Ströme ein Magnetfeld erzeugen.

	\item \textbf{1820-25} \textendash{}   André-Marie Ampère: Entdeckung der Grundlagen der Magnetostatik durch Messungen.

	\item \textbf{1831} \textendash{}  Michael Faraday: Beschreibung der magnetischen Induktion.

	\item \textbf{1852} \textendash{}  Michael Faraday: Formulierung des Nahwirkungsstandpunktes (Beschreibung elektrischer Phänomene über Felder statt Kräfte).

	\item \textbf{1864} \textendash{} James Clerk Maxwell: Formulierung der Maxwell-Gleichungen als fundamentale Feldgleichungen des elektromagnetischen Feldes und Nutzung von elektrischen und magnetischen Hilfsfeldern für die physikalische Beschreibung in Materie sowie Äußerung der Vermutung, dass Licht eine elektromagnetische Welle ist.

	\item \textbf{1886} \textendash{} Heinrich Hertz: Nachweis elektromagnetischer Wellen und Postulierung eines Äthers als hypothetisches Ausbreitungsmedium.

	\item \textbf{1881} \textendash{} Michelson-Morley-Experiment: Konstanz der Lichtgeschwindigkeit unabhängig von Beobachter und Quelle ${\Rightarrow}$ ein absolutes Bezugssystem Äther existiert nicht.

	\item \textbf{1905} \textendash{} Albert Einstein: spezielle Relativitätstheorie.

\end{itemize}





\section{Inhalt}

Der Inhalt dieser Vorlesung gliedert sich in folgende Abschnitte:

\begin{itemize}
	\item Einleitung

	\item Elemente der Vektoranalysis

	\item Elektrostatik

	\item Elektrische Felder in Materie

	\item Magnetostatik

	\item Grundgleichungen der Elektrodynamik: Die Maxwellschen Gleichungen

	\item Spezielle Relativitätstheorie

	\item Ebene elektromagnetische Wellen

	\item Elektromagnetische Felder bei vorgegebenen Ladungen und Strömen
\end{itemize}



\section{Grundlegende Konstanten der Elektrodynamik}

Für Konstanten deren Wert per Definition festgelegt wurde, wird ein $\equiv $-Zeichen verwendet.


\begin{table}[H]
	\centering
	\begin{tabular}{|l|l|} \hline
		\textbf{Konstante}         & \textbf{Wert}                                                     \\\hline
		Vakuumlichtgeschwindigkeit & \centering\arraybackslash{}$c_{0}\equiv \SI{299792458}{\m\per\s}$ \\
		Elektrische Feldkonstante  & $\varepsilon _{0}=\SI{8,8541878128e-12}{\A\s\per\V\per\m}$        \\
		Magnetische Feldkonstante  & $\mu _{0}=\SI{1,25663706212e-6}{\N\per\square\A}$                 \\
		\hline
	\end{tabular}
\end{table}




\section{Grundlegende Formeln der Elektrodynamik}

Maxwellsche Feldgleichungen:
\begin{align*}
	\nabla \cdot \vec {D}                        & =\rho _{f}                                                       &
	\nabla \cdot \vec {B}                        & =0                                                               &
	\nabla \times \vec {E}                       & =-\frac{\partial \vec {B}}{\partial t}                           &
	\nabla \times \vec {H}                       & =\vec {j}_{f}+\frac{\partial \vec {D}}{\partial t}                 \\
	\oint_A D \cdot\diffa{\vec{A}}               & =Q                                                               &
	\oint_A B \cdot\diffa{\vec{A}}               & =0                                                               &
	\oint_{\partial A}\vec E\cdot \diffa{\vec s} & =    -\int_A\frac{\partial\vec B}{\partial t}\cdot\diffa{\vec A} &
	\oint_{\partial A}\vec H\cdot \diffa{\vec s} & = I + \int_A\frac{\partial\vec D}{\partial t}\cdot\diffa{\vec A}
\end{align*}
% \begin{table}[H]
% 	\centering\begin{tabular}{llll}
% 		$\displaystyle \nabla \times \vec {E}=-\frac{\partial \vec {B}}{\partial t} $                                                  &
% 		$\displaystyle \nabla \cdot \vec {B}=0 $                                                                                       &
% 		$\displaystyle \nabla \times \vec {E}=-\frac{\partial \vec {B}}{\partial t} $                                                  &
% 		$\displaystyle \nabla \times \vec {H}=\vec {j}_{f}+\frac{\partial \vec {D}}{\partial t} $                                        \\ \\
% 		$\displaystyle \oint_A D \cdot\diffa{\vec{A}}=Q $                                                                              &
% 		$\displaystyle \oint_A B \cdot\diffa{\vec{A}}=0 $                                                                              &
% 		$\displaystyle \oint_{\partial A}\vec E\cdot \diffa{\vec s} =    -\int_A\frac{\partial\vec B}{\partial t}\cdot\diffa{\vec A} $ &
% 		$\displaystyle \oint_{\partial A}\vec H\cdot \diffa{\vec s} = I + \int_A\frac{\partial\vec D}{\partial t}\cdot\diffa{\vec A} $
% 	\end{tabular}
% 	\label{<label>}
% \end{table}

Materialgleichungen (allgemein und in linearen, isotropen Medien (Vakuum: $\epsilon_r=\mu_r=1$):
\begin{align*}
	\vec {D} & =\varepsilon _{0}\vec {E}+\vec {P}        & \vec {B} & =\mu _{0}\left(\vec {H}+\vec {M}\right) \\
	\vec {D} & =\varepsilon _{0}\varepsilon _{r}\vec {E} & \vec {B} & =\mu _{0}\mu _{r}\vec {H}
\end{align*}

Statische Potentiale:
\begin{align*}
	\phi(\vec r)   & =\frac{1}{4\pi\epsilon_0}\int \frac{\rho(\vec r')}{\left|\vec r-\vec r'\right|}\diffa[3]{\vec r'} &
	\vec A(\vec r) & =\frac{\mu_0}{4\pi}\int \frac{\vec j(\vec r')}{\left|\vec r-\vec r'\right|}\diffa[3]{\vec r'}       \\
	\vec E         & =-\nabla \phi                                                                                     &
	\vec B         & =\nabla \times\vec A                                                                              &
\end{align*}

Biot-Savart-Gesetz historisch (für unendlich lange, gerade Leiter) und modern:
\begin{align*}
	B(\rho) = \frac{\mu_0}{2\pi}\frac{I}{\rho} \qquad B(\vec r)=\frac{\mu_0}{4\pi}\int\vec j(\vec r')\times \frac{\vec r-\vec r'}{\left|\vec r-\vec r'\right|^3}\diffa[3]{\vec r'}
\end{align*}

Feldenergie und Feldenergiedichte:
\begin{align*}
	u(\vec r) & =\frac{1}{2}\vec E\cdot \vec D                      & u(\vec r) & =\frac{1}{2}\vec H\cdot \vec B                       \\
	U(\vec r) & =\frac{1}{2}\int\vec E\cdot \vec D\diffa[3]{\vec r} & U(\vec r) & =\frac{1}{2}\int\vec H\cdot \vec B \diffa[3]{\vec r}
\end{align*}


Multipole:
\begin{align*}
	p            & = \int\rho(\vec r)\vec r'\diffa[3]{\vec r'}                                                                                   & m                            & =\frac{1}{2}\int \vec r'\times\vec j(\vec r)	\diffa[3]{\vec r'} \\
	Q_{ij}       & =\int (3x_i'x_j'-\delta_{ij} {\vec r'}^2)\diffa[3]{\vec r'}                                                                   &                              &                                                                \\
	\phi(\vec r) & = \frac{1}{4\pi\epsilon_0}\left(\frac{q}{r}+\frac{\vec p\cdot \vec r}{r^3}+\frac{1}{2}Q_{ij} \frac{x_ix_j}{r^5}+\ldots\right) & \vec A_\text{Dipol} (\vec r) & =\frac{\mu_0}{4\pi} \frac{\vec m\times \vec r}{r^3}
\end{align*}

Randbedingungen an Grenzflächen:
\begin{align*}
	\hat{\vec n}\times(\vec E^{(1)}-\vec E^{(2)}) & = 0      & \hat{\vec n}\times(\vec H^{(1)}-\vec H^{(2)}) & = \vec k\cdot \vec m \\
	\hat{\vec n}\cdot(\vec D^{(1)}-\vec D^{(2)})  & = \sigma & \hat{\vec n}\cdot(\vec B^{(1)}-\vec B^{(2)})  & = 0
\end{align*}

Lorentzkraft und Lorentzkraft-Dichte:
\begin{align*}
	\vec F_\text{L} = q(\vec E+\vec v\times \vec B) = \int \vec f_\text{L}\diffa[3]{\vec r}, \qquad \vec f_\text{L}=\rho\vec E+\vec j\times \vec B
\end{align*}

Kontinuitätsgleichung
\begin{align*}
	\frac{\partial w}{\partial t}+\nabla\cdot \vec S & = -\vec j\cdot \vec E & w & =\frac{1}{2}\left(\vec E\cdot \vec D+\vec H\cdot \vec B\right) &
	\vec S                                           & =\vec E\times \vec H
\end{align*}

Relationen von Lichtgeschwindigkeit, Feldkonstanten und Brechungsindex:
\begin{align*}
	c_{0} & =\frac{1}{\sqrt{\varepsilon _{0}\mu _{0}}} & n & =\frac{c_{0}}{c} & \omega^2 & =c^2 k^2
\end{align*}
In linearen, isotropen Medien gilt
\begin{equation*}
	n=\sqrt{\varepsilon\mu}
\end{equation*}

Vektoranalysis
\begin{align*}
	\nabla\cdot (\vec a\times \vec b)   & =\vec b\cdot(\nabla\times\vec a)-\vec a\cdot(\nabla\times\vec b)                                             \\
	\nabla\times (\vec a\times  \vec b) & =\vec a(\nabla\cdot \vec b)-(\vec a\cdot \nabla)\vec b+(\vec b\cdot \nabla)\vec a-\vec b(\nabla\cdot \vec a) \\
	\nabla \times (\nabla\times \vec a) & =\nabla(\nabla\cdot \vec a) - \Delta\vec a                                                                   \\
	\nabla\times(\vec a\times \vec b f) & = f\nabla\times(\vec a\times  \vec b)-(\vec a\times  \vec b)\times \nabla f                                  \\
	\nabla\cdot(f\vec a)                & = \vec a\cdot (\nabla f)+f\nabla \cdot \vec a
\end{align*}
\begin{align*}
	\nabla \frac{1}{r} & =-\frac{\vec {r}}{r^{3}} & \Delta \frac{1}{r} & =-4\pi\delta(\vec r)
\end{align*}

Operatoren in Kugelkoordinaten
\begin{align*}
	\nabla            & =\vec e_r \frac{\partial}{\partial r}+\frac{1}{r}\vec e_\theta \frac{\partial}{\partial \theta} + \frac{1}{r\sin\theta}\vec e_\varphi \frac{\partial}{\partial\varphi}                                                                                                                    \\
	\nabla\cdot\vec F & = \frac{1}{r^2} \frac{\partial}{\partial r} (r^2 F_r)+ \frac{1}{r \sin\theta} \frac{\partial}{\partial\theta}(\sin\theta F_\theta)+\frac{1}{r\sin\theta} \frac{\partial}{\partial\varphi}F_\varphi                                                                                        \\
	\nabla^2          & =\frac{1}{r^2} \frac{\partial}{\partial r}\left(r^2 \frac{\partial}{\partial r}\right) +\frac{1}{r^2}\left[ \frac{1}{\sin\theta} \frac{\partial}{\partial\theta}\left(\sin\theta \frac{\partial}{\partial\theta}\right)+\frac{1}{\sin^2\theta}\frac{\partial^2}{\partial\varphi^2}\right]
\end{align*}

Operatoren in Zylinderkoordinaten
\begin{align*}
	\nabla             & = \vec e_\rho \frac{\partial}{\partial\rho} + \frac{1}{\rho}\vec e_\varphi \frac{\partial}{\partial\varphi}+\vec e_z \frac{\partial}{\partial z}                                    \\
	\nabla\cdot \vec F & = \frac{1}{\rho} \frac{\partial}{\partial\rho} \left(\rho F_\rho\right) + \frac{1}{\rho} \frac{\partial}{\partial\varphi}F_\varphi + \frac{\partial}{\partial z} F_z                \\
	\Delta             & = \frac{1}{\rho} \frac{\partial}{\partial\rho} \left(\rho \frac{\partial}{\partial\rho}\right)+\frac{1}{\rho^2}\frac{\partial^2}{\partial\varphi^2}+\frac{\partial^2}{\partial z^2}
\end{align*}
