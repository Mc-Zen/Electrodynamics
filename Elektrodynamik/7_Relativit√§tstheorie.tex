% !TEX root = Theo_III.tex


\chapter{Die spezielle Relativitätstheorie}

Die spezielle Relativitätstheorie revidierte die Vorstellung von Raum und Zeit, die seit Galileo Galilei gültig war.

Bei ihrer Entwicklung spielte die Elektrodynamik und die Lichtausbreitung im Äther eine große Rolle.
Schon vor der Relativitätstheorie entwickelte Hendrik A. Lorentz die sogenannten Lorentz-Transformationen für das elektrische und magnetische Feld und Henri Poincaré untersuchte die mathematischer Struktur dieser Lorentz-Transformation und zeigte die Invarianz der Maxwell-Gleichungen unter diesen Transformationen.

Diese Bemühungen galten aber nur zur Bewahrung der klassischen Sichtweise und erst Einstein veröffentlichte 1905 einen völlig neuen Ansatz.

\section{Vor der Relativitätstheorie}

Bevor Albert Einstein die Relativitätstheorie einführte, stellte man sich den physikalischen Raum als euklidischen (also flachen) Raum in drei Dimensionen vor mit Abständen $d=\sqrt{x^2+y^2+z^2}$ zum Ursprung, Längen $v=\left|\vec v\right|=\sqrt{\vec v\cdot \vec v}$ von Vektoren und Winkeln $\cos\alpha=\vec v\cdot \vec w/(vw)$ zwischen zwei Vektoren $\vec v$ und $\vec w$, während die Zeit getrennt davon und in allen Bezugssystemen gleich vergeht.
Man spricht auch von einer absoluten Zeit.

In der Newtonschen Mechanik spielen Inertialsysteme eine große Rolle.
Dabei handelt es sich um besondere Bezugssysteme, die sich alle gleichförmig zueinander bewegen und nicht beschleunigt sind.
In ihnen gelten die Newtonschen Axiome.

Nach dem Galileischen Relativitätsprinzip sind alle Inertialsystem gleichwertig.

\begin{formal}
    Die Newtonschen Axiome sind forminvariant (kovariant) unter Galilei-Transformationen zwischen verschiedenen Inertialsystemen.
\end{formal}



\subsection{Die Galilei-Transformation}

Die allgemeine Galilei-Transformation von einem Inertialsystem $\Sigma$ in ein anderes Inertialsystem $\Sigma'$ nimmt die Form
\begin{alignat*}{3}
    \vec r & \rightarrow & \:\vec r' & = R(\vec r-\vec vt-\vec a) \\
    t      & \rightarrow & t'        & =t-t_0
\end{alignat*}
an. Die Matrix $R$ beschreibt eine (zeitlich konstante) Rotation, $\vec v$ eine gleichförmige Bewegung zwischen den Systemen und $\vec a$ eine konstante Verschiebung. Zusammen mit der zeitlichen Verschiebung $t_0$ ergeben sich insgesamt zehn Parameter, die eine Galilei-Transformation beschreiben.

\begin{formal}
    Die Galilei-Transformationen bilden eine Gruppe mit zehn Parametern bezüglich Hintereinanderausführung.
\end{formal}


Die Addition von Geschwindigkeiten erfolgt linear,
\begin{equation*}
    \vec u' = \vec u - \vec v,
\end{equation*}
denn (hier zur Vereinfachung unter Vernachlässigung der Drehung)
\begin{align*}
    \vec u = \frac{\diff\vec r}{\diff t} \implication \vec u' = \frac{\diff\vec r'}{\diff t} = \frac{\diff\vec r}{\diff t}-\vec v= \vec u - \vec v.
\end{align*}
Das heißt, dass ein Körper, der sich im Inertialsystem $\Sigma$ mit der Geschwindigkeit $\vec u$ bewegt, in dem dazu mit $\vec v$ bewegten Inertialsystem $\Sigma'$ die Geschwindigkeit $\vec u'=\vec u-\vec v$ aufweist.


\subsection{Lichtausbreitung}

Wir wissen, dass Licht eine elektromagnetische Welle ist. Es gilt daher für die Ausbreitung die in den vorigen Kapiteln hergeleitete Wellengleichung für das elektrische Feld (hier für $\rho=0$ und $\vec j=0$):
\begin{align}
    \label{eq:wellengleichung_e}
    \left(\nabla^2-\frac{1}{c^2}\frac{\partial^2}{\partial t^2}\right) \vec E=0, \quad c^2=\frac{1}{\varepsilon_0\mu_0}
\end{align}
Diese wird zum Beispiel durch eine ebene Welle $\vec E=\vec E_0 e^{i(\vec k\cdot r-\omega t)}$ mit $\omega=ck$ gelöst. Wegen $\divg \vec E =0$ handelt es sich um eine Transversalwelle, $\vec E\perp \vec k$. Die Ausbreitungsgeschwindigkeit dieser elektromagnetischen Welle ist also $c$ und wird daher auch als Lichtgeschwindigkeit bezeichnet.

Allerdings ergibt sich im Rahmen der klassischen Physik ein Problem, da nicht klar ist, in welchem Inertialsystem $c$ gelten soll.

Die Maxwell-Gleichungen und damit auch die Wellengleichung \eqref{eq:wellengleichung_e} sind nämlich nicht invariant unter der Galilei-Transformation.

Außerdem hat man sich vorgestellt, dass sich Licht wie mechanische Wellen in einem elastischen Medium \textendash{} einem sogenannten Äther \textendash{} bewegt. Dieser durfte ferner keine Longitudinalwellen erlauben, da bereits bekannt war, dass Licht stets eine Transversalwelle ist.
Nach dieser Vorstellung gibt es ein ausgezeichnetes Inertialsystem, in dem der Äther ruht und die Lichtgeschwindigkeit gerade $c$ ist, während in dazu bewegten Systemen die Lichtgeschwindigkeit größer oder kleiner ist.




\subsection{Michelson-Morley-Experiment}

1881 führte Albert A. Michelson ein Experiment durch, das diesen Äther nachweisen sollte. 1887 wiederholte er das Experiment zusammen mit Edward W. Morley mit höherer Präzision und erhielt schließlich 1907 den Nobelpreis dafür.

Die Grundidee für diesen Versuch basiert auf der Vorstellung, dass der Lichtäther im Inertialsystem der Sonne ruht und sich die Erde bei der Rotation um die Sonne durch ihn hindurchbewegt (siehe \Abbref{fig:bewegung_durch_lichtaetherA}). Da die Umlaufgeschwindigkeit um die Sonne mit $v=\SI{30}{\kilo\m\per\s}$ groß ist, sollte sich entlang der Bewegungsrichtung der Erde um die Sonne eine andere Lichtgeschwindigkeit messen lassen als quer dazu.

\begin{figure}[htp]
    \centering
    \tfigBewegungDurchLichtaetherA
    \caption{Die Erde bewegt sich relativ zum Lichtäther. Die Achsen werden so gewählt, dass die relative Bewegung entlang $\vec e_1$ stattfindet. Wegen der Geschwindigkeitsaddition, die durch die Galilei-Transformation vorgegeben ist, wird erwartet, dass entlang $\vec e_1$ eine andere Lichtgeschwindigkeit gemessen wird, also entlang $\vec e_2$. }
    \label{fig:bewegung_durch_lichtaetherA}
\end{figure}

Bewegt sich ein Lichtstrahl im Äther mit der Geschwindigkeit $c\vec n$ und die Erde mit $\vec v = v\vec e_1$ durch den Äther, so ist nach der Geschwindigkeitsaddition der Galilei-Transformation die Geschwindigkeit des Lichtstrahls auf der Erde $\vec c'=c\vec n-\vec v$. Dabei gibt es zwei Spezialfälle für die Messrichtung, wie in \Abbref{fig:bewegung_durch_lichtaetherB} dargestellt:

\begin{figure}[htp]
    \centering
    \tfigBewegungDurchLichtaetherB
    \caption{Sonderfälle für die Messung der Lichtgeschwindigkeit bei der Bewegung durch den Lichtäther. Links: $\vec c'\parallel\vec e_1$, es gilt $|\vec c'|=c\pm v$, rechts: $\vec c'\parallel \vec e_2$, es gilt $|\vec c'|=\sqrt{c^2-v^2}$.}
    \label{fig:bewegung_durch_lichtaetherB}
\end{figure}

\begin{enumerate}
    \item $\vec c'\parallel \vec e_1$: Geschwindigkeit im Äther: $\pm c \vec e_1$, Geschwindigkeit auf der Erde: $\pm(c\mp v)\vec e_1$, also $|\vec c'|=\pm v$.
    \item $\vec c'\parallel \vec e_2$: Geschwindigkeit im Äther: $\pm c \vec e_2$, also $|\vec c'|=\sqrt{c^2-v^2}$.
\end{enumerate}


Zwar ist die Lichtgeschwindigkeit sehr hoch, aber die Geschwindigkeitsabweichungen sollten sich mithilfe von Interferenz beobachten lassen. In Abbildung \Abbref{fig:michelson_interferometer} ist schematisch das von Michelson entworfene Interferometer abgebildet, mithilfe dessen das Experiment durchgeführt wurde.

\begin{figure}[htb]
    \centering
    \tfigMichelsonInterferometer
    \caption{Schematische Abbildung eines Michelson-Interferometers. Eine Lichtquelle (idealerweise ein Laser) wird auf einen Strahlteiler gerichtet. Die beiden Teilstrahlen werden an jeweils einem Spiegel (S1 und S2) zurück durch den Strahlteiler reflektiert. Ein Teil beider Strahlen wird in Richtung eines Detektors (D) oder Schirms gelenkt. Die Abstände des Strahlteilers zu den beiden Spiegeln sind idealerweise gleich. }
    \label{fig:michelson_interferometer}
\end{figure}

Das Interferometer wird so ausgerichtet, dass ein Arm entlang $\vec e_1$ und der andere entlang $\vec e_2$ zeigt, wie dargestellt.
Obwohl jetzt der Weg, den die beiden Teilstrahlen auf ihrem Weg zu und von den Spiegeln zurücklegen, der gleiche ist (sofern $l_1=l_2$), sollten die Laufzeiten $t_1$ und $t_2$ unterschiedlich sein, weil die Lichtgeschwindigkeit wie oben beschrieben verschiedene Werte für die beiden Achsen annimmt:
\begin{align*}
    t_1 & = \frac{l_1}{c+v}+\frac{l_1}{c-v}=\frac{2l_1}{c\left(1-\frac{v^2}{c^2}\right)} \\
    t_2 & = \frac{2l_2}{\sqrt{c^2-v^2}} = \frac{2l_2}{c\sqrt{1-\frac{v^2}{c^2}}}
\end{align*}
In der Praxis lassen sich natürlich $l_1$ und $l_2$ nicht exakt gleich wählen (zumindest eine Präzision in der Größenordnung der Wellenlänge wäre notwendig). Stattdessen wird das Interferometer um \SI{90}{\degree} gedreht und beobachtet, ob sich das Interferenzmuster ändert.
Überraschenderweise wurde eine solche Änderung aber nicht festgestellt.

Um diesen Widerspruch zu der Äthertheorie aufzulösen, schlugen H. A. Lorentz und G. F. Fitzgerald vor, dass sich materielle Objekte entlang der Bewegungsrichtung durch den Äther um einen Faktor $\sqrt{1-\frac{v^2}{c^2}}$ verkürzen,
sodass die Länge $l_1$ in \Abbref{fig:michelson_interferometer} zu $l_1\sqrt{1-\frac{v^2}{c^2}}$ wird\footnote{Im Gegensatz zur Längenkontraktion in der speziellen Relativitätstheorie galt diese allerdings nicht allgemein, sondern wirklich nur für materielle Objekte.}.

Dagegen war Albert Einsteins Idee, anzunehmen, dass die Lichtgeschwindigkeit immer gleich ist.




\section{Einsteinsches Relativitätsprinzip}

Albert Einstein kannte zwar wahrscheinlich das Michelson-Morley-Experiment nicht, aber er war überzeugt, dass ein absolutes Bezugssystem (in dem der Äther ruht) nicht existieren kann.

\begin{formal}
    Jedes Inertialsystem ist gleichermaßen berechtigt zur Beschreibung der physikalischen Gesetze. Die physikalischen Gesetze sind kovariant unter Lorentz-Transformationen.
\end{formal}

Insbesondere gilt\footnote{Bei der 17. Generalkonferenz für Maß und Gewicht am 20.10.1983 wurde der Wert der Lichtgeschwindigkeit auf exakt \SI{299 792 458}{\m\per\s} festgelegt.
    Mit der Definition für die Sekunde (\SI{1}{\s} ist das 9192631770-fache der Periodendauer der beim Übergang zwischen den beiden Hyperfeinstruktur-Niveaus des Grundzustandes von Atomen des Nuklids $\isotope[133]{Cs}$ ausgesandten Strahlung) ergibt sich auch die Definition des Meters: \SI{1}{\m} ist die Länge, welche das Licht im Vakuum während des Zeitintervalls $(1/299792458)\,\si{\s}$ durchläuft.}:

\begin{formal}
    Die Lichtgeschwindigkeit $c$ ist unabhängig vom Inertialsystem.
\end{formal}

Dieses Relativitätsprinzip geht über die Galileische Relativität der Mechanik hinaus.
Die Maxwell-Gleichungen sind invariant unter Lorentz-Transformationen und damit in allen Inertialsystemen gültig.

Um das Postulat von der Konstanz der Lichtgeschwindigkeit durchzusetzen, ist es allerdings notwendig, Raum und Zeit neu zu beschreiben.
In der klassischen Mechanik ist der Raum ein euklidischer und vollständig unabhängig von der Zeit. In dem neuen Minkowski-Raum wird er allerdings mit der Zeit zur vierdimensionalen Raumzeit verbunden.

Das Relativitätsprinzip soll an einem kleinen Beispiel veranschaulicht werden. Betrachte einen Zug der mit hoher Geschwindigkeit durch einen Bahnhof fährt.
Von der Mitte des Zuges aus wird nun ein Lichtsignal nach vorne und hinten ausgesandt. Ein Betrachter, der sich im Zug befindet, wird bemerken, dass der Lichtstrahl exakt gleichzeitig am Anfang und am Ende des Zuges ankommt, da der Weg für beide sich gleichförmig bewegende Strahlen genau der Gleiche ist.
Ein am Bahnhof still stehender Beobachter sieht aber, dass das Signal zuerst das Ende des Zuges erreicht, weil sich dieses dem Lichtstrahl entgegen bewegt, während der Anfang von dem Signal davonläuft.
Dieses Paradoxon kommt zustande, weil die Lichtgeschwindigkeit für alle Beobachter die gleiche sein muss und es wird dadurch erklärt, dass die Gleichzeitigkeit von Ereignissen relativ ist.
Zwei Ereignisse wie das Ankommen der Lichtstrahlen an den Zugenden können in einem Inertialsystem zeitgleich stattfinden und einem anderen zu unterschiedlichen Zeiten.

In den folgenden Kapiteln sollen diese Phänomen quantitativ untersucht werden.
Dazu wird zunächst die Lorentz-Transformation eingeführt, welche in der relativistischen Physik die Galilei-Transformation ersetzt.


\section{Die Lorentz-Transformation}

In der speziellen Relativitätstheorie beschreibt die Lorentz-Transformation die Transformation zwischen zwei Inertialsystemen.
Wie die Lorentz-Transformation aussieht, lässt sich auf verschiedene Arten herleiten, von welchen hier der direkte Weg von Einstein dargestellt wird.
Zuerst wollen wir aber noch einige Vorbemerkungen zur Raumzeit machen.


\subsection{Invarianz des Lichtkegels}

Wir wollen zunächst mit einem Gedankenexperiment beginnen, bei dem sich ein Lichtpuls ab dem Zeitpunkt $t_0=0$ vom Ursprung ausbreitet.
Der Lichtpuls wird von zwei Inertialsystemen aus betrachtet, $\Sigma$ und $\Sigma'$, wie in \Abbref{fig:srt_gedankenexperiment_lichtkegel} dargestellt (vereinfacht, ohne $z$-Achse).

\begin{figure}[htb]
    \centering
    \tfigSRTGedankenExperimentLichtkegel
    \caption{Zwei Inertialsysteme $\Sigma$ und $\Sigma'$ fallen zum Zeitpunkt $t_0=0$ zusammen. Ihre Relativgeschwindigkeit ist $\vec v$, sodass sich die beiden Systeme zum Zeitpunkt $t_0+\diff t>0$ ein Stück weit auseinander bewegt haben.
    Ein Lichtstrahl (rot) wird zum Zeitpunkt $t_0$ aus dem gemeinsamen Ursprung emittiert.
    Zum Zeitpunkt $t_0+\diff t$ hat er im Inertialsystem $\Sigma$ den Weg $\sqrt{\diff x^2+\diff y^2}$ zurückgelegt, während er im Inertialsystem $\Sigma'$ den Weg $\sqrt{{\diff x'}^2+{\diff y'}^2}$ zurückgelegt hat. }
    \label{fig:srt_gedankenexperiment_lichtkegel}
\end{figure}

Nachdem im System $\Sigma$ eine Zeit $\diff t$ verstrichen ist, hat der Lichtstrahl den Weg $\sqrt{\diff x^2+\diff y^2+\diff z^2}$ zurückgelegt.
Wegen $c=s/t\equivalence c^2 t^2 = \vec r^2$ ist also
\begin{align*}
    -c^2 \diff t^2+\underbrace{\diff x^2+\diff y^2+\diff z^2}_{\diff\vec r^2}=0.
\end{align*}
Wir betrachten jetzt das zweite Inertialsystem $\Sigma'$, das sich mit der Relativgeschwindigkeit $\vec v$ zu $\Sigma$ bewegt und zu $t=0$ mit $\Sigma$ zusammenfällt.
Hier ist der vom Lichtstrahl zurückgelegte Weg $\diff{\vec r'}^2$.
Da aber die Lichtgeschwindigkeit $c$ diesselbe sein muss, folgt, dass eine andere Zeit verstrichen sein muss, die Zeit also anders als in $\Sigma$ verläuft.
Wir erhalten damit im System $\Sigma'$
\begin{align*}
    -c^2 \diff{t'}^2+\underbrace{\diff{x'}^2+\diff{y'}^2+\diff{z'}^2}_{\diff{\vec r'}^2}=0.
\end{align*}
Wir sehen folglich, dass die Größe
\begin{align*}
    -c^2\diff t^2+\diff r^2
\end{align*}
in jedem beliebigen Inertialsystem $\Sigma$ gleich $0$ ist, wenn die Ausbreitung eines Lichtstrahls betrachtet wird.
Der Gedanke liegt nahe, dass $-c^2\diff t^2+\diff r^2$ auch für andere Geschwindigkeiten als die Lichtgeschwindigkeit invariant bezüglich Wechseln des Inertialsystems ist.
Daher wird diese Größe als Norm im Minkowski-Raum (bzw. als Minkowski-Norm) bezeichnet.


\begin{figure}[htb]
    \centering
    \tfigSRTLichtkegel
    \caption{Relativistischer Lichtkegel}
    \label{fig:srt_lichtkegel}
\end{figure}


\subsection{Der Minkowski-Raum}

Im euklidischen Raum werden Punkte durch drei Raumkomponenten beschrieben, $\vec r=(x,y,z)$.
Die Länge eines Vektors wird durch die euklidische Norm ausgedrückt\footnote{Eigentlich ist die euklidische Norm die Wurzel aus $\Delta s^2$.}:
\begin{align}
    \label{eq:euklidische_norm}
    \Delta=x^2+y^2+z^2=\sum_i x_i^2=x_ig_{ij}x_j
\end{align}
mit dem metrischen Tensor $g_{ij}=\delta_{ij}$, also
\begin{align*}
    g = \begin{pmatrix}
            1 & 0 & 0 \\
            0 & 1 & 0 \\
            0 & 0 & 1
        \end{pmatrix}.
\end{align*}
Die Norm $\Delta s^2$ ist invariant gegenüber beliebigen Rotationen des Koordinatensystems und ist daher ein Skalar im Sinne des Tensorkalküls\footnote{Erinnerung: Ein Skalar ist in der Tensorrechung eine Größe, die sich beim Transformieren in andere Koordinatensysteme nicht ändert. }.
Ferner gilt die Gleichung \eqref{eq:euklidische_norm} auch für Differenzvektoren $\Delta\vec r=\vec r_2-\vec r_1$.

Im Minkowski-Raum werden Punkte der vierdimensionalen Raumzeit jetzt durch sogenannte kontravariante Vierervektoren (notiert durch hochgestellte, meist griechische Indizes) beschrieben:
\begin{align*}
    \begin{pmatrix} ct \\ \vec r \end{pmatrix} = \begin{pmatrix} ct \\ x \\ y \\ z \end{pmatrix}   = \begin{pmatrix} x^0 \\ x^1 \\ x^2 \\ x^3 \end{pmatrix}
\end{align*}
mit Komponenten
\begin{align*}
    x^\alpha,\quad \alpha=0,1,2,3.
\end{align*}
Die Minkowski-Norm ist dann
\begin{align*}
    \Delta s^2=x^\alpha g_{\alpha\beta}x^\beta=-(x^0)^2+(x^1)^2+(x^2)^2+(x^3)^2
\end{align*}
mit dem metrischen Tensor $g_{\alpha\beta}$ mit Elementen von
\begin{align*}
    g = \begin{pmatrix}
            -1 & 0 & 0 & 0 \\
            0  & 1 & 0 & 0 \\
            0  & 0 & 1 & 0 \\
            0  & 0 & 0 & 1
        \end{pmatrix}.
\end{align*}
Über den metrischen Tensor lässt sich auch der sogenannte kovariante Vierervektor definieren:
\begin{align*}
    x_\alpha=g_{\alpha\beta}x^\beta, \quad \begin{pmatrix} x_0 \\ x_1 \\ x_2 \\ x_3 \end{pmatrix}=\begin{pmatrix} -x^0 \\ x^1 \\ x^2 \\ x^3 \end{pmatrix}.
\end{align*}
Damit lässt sich die Minkowski-Norm schreiben als
\begin{align*}
    \Delta s^2=x^\alpha x_\alpha.
\end{align*}
In diesem neuen Raum suchen wir jetzt eine Transformation, die die Minkowski-Norm erhält \textendash{} die sogenannte Lorentz-Transformation
\begin{align*}
    {\Delta s'}^2 = \Delta s^2 \equivalence g=\Lambda^T g\Lambda.
\end{align*}



\subsection{Die Lorentz-Transformation}

Die allgemeinste Form der Lorentz-Transformation ist
\begin{align*}
    x^{\alpha'}=\tensor{\Lambda}{^i_j}x^\alpha+a^{\alpha'},
\end{align*}
wobei man für $a^{\alpha'}=0$ die homogene Lorentz-Transformation erhält und für $a^{\alpha'}$ die inhomogene bzw. Poincaré-Transformation.

Diese Form hat genau wie die Galilei-Transformation zehn Parameter (drei für die Geschwindigkeit, drei für die Rotation und vier für die Zeit- und Raumtranslationen).
Außerdem bilden diese Transformationen eine Gruppe (Poincaré-Gruppe) bezüglich Hintereinanderausführung.

Die spezielle Lorentz-Transformation ergibt sich für den Spezialfall eines Geschwindigkeitsboosts in $x$-Richtung,
\begin{align}
    \label{eq:spezielle_lorentz_trafo}
    \begin{split}
        x'  & = \gamma(x-\beta ct)   \\
        ct' & = \gamma (ct-\beta x),
    \end{split}
\end{align}
wobei $\beta=v/c$ definiert wird. Die Herleitung folgt im Unterkapitel \ref{sec:herleitung_1D_lorentz_trafo}.

Mithilfe von Matrizen lässt sich die spezielle Lorentz-Transformation zu
\begin{align*}
    \begin{pmatrix}ct' \\ x' \end{pmatrix} = \begin{pmatrix}\gamma&-\beta\gamma \\ -\beta\gamma &\gamma \end{pmatrix} \begin{pmatrix}ct \\ x \end{pmatrix}
\end{align*}
umschreiben, also
\begin{align*}
    \left(\begin{array}{c|ccc}
              \gamma       & -\beta\gamma & 0 & 0 \\
              \hline
              -\beta\gamma & \gamma       & 0 & 0 \\
              0            & 0            & 1 & 0 \\
              0            & 0            & 0 & 1
          \end{array}\right)
    =\left(\begin{array}{c|ccc}
               \gamma                      &  & -\frac{\gamma}{c}v\vec{e}_1^T           & \\
               \hline
                                           &  &                                         & \\
               -\frac{\gamma}{c}v\vec{e}_1 &  & 1 + (\gamma-1)\vec{e}_1\otimes\vec{e}_1 & \\
                                           &  &                                         &
           \end{array}\right).
\end{align*}

Für allgemeine Geschwindigkeiten $\vec v$ mit Komponenten in alle Raumrichtungen gilt
\begin{align*}
    \Lambda_{\vec v}=
    \left(\begin{array}{c|ccc}
                  \gamma                   &  & -\frac{\gamma}{c}\vec{v}^T                    & \\
                  \hline
                                           &  &                                               & \\
                  -\frac{\gamma}{c}\vec{v} &  & 1 + \frac{\gamma-1}{v^2}\vec{v}\otimes\vec{v} & \\
                                           &  &                                               &
              \end{array}\right).
\end{align*}
Die Rücktransformation lautet einfach
\begin{align*}
    \Lambda^{-1}_{\vec v}= \Lambda_{-\vec v}.
\end{align*}
Eine noch allgemeinere Form erhält man durch Anwenden einer (rein räumlichen) Rotation,
\begin{align*}
    \Lambda = \Lambda_R\Lambda_{\vec v}, \quad \Lambda_R=\begin{pmatrix}
                                                             1      & \vec 0^T \\
                                                             \vec 0 & R
                                                         \end{pmatrix},
\end{align*}
wobei $R$ eine orthogonale $3\times 3$-Matrix ist.



\subsection{Herleitung der eindimensionalen Lorentz-Transformation\label{sec:herleitung_1D_lorentz_trafo}}

Zur Vereinfachung soll nur eine Relativgeschwindigkeit $v$ in $x$-Richtung betrachtet werden.
Die anderen Raumkoordinaten transformieren also mit
\begin{align*}
    y'=y, \quad z'=z.
\end{align*}
Wir erwarten eine lineare Transformation, die wir zunächst allgemein beschreiben:
\begin{align}
    \label{eq:herleitung_lorentz_lin_trafo}
    \begin{pmatrix} ct' \\ x' \end{pmatrix} = \begin{pmatrix} a_{11} & a_{12} \\ a_{21} & a_{a_22} \end{pmatrix} \begin{pmatrix} ct \\ x \end{pmatrix}
\end{align}
bzw.
\begin{align}
    \label{eq:herleitung_lorentz_lin_trafo_ct}
    ct' & = a_{11} ct + a_{12}x \\
    \label{eq:herleitung_lorentz_lin_trafo_x}
    x'  & = a_{21} ct + a_{22}x
\end{align}
Der Ursprung des bewegten Systems $x'=0$ bewegt sich im ruhenden System gleichförmig,
\begin{align}
    0              & = a_{21} ct + a_{22}x \nonumber                       \\
    \equivalence x & = -\frac{a_{21}}{a_{22}}ct \overset{!}{=}vt \nonumber \\
    \label{eq:herleitung_lorentz_v1}
    \equivalence v & = -\frac{a_{21}}{a_{22}} c.
\end{align}
Einseten in \eqref{eq:herleitung_lorentz_lin_trafo_x} liefert
\begin{align}
    \label{eq:herleitung_lorentz_xprime}
    x'=a_{22} (x-vt).
\end{align}
Analog lässt sich die Bewegung des Ursprungs $x=0$ des ruhenden Systems im bewegten System beschreiben:
\begin{align}
    ct'            & = a_{11}ct \nonumber                                                \\
    x'             & = a_{21}ct = \frac{a_{21}}{a_{11}}ct' \overset{!}{=} -vt' \nonumber \\
    \label{eq:herleitung_lorentz_v2}
    \equivalence v & =-\frac{a_{21}}{a_{11}}c
\end{align}
Durch Vergleich von \eqref{eq:herleitung_lorentz_v1} und \eqref{eq:herleitung_lorentz_v2} erhält man
\begin{align*}
    a_{11}=a_{22}
\end{align*}
und damit
\begin{align}
    \label{eq:herleitung_lorentz_ctprime}
    ct'=a_{22}\left(ct+\frac{a_{12}}{a_{22}}x\right).
\end{align}
Setzt man nun die Gleichungen \eqref{eq:herleitung_lorentz_xprime} und \eqref{eq:herleitung_lorentz_ctprime} in die Bedingungn der Invarianz der Minkowski-Norm
\begin{align*}
    x^2-(ct)^2 = {x'}^2-(ct')^2
\end{align*}
ein, so erhält man
\begin{align*}
    x^2-(ct)^2 = \underbrace{\left(a_{22}^2-a_{12}^2\right)}_{=1}x^2 - \underbrace{\left(a_{22}^2-a_{22}^2 \frac{v^2}{c^2}\right)}_{=1}(ct)^2 + \underbrace{\left(-a_{22}a_{12}-a_{22}^2 \frac{v}{c}\right)}_{=0}2xct
\end{align*}
Durch Koeffizientenvergleich erhält man
\begin{align*}
    a_{22}^2 \left(1-\frac{v^2}{c^2}\right) = 1 \implication a_{22} = \frac{1}{\sqrt{1-\frac{v^2}{c^2}}}
\end{align*}
und
\begin{align*}
    a_{12} = -a_{22} \frac{v}{c} \implication a_{12} = -\frac{v}{c \sqrt{1-\frac{v^2}{c^2}}}.
\end{align*}
Daraus ergibt sich die Lorentz-Transformation
\begin{align*}
    x'  & = \gamma(x-\beta ct)  \\
    ct' & = \gamma(ct- \beta x)
\end{align*}
mit
\begin{align*}
    \beta=\frac{v}{c}, \quad \gamma=\frac{1}{\sqrt{1-\beta^2}}.
\end{align*}

\subsection{Geschwindigkeitsaddition}

Betrachte zwei Inertialsysteme $\Sigma$ und $\Sigma'$, die gleichförmig mit der Relativgeschwindigkeit  $v$ zueinander bewegt sind, wie in \Abbref{fig:Geschwindigkeitsaddition} dargestellt.
Eine Punktmasse bewegt sich im System $\Sigma'$ mit der Geschwindigkeit $u$.
Um nun die Geschwindigkeit der Masse im System $\Sigma$ zu bestimmen, wird die (spezielle, in $x$-Richtung) Lorentz-Transformation
\eqref{eq:spezielle_lorentz_trafo} invertiert, indem $v$ durch $-v$ (bzw. $\beta$ durch $-\beta$ in $\beta=v/c$) ersetzt wird:
\begin{align}
    \label{eq:spezielle_inv_lorentz_trafo}
    \begin{split}
        x  & =\gamma(x'+\beta ct') \\
        ct & =\gamma(ct'+\beta x')
    \end{split}
\end{align}
\begin{figure}[htp]
    \centering
    \tfigGeschwindigkeitsaddition
    \caption{Geschwindigkeitsaddition: Das Inertialsystem $\Sigma'$, in dem sich eine Punktmasse mit der Geschwindigkeit $u$ bewegt, hat eine Relativgeschwindigkeit $v$ zum Inertialsystem $\Sigma$. In diesem wird für die Punktmasse die Geschwindigkeit $w=\frac{x}{t}=\frac{u+v}{1+\frac{uv}{c^2}}$ gemessen. }
    \label{fig:Geschwindigkeitsaddition}
\end{figure}

Für die Punktmasse, die sich nach dem Gesetz $x'=ut'$ bewegt gilt also
\begin{align*}
    x & =\gamma t'(u+\beta c)              \\
    t & =\gamma t'(ct'+\beta \frac{u}{c}),
\end{align*}
sodass die Geschwindigkeit in $\Sigma$ durch
\begin{align*}
    w=\frac{x}{t}=\frac{u+v}{1+\frac{uv}{c^2}}
\end{align*}
berechnet wird.

Besonders bemerkenswert sind der Grenzfall $u,v\ll c$, für den die klassische Näherung $w=u+v$ gilt, sowie der Extremfall $u=c$ mit $w=c$, der zeigt, dass die Lichtgeschwindigkeit tatsächlich die maximale Geschwindigkeit ist.




\section{Die vierdimensionale Raumzeit der speziellen Relativitätstheorie}

\subsection{Raumzeit-Diagramme}

Raumzeit-Diagramme bzw. Minkowski-Diagramme dienen zur Veranschaulichung von Bewegungen und Ereignissen in der Raumzeit.
Zur Vereinfachung wird nur eine Raumdimension berücksichtigt.

\begin{figure}[htp]
    \centering
    \tfigMinkowskiDiagramA
    \caption{Weltlinie im Raumzeit-Diagramm mit Ortskoordinate auf der horizontalen und Zeitkoordinate auf der vertikalen Achse.
        Der Bereich $ct>0$ liegt in der Zukunft, der Bereich $ct<0$ in der Vergangenheit.
        Zeitartige Ereignisse und raumartige Ereignisse werden durch den Lichtkegel getrennt, wobei die zeitartigen innerhalb und die raumartigen außerhalb liegen.
        Teilchen bewegen sich auf Weltlinien (grün) innerhalb des Lichtkegels. Gestrichelt ist auch der Lichtkegel für einen Punkt auf der Weltlinie angedeutet. }
    \label{fig:minkowski_diagram}
\end{figure}

In \Abbref{fig:minkowski_diagram} ist ein einfaches Raumzeit-Diagramm dargestellt mit der Raumkoordinate $x$ auf der horizontalen Achse und der Zeitkoordinate $ct$ auf der vertikalen Achse.
Punkte in diesem Diagramm sind Ereignisse, die wie folgt klassifiziert werden (in Referenz zum Ursprung, also dem Punkt mit $x=0$ und $t=0$):

\begin{itemize}
    \item Punkte mit $ct>0$ liegen in der Zukunft vom Raumzeitpunkt im Urspung.
    \item Punkte mit $ct<0$ liegen in der Vergangenheit vom Raumzeitpunkt im Urspung.
    \item Diejenigen Ereignisse, die sich mit Lichtgeschwindigkeit ausbreiten, liegen auf dem Lichtkegel (violett) mit
          \begin{align*}
              \Delta s^2=x^2-(ct)^2=0
          \end{align*}
          und heißen lichtartige Ereignisse. Diese Ereignisse können den Ursprung nur mit Lichtgeschwindigkeit erreichen oder von ihm erreicht werden.
    \item Zeitartige Ereignisse liegen im Lichtkegel, also
          \begin{align*}
              \Delta s^2 = x^2-(ct)^2 < 0 \equivalence x^2 < (ct)^2.
          \end{align*}
          Zeitartige Ereignisse sind kausal mit dem Ursprung verbunden und können die Geschehnisse im Ursprung beinflussen (falls $ct<0$, Vergangenheit) oder sind durch den Ursprung
          beeinflussbar (falls $ct>0$, Zukunft). Die Trajektorien, auf der sich Teilchen durch die Raumzeit bewegen, heißen Weltlinien und liegen immer im Lichtkegel, da sich Teilchen nicht mit
          Überlichtgeschwindigkeit bewegen können. Eine solche Weltlinie ist auch in \Abbref{fig:minkowski_diagram} abgebildet.
    \item Raumartige Ereignisse liegen außerhalb des Lichtkegels, also
          \begin{align*}
              \Delta s^2 = x^2-(ct)^2 > 0 \equivalence x^2 > (ct)^2.
          \end{align*}
          Diese sind nicht kausal mit dem Ursprung verbunden.
\end{itemize}

Außerdem ist der Lichtkegel eines anderen Punktes in der Raumzeit gestrichelt dargestellt.
Auf der Weltlinie beschreibt das Innere des dort gültigen Lichtkegels immer alle Raumzeitpunkte, die erreicht werden könnten bzw. kausal verbunden sind.


Mit Raumzeit-Diagrammen lässt sich auch die Lorentz-Transformation
\begin{align*}
    ct' & =\gamma(ct-\beta x) \\
    x'  & =\gamma(x-\beta ct)
\end{align*}
veranschaulichen, siehe \Abbref{fig:minkowski_diagramB}.

\begin{figure}[htp]
    \centering
    \tfigMinkowskiDiagramB
    \caption{Lorentz-Transformation im Raumzeit-Diagramm: Die schwarzen Koordinatenachsen $(ct,x)$ beschreiben das ruhende System und die grünen Achsen ein bewegtes Inertialsystem $(ct',x')$.
    Punkte mit gleicher Zeit $ct'=\mathrm{const}$ liegen auf Geraden, die parallel zur $x'$-Achse sind und Punkte mit gleicher Raumkoordinate $x'=\mathrm{const}$ auf Parallelen zur $ct'$-Achse.
    Die Achsen des bewegten Systems sind gegen die des ruhenden Systems um den Winkel $\tan\delta=\beta$ gedreht, sodass der Lichtkegel als Winkelhalbierende mit dem Lichtkegel des ruhenden Systems zusammenfällt.
    Die Skalierung der Achsen eines bewegten Systems wird durch die Schnittpunkte mit den Hyperbeln $x^2-(ct)^2={x'}^2-(ct')^2=\pm1$ beschrieben. }
    \label{fig:minkowski_diagramB}
\end{figure}

Hier sind die Achsen des Ruhesystems $\Sigma$ mit $(ct,x)$ und eines relativ dazu bewegten Inertialsystems $\Sigma$ mit $(ct',x')$ dargestellt.
Man erhält die $ct'$-Achse aus der Lorentz-Transformation durch Setzen von $x'=0$. Man bekommt die Geradengleichung $ct=x/\beta$.
Analog erhält man für die $x'$-Achse $ct=\beta x$.
Die Achsen des bewegten Systems sind also um den Winkel $\delta$ gegen die Achsen von $\Sigma$ gedreht, wobei
\begin{align*}
    \tan\delta=\beta=\frac{v}{c}.
\end{align*}
Da die Drehwinkel beider Achsen gleich sind, ist auch der Lichtkegel für beide und damit alle Inertialsysteme gleich.
Ob ein Ereignis licht-, zeit- oder raumartig ist, ist demnach unabhängig vom Inertialsystem.

Zunkunft und Vergangenheit sind zwar relativ (da zum Beispiel der Bereich $ct>0$ ungleich dem Bereich $ct'>0$), aber die Kausalität wird nicht verletzt,
weil alle Punkte im Lichtkegel für alle Inertialsysteme eindeutig und kohärent der Vergangenheit bzw. der Zukunft zugeordnet werden.

Auch ist leicht zu sehen, dass die Gleichzeitigkeit relativ ist, denn die Linie $ct'=\mathrm{const}$ fällt nicht mit $ct=\mathrm{const}$ zusammen.

Zuletzt soll noch der Einheitsmaßstab erwähnt werden.
Im euklidischen Raum liegen alle Punkte mit gleichem Abstand $1$ zu einem Mittelpunkt auf einem Einheitskreis, $x^2+y^2+z^2=1$.
Im Minkowski-Raum dagegen beschreiben Lösungen von
\begin{align*}
    x^2-(ct)^2={x'}^2-(ct')^2=\pm1
\end{align*}
Hyperbeln (zweischalige Hyperboloide für drei Dimensionen).



\subsection{Zeitdilatation und Längenkontraktion}

Zwei prominente Beispiele für Effekte, die aus der speziellen Relativitätstheorie folgen, sind die Zeitdilatation und die Längenkontraktion.
Beide sollen anhand eines illustrativen Beispiels erläutert werden.

Wenn die kosmische Strahlung auf die Schichten der Atmosphäre trifft, entstehen in ungefähr \SI{10}{\km} Höhe sogenannte Myonen \textendash{} Elementarteilchen, die sich dann mit nahezu Lichtgeschwindigkeit (ca. $\num{0,9995}c$) auf die Erdoberfläche zu bewegen.
Allerdings ist ihre Lebensdauer mit $\tau=\SI{2,2e-6}{\s}$ sehr gering \textendash{} so gering, dass nach Zurücklegen der Strecke bis zum Erdboden nur noch ca. \SI{0,000026}{\percent} der Myonen nicht zerfallen wäre, was sich nicht mit den gemachten Messungen deckt (Detektion einer sehr hohen Rate von Myonen).

Um dieses Problem zu lösen, müssen die Effekte der spezielle Relativitätstheorie berücksichtigt werden, weil bei dieser Geschwindigkeit der Lorentzfaktor schon bei \num{31,63} liegt.
Legt man die Erde als ruhendes System $\Sigma$ fest, vergeht von der Entstehung der Myonen bis zum Auftreffen die Zeit nach den Gleichungen \eqref{eq:spezielle_lorentz_trafo}
\begin{align}
    \label{eq:myonen_x}
    \Delta x'  & = \gamma(\Delta x-\beta c\Delta t) \overset{!}{=} 0 \\
    \label{eq:myonen_t}
    c\Delta t' & = \gamma(c \Delta t-\beta \Delta x)
\end{align}
Die Myonen bewegen sich in ihrem System $\Sigma'$ nicht, daher ist $\Delta x'= 0$.
Einsetzen von \eqref{eq:myonen_x} in \eqref{eq:myonen_t} liefert
\begin{align}
    \label{eq:zeitdilatation}
    \Delta t' = \frac{1}{\gamma}\Delta t.
\end{align}
Da $1/\gamma < 1$, ist
\begin{align*}
    \Delta t' < \Delta t,
\end{align*}
die Zeit vergeht also im bewegten System langsamer. 
Dadurch leben die Myonen aus Sicht des Systems der Erde länger und erreichen den Erdboden in hoher Zahl,
denn während die Myonen innerhalb ihrer mittleren Lebensdauer klassisch nur $0,995c\cdot\SI{2,2e-6}{\s}=\SI{656}{\m}$ zurücklegen,
reicht die Zeit relativistisch für $0,995c\cdot\SI{2,2e-6}{\s}/\gamma\approx\SI{20}{\km}$.


Aus Sicht der Myonen muss das Problem anders gelöst werden, da ja die Zeit für die Myonen nicht langsamer vergeht
(aus Gründen der Symmetrie muss sogar aus Sicht der Myonen stattdessen die Zeit im Erdsystem langsamer vergehen, was allerdings für diesen Effekt keine Rolle spielt).
Wir untersuchen einmal die Länge der Strecke, die die Myonen zurückzulegen haben, aus deren System. Dafür setzen wir $\Delta t'=0$ und erhalten
\begin{align}
    \Delta x'              & = \gamma(\Delta x-\beta c\Delta t)                   \nonumber\\
    c\Delta t'             & = \gamma(c \Delta t-\beta \Delta x) \overset{!}{=} 0 \nonumber\\
    \label{eq:laengenkontraktion}
    \implication \Delta x' & = \frac{1}{\gamma}\Delta x,
\end{align}
es ist also
\begin{align*}
    \Delta x'< \Delta x,
\end{align*}
für die Myonen ist der Weg durch die Atmosphäre kürzer. 

Zusammenfassend:
\begin{formal}
    Bewegte Uhren gehen langsamer (aus Sicht des ruhenden Systems). Also messen ruhende Systeme das dilatierte Zeitintervall
    \begin{align*}
        \Delta t = \gamma \Delta t_\text{E}
    \end{align*}
    der Eigenzeit $\Delta_\text{E}$. 
\end{formal}
\begin{formal}
    Bewegte Maßstäbe sind kürzer (aus Sicht des ruhenden Systems). Die Ruhelänge $\Delta x_\text{E}$ im bewegten System ist im ruhenden System kontrahiert\footnotemark,
    \begin{align*}
        \Delta x = \frac{1}{\gamma} \Delta x_\text{E}.
    \end{align*}
\end{formal}
\footnotetext{Achtung, diese Gleichung steht nicht im Widerspruch zu Gleichung \eqref{eq:laengenkontraktion}, 
denn für die Myonen haben wir die Ruhelänge der Atmosphäre aus dem bewegten System betrachtet und 
nicht beispielsweise den Radius der Myonen aus dem ruhenden System. 
Natürlich dürfen genausogut die Myonen als ruhend angenommen werden. 
Wie die Zeitdilatation ist auch die Längenkontraktion ein symmetrischer Effekt. 
Die größte Länge (Ruhelänge) wird stets im eigenen System (in dem die zu messende Länge ruht) gemessen. }

In Abbildung \Abbref{fig:minkowski_zeitdilatation_laengenkontraktion} sind die Maßstäbe veranschaulicht, die zur Zeitdilatation und zur Längenkontraktion führen. 

\begin{figure}[htp]
    \centering
    \tfigMinkowskiZeitdilatationLaengenkontraktion
    \caption{Minkowski-Diagramm zur Veranschaulichung der Zeitdilatation und der Längenkontraktion: 
    Die Länge vom Ursprung zum Punkt P beschreibt den Einheitsmaßstab der Zeitachse im bewegten System. 
    Im ruhenden System wirkt diese Zeit um den Faktor $\gamma$ länger. 
    Der Abstand von Ursprung zum Punkt M beschreibt den Längeneinheitsmaßstab im bewegten System. 
    Eine Ruhelänge wirkt im bewegten System kürzer. }
    \label{fig:minkowski_zeitdilatation_laengenkontraktion}
\end{figure}



\subsection{Die Eigenzeit}

Als Eigenzeit $\tau$ wird die Zeit im Ruhesystem einer bewegten Uhr bezeichnet. Es ist 
\begin{align*}
    c\diff\tau=c\diff t' \overset{\eqref{eq:zeitdilatation}}{=} \frac{1}{\gamma} c\diff t =\left(c^2\diff t^2-v^2\diff t^2\right)^{1/2} = \left(c^2\diff t^2-\diff\vec r^2\right)^{1/2},
\end{align*}
was eine Invariante ist \textendash{} ein sogenanntes Lorentzskalar, das in allen Inertialsystemen gleich ist. 
Wir wollen den Ausdruck $\left(c^2\diff t^2-\diff\vec r^2\right)^{1/2}$ als Wegelement $\diff\bar{s}$ im vierdimensionalen Raum bezeichnen. 
Durch Integration erhält man die Eigenzeit. 
\begin{align*}
    \boxed{\tau=\int \frac{1}{\gamma}\diff t=\frac{1}{c}\int\diff\bar{s}}. 
\end{align*}