\chapter{Elektrische Felder in Materie}

In diesem Kapitel werden die makroskopischen Gleichungen der Elektrostatik in Materie beschrieben und erläutert.

\section{Mikroskopische Gleichungen der Elektrostatik und Mittelung}

Bis jetzt haben wir nur freie Ladungen betrachtet. Die Ladungsdichte $\rho \left(\vec {r}\right)$ erzeugt ein elektrisches Feld $\vec {E}(\vec r)$. In Materie sind zusätzlich auch gebundene Ladungen vorhanden, die mit dem Feld wechselwirken. Das können (nach außen hin elektrisch neutrale) Atome, geladenen Ionen, permanente Dipole (oder Multipole) sein (z.B. $\mathrm{H}_{2}\mathrm{O}$), sowie Dipole sein, die durch ein äußeres elektrisches Feld induziert werden.

Um diese Wechselwirkung zu beschreiben, wird eine Mittelung der mikroskopischen Gleichungen
\begin{equation*}
	\divg \vec {e}=\frac{1}{\varepsilon _{0}}\rho \left(\vec {r}\right),\quad \rot \vec {e}=0
\end{equation*}
durchgeführt. Wir haben bisher einzelne Ladungen durch $\delta $-Funktionen in der Ladungsdichte beschrieben. Dadurch kommt es zu starken räumlichen Ladungsschwankungen. Für eine makroskopische Betrachtung in Größenordnungen von Nanometern wenden wir eine räumliche Mittelung bzw. Glättungsfunktion auf die Ladungsdichteverteilung an.


\subsection{Glättungsfunktion}

Um eine stark variierende Funktion $F\left(\vec {r},t\right)$ zu mitteln, wird sie mit einer sogenannten Glättungsfunktion $f$ gefaltet. Dabei kann es sich z.B. um eine Gauß-Funktion handeln:
\begin{equation*}
	F\left(\vec {r},t\right) \xrightarrow{\text{Mittelung}} \left\langle F\left(\vec {r},t\right)\right\rangle =\int f\left(\left| \vec {r}-\vec {r}'\right| \right)F\left(\vec {r}',t\right)\diff ^{3}\vec {r}'
\end{equation*}
Für eine Punktladung $F\left(\vec {r}\right)=F_{0}\delta \left(\vec {r}-\vec {r}_{0}\right)$ ist dann zum Beispiel $\left\langle F\right\rangle =F_{0}f\left(\vec {r}-\vec {r}_{0}\right)$.

Für die Mittelung gelten die folgenden Eigenschaften:\begin{enumerate}
	\item Die Mittelung der konstanten Funktion $F=1$ ist genau dann konstant $1$, wenn die Glättungsfunktion über den gesamten Raum auf $1$ normiert ist,
		\begin{equation*}
			\left\langle 1\right\rangle =1\equivalence \int \diff ^{3}\vec {r}f=1.
		\end{equation*}
	\item $\partial _{i}\left\langle F\right\rangle =\left\langle \partial _{i}F\right\rangle $.
\end{enumerate}

\section{Makroskopische Gleichungen der Elektrostatik}

Mithilfe der Glättung kann man das makroskopische $\vec {E}$-Feld als
\begin{equation*}
	\vec {E}\left(\vec {r},t\right)=\left\langle \vec {e}\left(\vec {r},t\right)\right\rangle
\end{equation*}
schreiben. Für die Ladungsdichte erhält man
\begin{equation*}
	\left\langle \rho \left(\vec {r}\right)\right\rangle =\left\langle \rho _{f}\left(\vec {r}\right)+\rho _{b}\left(\vec {r}\right)\right\rangle =\left\langle \rho _{f}\left(\vec {r}\right)\right\rangle +\left\langle \rho _{b}\left(\vec {r}\right)\right\rangle \equiv \rho _{F}+\rho _{B}.
\end{equation*}
Die gebundenen Ladungen werden als Summe der Ladungsdichten einzelner Moleküle geschrieben:
\begin{equation*}
	\rho _{b}\left(\vec {r}\right)=\sum _{n}\rho _{n}\left(\vec {r}\right), \rho _{n}\left(\vec {r}\right)=\sum _{i}q_{i}\delta \left(\vec {r}-\vec {r}_{i}\right)=\sum _{i}q_{i}\delta \left(\vec {r}-\left(\vec {r}_{n}+\vec {r}_{ni}\right)\right)
\end{equation*}
mit neuen Bezugspunkten $\vec {r}_{n}$ für die einzelnen Moleküle. Für die Mittelung wird dann eine Taylor-Entwicklung um diese neuen Bezugspunkte $\vec {r}_{n}$ durchgeführt:
\begin{align*}
	\left\langle \rho _{n}\left(\vec {r}\right)\right\rangle &=\sum _{i}q_{i}f\left(\vec {r}-\left(\vec {r}_{n}+\vec {r}_{ni}\right)\right)\\&=\sum _{i}q_{i}\left[f\left(\vec {r}-\vec {r}_{n}\right)-\vec {r}_{ni}\cdot \nabla f\left(\vec {r}-\vec {r}_{n}\right)+\frac{1}{2}\left(\vec {r}_{ni}\right)_{k}\left(\vec {r}_{ni}\right)_{l}\nabla _{k}\nabla _{l}f\left(\vec {r}-\vec {r}_{n}\right)+\ldots \right]
\end{align*}
Daraus können die molekularen Dipolmomente bestimmt werden:
\begin{align*}
		q_{n}                            & =\sum _{i}q_{i} &\text{(Molekulare Ladung)}     \\
		\vec {p}_{n}                     & =\sum _{i}q_{i}\vec {r}_{ni} &\text{(Molekulares Dipolmoment)}    \\
		\left(\mathrm{Q}_{n}\right)_{kl} & =3\sum _{i}q_{i}\left(\vec {r}_{ni}\right)_{k}\left(\vec {r}_{ni}\right)_{l} &\text{(Molekulares Quadrupolmoment)}
\end{align*}
(vgl. Multipolmomente einer kontinuierlichen Ladungsverteilung, aber hier jetzt diskret). Insgesamt ergibt sich eine Verschmierung punktförmiger molekularer Multipole:
\begin{align*}
	\left\langle \rho _{n}\left(\vec {r}\right)\right\rangle &=q_{n}f\left(\vec {r}-\vec {r}_{n}\right)-\vec {p}_{n}\cdot \nabla f\left(\vec {r}-\vec {r}_{n}\right)+\frac{1}{6}\left(\mathrm{Q}_{n}\right)_{kl}\nabla _{k}\nabla _{l}f\left(\vec {r}-\vec {r}_{n}\right) \\&=\left\langle q_{n}\delta \left(\vec {r}-\vec {r}_{n}\right)\right\rangle -\nabla \cdot \left\langle p_{n}\delta \left(\vec {r}-\vec {r}_{n}\right)\right\rangle +\frac{1}{6}\nabla _{k}\nabla _{l}\left\langle \left(\mathrm{Q}_{n}\right)_{kl}\delta \left(\vec {r}-\vec {r}_{n}\right)\right\rangle
\end{align*}
und für die gemittelte gebundene Ladungsdichte:
\begin{equation*}
	\left\langle \rho _{b}\left(\vec {r}\right)\right\rangle =\rho _{\mathrm{m}}\left(\vec {r}\right)-\nabla \cdot \vec {P}\left(\vec {r}\right)+\nabla _{k}\nabla _{l}{Q}_{\mathrm{kl}}+\ldots
\end{equation*}
mit der makroskopischen Ladungsdichte (Monopoldichte)
\begin{equation*}
	\rho _{\mathrm{m}}\left(\vec {r}\right)=\left\langle \sum _{n}q_{n}\delta \left(\vec {r}-\vec {r}_{n}\right)\right\rangle ,
\end{equation*}
der Polarisation (Dipolmomentdichte)
\begin{equation*}
	\vec {P}\left(\vec {r}\right)=\left\langle \sum _{n}\vec {p}_{n}\delta \left(\vec {r}-\vec {r}_{n}\right)\right\rangle
\end{equation*}
und so weiter.
