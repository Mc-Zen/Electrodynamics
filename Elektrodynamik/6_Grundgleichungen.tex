% !TEX root = Theo_III.tex


\chapter{Grundgleichungen der Elektrodynamik}

\section{Die Maxwellschen Gleichungen: Zusammenstellung}

Die Maxwellgleichungen bilden die Basis der Elektrodynamik und fast alle Gleichungen zur Beschreibung elektromagnetische Phänomene können aus ihnen hergeleitet werden. Das Ziel ist die Beschreibung der Dynamik von elektrischen und magnetischen \textendash{} den elektromagnetischen \textendash{} Feldern.

Alle Maxwell-Gleichungen wurden in den vorangegangenen Kapiteln bereits in einer oder mehreren Formen behandelt.

Das Gaußsche Gesetz
\begin{equation*}
	\divg \vec {D}=\rho
\end{equation*}
besagt, dass die Quellen des dielektrischen Feldes makroskopische Ladungsdichten sind. Die magnetische Flussdichte hat keine Quellen, es gibt keine magnetischen Monopole und die Feldlinien sind geschlossen,
\begin{equation*}
	\divg \vec {B}=0.
\end{equation*}
Nach dem Faradayschen Induktionsgesetz
\begin{equation*}
	\rot \vec {E}=-\frac{\partial \vec {B}}{\partial t}
\end{equation*}
erzeugen veränderliche Magnetfelder rotierende elektrische Felder. Zuletzt beschreibt das Ampèresche Gesetz (mit Maxwellschem Verschiebungsstrom), dass rotierende Magnetfelder aus Stromdichten und veränderlichen dielektrischen Feldern entstehen,
\begin{equation*}
	\rot \vec {H}=\vec {j}+\frac{\partial \vec {D}}{\partial t}.
\end{equation*}
Diese Gleichungen gelten sowohl im Vakuum als auch in Materie. Dabei ist $\rho \left(\vec {r},t\right)$ die makroskopische Ladungsdichte und $\vec {j}\left(\vec {r},t\right)$ die makroskopische Stromdichte. Diese erfüllen die schon bekannte Kontinuitätsgleichung
\begin{equation*}
	\frac{\partial \rho }{\partial t}+\divg \vec {j}=0.
\end{equation*}
Im Vakuum nehmen die Maxwell-Gleichungen die folgende Form an:
\begin{equation*}
	\divg \vec {E}=\frac{1}{\varepsilon _{0}}\rho ,\quad \rot \vec {E}=-\frac{\partial \vec {B}}{\partial t},\quad \divg \vec {B}=0,\quad \rot \vec {B}=\mu _{0}\vec {j}+\frac{1}{c^{2}}\frac{\partial }{\partial t}\vec {E}
\end{equation*}
Auch die mikroskopischen Maxwellgleichungen haben wir kennengelernt. In Materie wird über stark fluktuierende Felder gemittelt, $\vec {E}=\left\langle \vec {e}\right\rangle $ und $\vec {B}=\left\langle \vec {b}\right\rangle $, sodass für die Ladungs- und Flussdichte gilt:
\begin{align*}
	\left\langle \tilde{\rho }\right\rangle    & =\rho -\nabla \cdot \vec {P}+\partial _{k}\partial _{l}Q_{kl}+\ldots  \\
	\left\langle \tilde{\vec {j}}\right\rangle & =\vec {j}+\frac{\partial }{\partial t}\vec {P}+\nabla \times \vec {M}
\end{align*}
Durch Einführung des dielektrischen Verschiebungsfeldes $\vec {D}$ und des Magnetfeldes $\vec {H}$ als Hilfsfelder erhält man dann die Materialgleichungen
\begin{equation*}
	\vec {D}=\varepsilon _{0}\vec {E}+\vec {P}-\nabla Q, \vec {B}=\mu _{0}\left(\vec {H}+\vec {M}\right)
\end{equation*}
bzw. in linearen Medien einfacher
\begin{alignat*}{4}
	\vec {P} & =\varepsilon _{0}\chi _{e}\vec {E} &  & \implication \vec {D} &  & =\varepsilon \vec {E},\quad &  & \varepsilon =\varepsilon _{0}\left(1+\chi _{e}\right)=\varepsilon _{0}\varepsilon _{r} \\
	\vec {M} & =\chi _{m}\vec {H}                 &  & \implication \vec {B} &  & =\mu \vec {H},\quad         &  & \mu =\mu _{0}\left(1+\chi _{m}\right)=\mu _{0}\mu _{r}.
\end{alignat*}
Die Kraftdichte auf Ladungsdichten und Stromdichten ist
\begin{equation*}
	f\left(\vec {r},t\right)=\rho \vec {E}+\vec {j}\times \vec {B},
\end{equation*}
also gerade die "`Lorentzkraft-Dichte`` als Kontinuumsform der Lorentzkraft.

\section{Elektromagnetische Potentiale}

Eine allgemeine Lösungsstrategie der Maxwellschen Gleichungen liefert die Bestimmung der elektromagnetischen Potentiale. Die Maxwellschen Gleichungen lassen sich als Paare von zwei homogenen Differentialgleichungen,
\begin{equation*}
	\divg \vec {B}=0, \quad\rot \vec {E}+\frac{\partial \vec {B}}{\partial t}=0
\end{equation*}
und zwei inhomogenen Differentialgleichungen,
\begin{equation*}
	\divg \vec {D}=\rho ,\quad \rot \vec {H}-\frac{\partial \vec {D}}{\partial t}=\vec {j}
\end{equation*}
schreiben.

Zuerst werden die homogenen Maxwellgleichungen gelöst. Die zweite Gleichung lässt sich umschreiben zu
\begin{equation*}
	0=\rot \left(\vec {E}+\frac{\partial \vec {A}}{\partial t}\right)\equiv -\rot \left(\nabla \varphi \right).
\end{equation*}
Dabei haben wir ein Vektorpotential $\vec {B}\equiv \nabla \times \vec {A}$ und ein skalares Potential $\vec {E}+\partial _{t}\vec {A}=-\nabla \varphi $ eingeführt. Das magnetische Vektorpotential $\vec {A}$ ist in dieser Form bereits bekannt und das skalare Potential $\varphi $ ist die Verallgemeinerung des elektrischen Potentials auf dynamische Felder.

Die so gewonnenen Gleichungen
\begin{equation}
	\label{eq:equations_for_B_and_E}
	\vec {B}=\rot \vec {A},\quad \vec {E}=-\nabla \varphi -\frac{\partial \vec {A}}{\partial t}
\end{equation}
können jetzt in die inhomogenen Maxwellgleichungen eingesetzt werden, wobei wir uns im Folgenden auf die Gleichungen im Vakuum beschränken. Einsetzen in das Gaußsche Gesetz liefert
\begin{equation}
	\label{eq:dgl_phi_A_rho}
	\nabla ^{2}\varphi +\frac{\partial }{\partial t}\nabla \cdot \vec {A}=-\frac{\rho }{\varepsilon _{0}}
\end{equation}
und aus dem Ampèreschen Gesetz folgt
\begin{equation*}
	\nabla \times \left(\nabla \times \vec {A}\right)=\mu _{0}\vec {j}-\frac{1}{c^{2}}\left(\nabla \frac{\partial \varphi }{\partial t}+\frac{\partial ^{2}\vec {A}}{\partial t^{2}}\right),
\end{equation*}
was man umstellen kann zu
\begin{equation}
	\label{eq:em_wellengleichung}
	\nabla ^{2}\vec {A}=\frac{1}{c^{2}}\frac{\partial ^{2}}{\partial t^{2}}\vec {A}-\nabla \left(\nabla \cdot \vec {A}+\frac{1}{c^{2}}\frac{\partial \varphi }{\partial t}\right)=-\mu _{0}\vec {j}.
\end{equation}
Wir erhalten also insgesamt vier gekoppelte partielle Differentialgleichungen für $\vec {A}$ und $\varphi $ statt acht für $\vec {E}$ und $\vec {B}$. In der Wellengleichung \eqref{eq:em_wellengleichung} können wir zur Bequemlichkeit den D’Alembert-Operator oder auch Wellenoperator
\begin{equation*}
	\Box\equiv\frac{1}{c^{2}}\frac{\partial ^{2}}{\partial t^{2}}-\nabla ^{2}
\end{equation*}
einführen.

In den Gleichungen \eqref{eq:equations_for_B_and_E} sind allerdings die Potentiale $\varphi $ und $\vec {A}$ nicht eindeutig, denn sie können mithilfe einer beliebigen skalaren Eichfunktion $\lambda \left(\vec {r},t\right)$ umgeeicht werden:
\begin{equation}
	\label{eq:umeichung}
	\tilde{\vec {A}}=\vec {A}+\nabla \lambda , \quad\tilde{\varphi }=\varphi -\frac{\partial \lambda }{\partial t}
\end{equation}
Alle auf diese Weise gewonnenen Potentiale $\tilde{\varphi }$ und $\tilde{\vec {A}}$ lassen die physikalischen Felder in \eqref{eq:equations_for_B_and_E} unverändert, was als Eichinvarianz bezeichnet wird. Es wird meist eine spezielle Wahl für die Eichung getroffen.

\subsection{Lorenz-Eichung}

Für die Lorenz-Eichung werden $\vec {A}$ und $\varphi $ so gewählt, dass
\begin{equation}
	\label{eq:lorenzeichung}
	\nabla \cdot \vec {A}+\frac{1}{c^{2}}\frac{\partial \varphi }{\partial t}=0,
\end{equation}
Diese Gleichung ist invariant unter den Lorentz-Transformation der speziellen Relativitätstheorie. Es ergeben sich damit die folgenden inhomogenen Wellengleichungen für $\varphi $ und $\vec {A}$:
\begin{align*}
	\nabla ^{2}\vec {A}-\frac{1}{c^{2}}\frac{\partial ^{2}}{\partial t^{2}}\vec {A} & =-\mu _{0}\vec {j}\equivalence \Box \vec {A}=\mu _{0}\vec {j}                              \\
	\nabla ^{2}\varphi -\frac{1}{c^{2}}\frac{\partial ^{2}}{\partial t^{2}}\varphi  & =-\frac{1}{\varepsilon _{0}}\rho \equivalence \Box \varphi =\frac{1}{\varepsilon _{0}}\rho
\end{align*}
Bemerkenswerterweise sind jetzt die verschiedenen Potentiale entkoppelt. Die Lorenz-Eichung ist immer durchführbar, denn man $\lambda $ nur nach der Differentialgleichung (setze \eqref{eq:umeichung} in \eqref{eq:lorenzeichung} ein)
\begin{equation*}
	\nabla ^{2}\lambda -\frac{1}{c^{2}}\frac{\partial ^{2}}{\partial t^{2}}\lambda =-\left(\nabla \cdot \vec {A}+\frac{1}{c^{2}}\frac{\partial \varphi }{\partial t}\right)
\end{equation*}
zu wählen, was stets möglich ist.


\subsection{Coulomb-Eichung}

Für die Coulomb-Eichung wählt man das magnetische Vektorpotential $\vec {A}$ divergenzfrei, also sodass
\begin{equation*}
	\nabla \cdot \vec {A}=0.
\end{equation*}
Dann gilt mit Gleichung \eqref{eq:dgl_phi_A_rho}
\begin{equation*}
	\nabla ^{2}\varphi =-\frac{1}{\varepsilon _{0}}\rho
\end{equation*}
und $\varphi $ ist gerade das Coulomb-Potential von $\rho $:
\begin{equation*}
	\varphi \left(\vec {r},t\right)=\frac{1}{4\pi \varepsilon _{0}}\int \frac{\rho \left(\vec {r}',t'\right)}{\left| \vec {r}-\vec {r}'\right| }\diff ^{3}\vec {r}'
\end{equation*}
Aus der Gleichung \eqref{eq:dgl_phi_A_rho} erhält man dann
\begin{equation*}
	\nabla ^{2}\vec {A}-\frac{1}{c^{2}}\frac{\partial ^{2}}{\partial t^{2}}\vec {A}=-\mu _{0}\vec {j}+\frac{1}{c^{2}}\nabla \frac{\partial \varphi }{\partial t}.
\end{equation*}
Auf der rechten Seite erkennen wir den transversalen Anteil der Stromdichte,
\begin{equation*}
	\nabla ^{2}\vec {A}-\frac{1}{c^{2}}\frac{\partial ^{2}}{\partial t^{2}}\vec {A}\equiv\mu _{0}\vec {j}_{t},\quad \vec {j}_{t}=\frac{1}{4\pi }\nabla \times \int \frac{\rot \vec {j}\left(\vec {r}',t'\right)}{\left| \vec {r}-\vec {r}'\right| }\diff ^{3}\vec {r}'.
\end{equation*}
Man sieht sofort, dass $\divg \vec {j}_{t}=0$, weil $\vec {j}_{t}$ ein reines Wirbelfeld ist.

Wegen $\nabla \cdot \vec {A}=0$ sind die Wellen von $\vec {A}$ transversal, weswegen man auch von der transversalen Eichung oder Strahlungseichung spricht. Das Vektorpotential $\vec {A}$ bestimmt die Wellenausbreitung, während $\varphi $ die statische Lösung beschreibt.

\section{Erhaltungssätze der Elektrodynamik}

\subsection{Erhaltung von Energie: Satz von Poynting}

Wir erinnern uns an die elektromagnetische Kraftdichte, die aus der Lorentzkraft hervorgeht:
\begin{equation*}
	\vec {f}\left(\vec {r},t\right)=\rho \vec {E}+\vec {j}\times \vec {B},\quad \vec {j}=\rho \vec {v}
\end{equation*}
Multiplikation mit der Geschwindigkeit der Ladungsdichte $\vec {v}$ liefert die pro Volumen- und Zeiteinheit geleistete Arbeit des elektrischen Feldes:
\begin{equation*}
	\vec {f}\cdot \vec {v}=\vec {j}\cdot \vec {E}.
\end{equation*}
Sie entspricht genau der mechanischen Arbeit bzw. Jouleschen Wärme. Das magnetische Feld verrichtet keine Arbeit, da $\vec {v}\cdot \left(\vec {j}\times \vec {B}\right)$.

Den Satz von Poynting leiten wir nun her, indem wir das Induktionsgesetz skalar mit dem Magnetfeld $\vec {H}$ multiplizieren und das Durchflutungsgesetz mit dem elektrischen Feld $\vec {E}$ und die beiden Gleichungen subtrahieren:
\begin{equation*}
	\vec {H}\cdot \left(\nabla \times \vec {E}\right)-\vec {E}\cdot \left(\nabla \times \vec {H}\right)=-\vec {j}\cdot \vec {E}-\vec {E}\cdot \frac{\partial \vec {D}}{\partial t}-\vec {H}\cdot \frac{\partial \vec {B}}{\partial t}.
\end{equation*}
Umstellen und Anwenden von $\nabla \cdot \left(\vec {E}\times \vec {H}\right)=\vec {H}\cdot \left(\nabla \times \vec {E}\right)-\vec {E}\cdot \left(\nabla \times \vec {H}\right)$ führt auf die Gleichung
\begin{equation}
	\label{eq:satz_von_poynting}
	\boxed{\vec {E}\cdot \frac{\partial \vec {D}}{\partial t}+\vec {H}\cdot \frac{\partial \vec {B}}{\partial t}+\nabla \cdot \left(\vec {E}\times \vec {H}\right)=-\vec {j}\cdot \vec {E}.}
\end{equation}
Die ersten beiden Terme auf der linken Seite geben die Änderung der elektromagnetischen Feldenergie pro Volumen und Zeiteinheit an. Der Teil $\nabla \cdot \left(\vec {E}\times \vec {H}\right)$ beschreibt den Fluss der elektromagnetischen Feldenergie aus einem Volumen heraus und die rechte Seite ist gerade die mechanische Arbeit bzw. Joulesche Wärme (wie wir gerade gesehen haben), die dem System verloren geht.

Die Größe $\vec {E}\times \vec {H}$ wird als Poynting-Vektor definiert,
\begin{equation*}
	\vec {S}=\vec {E}\times \vec {H}.
\end{equation*}
Er gibt die elektromagnetische Energiestromdichte an und seine Einheit ist dementsprechend Energie pro Fläche und Zeit.

Wir betrachten jetzt zwei Spezialfälle. Im Vakuum und in linearen Materialien ist $\vec {D}=\varepsilon \vec {E}$ und $\vec {B}=\mu \vec {H}$ und wir vernachlässigen Zeitabhängigkeiten von $\varepsilon $ und $\mu $. Mit der elektromagnetischen Feldenergiedichte $u=\left(\vec {E}\cdot \vec {D}+\vec {H}\cdot \vec {B}\right)/2$ lässt sich die Gleichung \eqref{eq:satz_von_poynting} nun umschreiben zu
\begin{equation*}
	\frac{\partial u}{\partial t}+\nabla \cdot \vec {S}=-\vec {j}\cdot \vec {E}
\end{equation*}
bzw. in der integralen Formulierung
\begin{equation*}
	\frac{\diff }{\diff t}\int _{V}u\diff ^{3}\vec {r}+\int _{\partial V}\vec {S}\cdot \diff \vec {f}=-\int \vec {j}\cdot \vec {E}\diff ^{3}\vec {r}.
\end{equation*}
Als zweiter Sonderfall sollen linear dispersive Materialien behandelt werden. Die Materialkonstanten $\varepsilon $ und $\mu $ werden als komplexe, frequenzabhängige Konstanten geschrieben,
\begin{equation*}
	\varepsilon \left(\omega \right)=\varepsilon '\left(\omega \right)+i\varepsilon ''\left(\omega \right),\quad \mu \left(\omega \right)=\mu '\left(\omega \right)+i\mu ''\left(\omega \right).
\end{equation*}
Wir betrachten zeitlich periodische (und harmonische) Felder und beschreiben sie durch komplexe Größen:
\begin{align*}
	\vec {E}\left(\vec {r},t\right) & =\vec {E}_{0}\left(\vec {r}\right)e^{-i\omega t},                 & \vec {H}\left(\vec {r},t\right) & =\vec {H}_{0}\left(\vec {r}\right)e^{-i\omega t}         \\
	\vec {D}\left(\vec {r},t\right) & =\varepsilon \left(\omega \right)\vec {E}\left(\vec {r},t\right), & \vec {B}\left(\vec {r},t\right) & =\mu \left(\omega \right)\vec {H}\left(\vec {r},t\right)
\end{align*}
Physikalisch sind natürlich nur die reellen Felder $\vec {E}_{\mathrm{R}}=\mathfrak{R}\left(\vec {E}\right), \vec {H}_{\mathrm{R}}=\mathfrak{R}\left(\vec {H}\right)$ usw. messbar.

Die Änderung der elektromagnetischen Energiedichte lässt sich als Mittelung über eine Periode schreiben:
\begin{align*}
	\dot{u} & =\frac{1}{T}\int _{0}^{T=\frac{2\pi }{\omega }}\left(\vec {E}_{\mathrm{R}}\cdot \frac{\partial \vec {D}_{\mathrm{R}}}{\partial t}+\vec {H}_{\mathrm{R}}\cdot \frac{\partial \vec {B}_{\mathrm{R}}}{\partial t}\right)\diff t                  \\
	        & =\frac{1}{T}\int _{0}^{T=\frac{2\pi }{\omega }}\left(\omega E_{0}^{2}\cos \left(\omega t\right)\left[-\varepsilon '\left(\omega \right)\sin \left(\omega t\right)+\varepsilon ''\left(\omega \right)\cos \left(\omega t\right)\right] \right. \\
	        & \qquad\qquad\qquad \left. +\omega H_{0}^{2}\cos \left(\omega t\right)\left[-\mu '\left(\omega \right)\sin \left(\omega t\right)+\mu ''\left(\omega \right)\cos \left(\omega t\right)\right]\right)\diff t
\end{align*}
Man könnte zuerst denken, dass dieser Ausdruck verschwindet, da die Feldenergie konstant sein sollte. Das Integral der Terme, die den realen Anteil $\varepsilon '$ und $\mu '$ der elektromagnetischen Materialkonstanten enthalten, verschwindet tatsächlich, weil das Produkt $\cos \left(\omega t\right)\sin \left(\omega t\right)$ über eine Periode integriert $0$ ist. Die Terme mit den imaginären Anteilen $\varepsilon ''$ und $\mu ''$ heben sich dagegen nicht auf, weil
\begin{align*}
	\dot{u} & =\frac{\omega }{T}\left(E_{0}^{2}\varepsilon ''\left(\omega \right)+H_{0}^{2}\mu ''\left(\omega \right)\right)\int _{0}^{T=\frac{2\pi }{\omega }}\cos ^{2} \left(\omega t\right)\diff t \\
	        & =\frac{\omega }{2\pi }\left(E_{0}^{2}\varepsilon ''\left(\omega \right)+H_{0}^{2}\mu ''\left(\omega \right)\right)\int _{0}^{2\pi }\cos ^{2} \left(u\right)\diff u                      \\
	        & =\frac{\omega }{2}\left(E_{0}^{2}\varepsilon ''\left(\omega \right)+H_{0}^{2}\mu ''\left(\omega \right)\right).
\end{align*}
Die imaginären Anteile $\varepsilon ''\left(\omega \right)$ und $\mu ''\left(\omega \right)$ der elektromagnetischen Materialkonstanten sind also verantwortlich für die Dissipation der elektromagnetischen Energie.

Wir betrachten jetzt geladene Teilchen und elektromagnetische Felder im Vakuum und führen die "`mechanische`` Energie $u_{\text{mech}}$ ein, bzw. die Energie, die nicht elektromagnetisch ist. Wir können dann wegen
\begin{equation*}
	\frac{\diff u_{\text{mech}}}{\diff t}=\vec {j}\cdot \vec {E}
\end{equation*}
sagen, dass
\begin{equation*}
	\frac{\partial }{\partial t}\left(u_{\text{mech}}+u_{\text{Feld}}\right)+\divg \vec {S}=0.
\end{equation*}
Dies ist die Kontinuitätsgleichung für die Gesamtenergie und beschreibt damit die Energieerhaltung des Systems.

\subsection{Impulserhaltung}

Wenn die "`mechanische`` Energie in elektromagnetische umgewandelt werden kann und umgekehrt, dann ist es auch sinnvoll, analog zum mechanischen Impuls so etwas wie einen Impuls des elektromagnetischen Feldes zu definieren, um eine Kontinuitätsgleichung im Vakuum abzuleiten.

Die zeitliche Änderung des mechanischen Impulses $P_{\text{mech}}$ lässt sich als Integral über die Impulsdichte $p_{\text{mech}}$ beschreiben, dessen zeitliche Änderung gerade der Kraftdichte $f$ entspricht, die durch die Lorentzkraft ausgedrückt werden kann:
\begin{equation*}
	\frac{\diff P_{\text{mech}}}{\diff t}=\frac{\diff }{\diff t}\int p_{\text{mech}}\diff ^{3}\vec {r}=\int f\left(\vec {r},t\right)\diff ^{3}\vec {r}=\int \left(\rho \vec {E}+\vec {j}\times \vec {B}\right)\diff ^{3}\vec {r}
\end{equation*}
Die elektromagnetische Kraftdichte lässt sich mithilfe der Maxwellschen Gleichungen umformen zu
\begin{align*}
	\rho \vec {E}+\vec {j}\times \vec {B} & =\nabla \cdot \vec {D}\vec {E}+\left(\nabla \times \vec {H}-\frac{\partial \vec {D}}{\partial t}\right)\times \vec {B}                                                                                                                                                                                                          \\
	                                      & =\vec {E}\left(\nabla \cdot \vec {D}\right)+\vec {B}\times \frac{\partial \vec {D}}{\partial t}-\vec {B}\times \left(\nabla \times \vec {H}\right)                                                                                                                                                                              \\
	                                      & =\underset{\mathrm{A}}{\underbrace{\vec {E}\left(\nabla \cdot \vec {D}\right)+\left(\nabla \times \vec {E}\right)\times \vec {D}}}-\frac{\partial }{\partial t}\underset{\mathrm{B}}{\underbrace{\left(\vec {D}\times \vec {B}\right)}}-\underset{\mathrm{C}}{\underbrace{\vec {B}\times \left(\nabla \times \vec {H}\right)}}.
\end{align*}
Der Term B beschreibt die neue Impulsdichte des elektromagnetischen Feldes
\begin{equation*}
	p_{\mathrm{em}}=\vec {D}\times \vec {B}=\frac{1}{c^{2}}\vec {E}\times \vec {H}=\frac{1}{c^{2}}\vec {S}.
\end{equation*}
Die $i$-te Komponente vom Term A lässt sich als
\begin{equation*}
	E_{i}\partial _{j}D_{j}-\partial _{i}^{\left(E\right)}\left(E_{j}D_{j}\right)+D_{j}\partial _{j}E_{i}=\partial _{j}\left(E_{i}D_{j}\right)-\frac{1}{2}\partial _{i}\left(E_{j}D_{j}\right)=\partial _{j}\left(E_{i}D_{j}-\frac{1}{2}\delta _{ij}E_{k}D_{k}\right)
\end{equation*}
schreiben (wobei der Differentialoperator $\partial _{i}^{\left(E\right)}$ nur auf Komponente von $E$ wirkt) und die $i$-te Komponente von Teil C als
\begin{equation*}
	\partial _{i}^{\left(H\right)}\left(H_{j}B_{j}\right)+B_{j}\partial _{j}H_{i}+\underset{0}{\underbrace{H_{i}\partial _{j}B_{j}}}=\partial _{j}\left(H_{i}B_{j}-\frac{1}{2}\delta _{ij}H_{k}B_{k}\right).
\end{equation*}
Zusammen können die Terme B und C als Ableitung $\partial _{j}T_{ij}$ eines Tensors $T_{ij}$ geschrieben werden. $T_{ij}$ ist der Maxwellsche Spannungstensor (dessen elektrostatischer Anteil schon aus Kapitel \ref{sec:maxwellscher_spannungstensor} bekannt ist):
\begin{equation*}
	T_{ij}=E_{i}D_{j}+H_{i}B_{j}-\frac{1}{2}\delta _{ij}\left(E_{k}D_{k}+H_{k}B_{k}\right)
\end{equation*}
Er erlaubt eine integrale Beschreibung der Impulsbilanz
\begin{equation*}
	\frac{\diff }{\diff t}\int \left(p_{\text{mech}}+p_{\mathrm{em}}\right)_{i}\diff ^{3}\vec {r}=\int _{V}\partial _{j}T_{ij}\diff ^{3}\vec {r}=\int _{\partial V}T_{ij}\diff f_{j}.
\end{equation*}
