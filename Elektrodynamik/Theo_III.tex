%
% -------------------------------------------------------------------
% Electrodynamics lecture at TU Berlin, read by Prof. Holger Stark
% -------------------------------------------------------------------
%
%
% Anmerkungen: 

% - Sections
% Der Stil der sections und deren Nummerierungen orientiert sich an der Vorlesung. 
% Statt subsection*{} gibt es einen neuen Befehl subsectiona{}, der die Unter-
% überschriften nicht nummeriert, aber trotzdem in das Inhaltsverzeichnis integriert. 

% - Vektorpfeile
% Die Vektorpfeile des esvect-Pakets bieten ein besseres Spacing, vorallem wenn
% danach eine geschlossene Klammer oder eine Hochstellung folgt. In der Datei 
% esvectmod.sty in diesem Projekt ist noch eine kleine Verbesserung des Layouts drin, 
% aber ganz perfekt funktionieren Vektorpfeile in Latex nie. Weiter unten gibt es noch 
% eine Option, alle Vektoren stattdessen fettgedruckt darzustellen...

% - Abbildungen
% Alle Abbildungen sind vektorbasierte Skizzen im PDF-Format. Das Paket wrapfig erlaubt
% zwar Abbildungen in den Textfluss zu integrieren, aber manchmal ist der Abstand
% unter der Abbildung zum Text zu groß/klein. Dann kann mit \vspace{10mm} der Abstand
% vergrößert oder mit negativen Werten verkleinert werden. 

\documentclass[hidelinks, 11pt]{scrbook} 

\usepackage[left=2cm, right=2cm, bottom=1.5cm, top=1.5cm, includeheadfoot]{geometry} 
\usepackage[T1]{fontenc} 
\usepackage[utf8]{inputenc} 
\usepackage{graphicx} 
\usepackage{multirow} 
\usepackage{tabularx} 
\usepackage{xcolor} 
\usepackage{amsmath} 
\usepackage{amssymb} 
\usepackage{amsfonts} 
\usepackage{amsxtra} 
\usepackage{mathtools} 
\usepackage{upgreek} 
\usepackage{enumerate} 
\usepackage{tensor} 
\usepackage{float}
\usepackage[english,ngerman]{babel}
\usepackage{siunitx}
\usepackage{longtable}
\usepackage{cancel}
\usepackage{wrapfig}

\usepackage{lmodern}
\usepackage[f]{esvectmod}
\usepackage[font=footnotesize]{caption}
\sisetup{locale = DE, separate-uncertainty}  
\usepackage[immediate]{silence}
\WarningFilter[temp]{latex}{Command} % filter underline/underbar command warning from sectsty 

% --------------------------
% spacing

\setlength{\parindent}{0em}
\setlength{\parskip}{1em}

\usepackage[hang]{footmisc}
\setlength{\footnotemargin}{3mm} % space between number and content of footnote
\setlength{\skip\footins}{.5cm} % space between body and footnote section
\setlength{\footnotesep}{0.5cm}  % space between footnotes

%\renewcommand{\arraystretch}{1}

\raggedbottom

\makeatletter
\renewcommand\@pnumwidth{2em} % fix toc overfull hbox
\makeatother

% --------------------------
% section styles

%\usepackage{sectsty}
%\renewcommand{\thesection}{\Alph{section}} 
%\newcommand{\subsubsectiona}[1]{\subsubsection*{#1}}%\addcontentsline{toc}{subsection}{#1}} 
%\subsubsectionfont{\normalfont\large\underline}
\newcommand{\example}[1]{\paragraph{#1:}}

\makeatletter
\renewcommand\subsubsection{\@startsection{subsubsection}{3}{\z@}%
                                     {-3.25ex\@plus -1ex \@minus -.2ex}%
                                     {1.5ex \@plus .2ex}%
                                     {\normalfont\normalsize}}
\makeatother
\newcommand{\subsubsectiona}[1]{\subsubsection*{\underline{#1}}}
                                   
%\labelformat{section}{\thechapter#1} % section refs should include chapter letter

% --------------------------
% math
\DeclareMathOperator{\grad}{grad} 
\DeclareMathOperator{\divg}{div} 
\DeclareMathOperator{\rot}{rot} 

\newcommand{\upupharpoons}{\upharpoonleft\!\upharpoonright}
\newcommand{\updownharpoons}{\upharpoonleft\!\downharpoonright}
% pm sign with minus in paretheses
\newcommand\varpm{\mathbin{\vcenter{\hbox{
  \oalign{\hfil$\scriptstyle+$\hfil\cr\noalign{\kern-.3ex}$\scriptscriptstyle({-})$\cr}
}}}}
% mp sign with plus in parentheses
\newcommand\varmp{\mathbin{\vcenter{\hbox{
  \oalign{$\scriptstyle({+})$\cr\noalign{\kern-.3ex}\hfil$\scriptscriptstyle-$\hfil\cr}
}}}}
% vector than can be primed without looking ugly af 
%\newcommand*\pvec[1]{\vec{#1}\mkern2mu\vphantom{#1}} 
%\renewcommand{\pvec}[1]{\vec{#1}} 
% differential operator have a non-italic d in German equation typesetting 
%\newcommand*\diff{\mathop{}\!\diff } 
\newcommand{\diff}{\text{d}}	 
\newcommand{\Angstroem}{\text{\normalfont\AA}}   
\newcommand{\Abbref}[1]{Abb.~\ref{#1}} 
% vector arrow that does not clash with exponents 
\makeatletter 
\newcommand{\svec}[1]{\vec{#1}\@ifnextchar{^}{\,}{}} 
\makeatother 
% NO, better use esvect, much better spacing 
%\let\oldvec\vec 
%\renewcommand{\vec}[1]{\oldvec{\mkern2mu#1\mkern2mu}} 
%\renewcommand{\vec}[1]{\oldvec{#1\mkern-2mu}} 
%\renewcommand{\vec}[1]{\vv{#1}} % esvect vector 
% Alternative: bold non-cursive symbols for vectors instead of arrows 
\renewcommand{\vec}[1]{\mathbf{#1}} 
%\renewcommand{\svec}[1]{\vec{#1}} 

\newcommand\equivalence{\;\Leftrightarrow\;}
\newcommand\implication{\;\Rightarrow\;}

% Alternative: bold non-cursive symbols for vectors instead of arrows
%\renewcommand{\vec}[1]{\mathbf{#1}}

% set figure description
\addto\captionsngerman{
  \renewcommand{\figurename}{Abb.}
  \renewcommand{\tablename}{Tab.}
}
\graphicspath{{images/}} 
\usepackage[strict]{changepage}

% for formal definitions
\usepackage{framed}

% environment derived from framed.sty: see leftbar environment definition
\definecolor{formalbar}{rgb}{0.1,0.1,.2}
\definecolor{formalshade}{rgb}{0.95,0.95,1}

% Quote box for important statements
\newenvironment{formal}{%
  \def\FrameCommand{%
    \hspace{1pt}%
    {\color{formalbar}\vrule width 2pt}%
    {\color{formalshade}\vrule width 4pt}%
    \colorbox{formalshade}%
  }%
  \MakeFramed{\advance\hsize-\width\FrameRestore}%
  \noindent\hspace{-4.55pt}% disable indenting first paragraph
  \begin{adjustwidth}{}{7pt}%
  \vspace{-10pt}\vspace{2pt}%
}
{%
  \vspace{2pt}\end{adjustwidth}\endMakeFramed%
}

% !TEX root = Theo_III.tex

\usepackage{tikz}
\usetikzlibrary{arrows.meta,positioning,decorations.markings,intersections,calc,decorations.pathreplacing}


\tikzset{
legreen/.style={green!50!black},
charge color/.style={blue!50!white!70!black},
red laser/.style={red!70!black},
moving system color/.style={blue!60!black!70!white},
coordsystem/.style={very thin, color=#1!50},
invisiblePoint/.style={circle,inner sep=0pt,outer sep=0pt,minimum size=0pt},
point/.style={invisiblePoint,fill=black,minimum size=4pt},
arr/.style={->,>={Stealth},thin},
midarrow/.style={postaction=decorate,decoration={markings, mark=at position #1 with {\arrow{Stealth}}} },
midarrow/.default=.5,
rmidarrow/.style={postaction=decorate,decoration={markings, mark=at position #1 with {\arrowreversed{Stealth}}} },
rmidarrow/.default=.5,
distance marker/.style={|<->|,>={Stealth}},
}
\tikzstyle{every node}=[font=\footnotesize]


\newcommand{\tfigWatermolecule}{

    % water molecule
    \begin{tikzpicture}[scale=3]
        \node[circle] (H1) at ($(-90+104.45/2:1)$) {$\mathrm{H}^+$};
        \node[circle] (H2) at ($(-90-104.45/2:1)$) {$\mathrm{H}^+$};
        \node[circle] (p) at (-90:.8) {};
        \node[circle] at (0,0) [label={east,xshift=-2mm,yshift=1mm}:-] {O} 
            edge (H1) 
            edge (H2) 
            edge[arr, charge color] 
            node[midway, left] {$\vec p$} (p);
    \end{tikzpicture}
}

\newcommand{\tfigElementalQuadrupoles}{
    % elemental quadrupoles
    \begin{tikzpicture}[
        scale=1.2,
        charge/.style={point, charge color, outer sep=2mm}
    ]
    \node[charge] (Q1) at (-1,-1) [label={left}:$+q$] {};
    \node[charge] at (-1,1) [label={left}:$-q$] {} edge[arr] (Q1);
    \node[charge] (Q2) at (1,1) [label={left}:$+q$] {};
    \node[charge] at (1,-1) [label={left}:$-q$] {} edge[arr] (Q2);


    \begin{scope}[shift={(5,0)}]
        \node[charge] (Q3) at (0,1) [label={left}:$+q$] {};
        \node[charge] (Q4) at (0,-1) [label={left}:$+q$] {};
        \node[charge, inner sep=1mm] at (0,0) [label={left}:$-2q$] {} 
            edge[arr] (Q3) 
            edge[arr] (Q4);
    \end{scope}
    \end{tikzpicture}
}

\newcommand{\tfigMagnetfeldBeiLeiter}{
    \begin{tikzpicture}[scale=3]
        \coordinate (O) at (0. 8,0);   % Koordinatenursprung
        \coordinate (A) at (1, 0.5);   % Punkt auf Leiter
        \coordinate (B) at (1.5, 0.2); % Punkt, wo Magnetfeld ausgewertet wird
        
        % Leiter (Kurve)
        \draw[color=blue!50!black] (0,0) .. controls (.3,.7) and (1.7,.3) .. (2,1);
        % Stromelement Idr
        \draw[arr] (1-0.1, 0.55-0.015) -- +(0.2, 0.03) 
            node[midway, above] {$I \diff \vec r'$}; 
            
        % Punkt, wo Magnetfeld ausgewertet wird
        \node[invisiblePoint,text=orange!80] (b point) at (B) 
            [label={south east}:\textcolor{orange!80}{$\vec B(\vec r)$}] {$\mathbf{\otimes}$};
        % Punkt auf Leiter
        \node[invisiblePoint] (conductor point) at (A) {}
            edge[arr] node[midway,auto,yshift=-1mm] {$\vec r-\vec r'$} 
            (b point);
        % Ursprung
        \node[point] (origin) at (O) [label={south west}:$O$] {} 
            edge[arr] node[midway,left] {$\vec r$} 
            (conductor point) 
            edge[arr] node[midway,below] {$\vec r'$} 
            (b point);
            
    \end{tikzpicture}
}
\newcommand{\tfigEfieldAndPotLinesAndChargeDensitityHomoChargedSphere}{
    % Field and equipotential lines of a charged sphere
    \begin{tikzpicture}[
        scale=2,
        potential color/.style={violet!60!black}
    ]
    \coordinate (O) at (0.0,0.0);

    \foreach \angle in {0,45,...,325}
        \draw[arr] (O) -- (\angle:1);
    \foreach \radius in {0,0.2,...,0.8}
        \draw[dashed,potential color] (O) circle [radius=\radius];
        
    \node[potential color] at (-20:1) {$V_r$};
    \node at (-150:1.2) {$\vec E$};

    \draw[charge color] (O) circle [radius=.7];
    \draw[arr, charge color] (O) -- (60:0.7) node[near end, left] {$R$};

    \begin{scope}[shift={(2.3,-.6)}]
        \coordinate (A) at (1.2, 0.8);
        
        \draw[arr] (0,0) -- (0, 1.3) 
            node[at end, anchor=south east] {$\rho(r)$};
        \draw[arr] (0,0) -- (1.6, 0) 
            node[at end, anchor=north west] {$r$};
        \draw[charge color] let \p1 = (A) in
            (0, \y1) -- (A) 
            node[at start, left] {$\rho_0$}
            -- (\x1, 0)
            node[charge color, below] {$R$};
    \end{scope}
    \end{tikzpicture}
}
\newcommand{\tfigEfieldAndPotentialHomoChargedSphere}{
    % field and potential function for sphere
    \begin{tikzpicture}[scale=3]
        \coordinate (A) at (0.8, 0.8);
        
        % electric field
        \draw[arr] (0,0) -- +(0, 1.1) 
            node[at end, anchor=south east] {$E(r)$};
        \draw[arr] (0,0) -- +(1.8, 0) 
            node[at end, anchor=north west] {$r$};
            
        \draw[charge color,yscale=.2] 
            let \n1={0.5},\n2={1.7} in
            plot[domain=0:\n1] (\x,\x/\n1^3) 
            node[above left] at (\n1/2,\n1/2/\n1^3) {$\propto r$} 
            plot[domain=\n1:\n2] (\x,1/\x^2) 
            let \n4={.5*\n2-.5*\n1+\n1} in % x value of middle point of second graph 
            node[above] at (\n4,1/\n4^2) {$\propto\frac{1}{r^2}$};
        \draw[dashed, very thin] (0.5,.8) -- +(0,-.8) 
            node[below] {$R$};
            
        % potential
        \begin{scope}[shift={(2.5,0)}]
            \draw[arr] (0,0) -- +(0, 1.1) 
                node[at end, anchor=south east] {$\phi(r)$};
            \draw[arr] (0,0) -- +(1.8, 0) 
                node[at end, anchor=north west] {$r$};
            \draw[charge color,yscale=.3] 
                let \n1={0.5},\n2={1.7} in 
                plot[domain=0:\n1] (\x,3/\n1/2 -\x^2/\n1^3/2)
                node[right, xshift=-25,yshift=25] {$\propto \frac3{2R}-\frac{r^2}{2R^3}$}
                plot[domain=\n1:\n2] (\x,1/\x)
                let \n4={.5*\n2-.5*\n1+\n1} in % x value of middle point of second graph 
                node[above] at (\n4,1/\n4) {$\propto\frac{1}{r}$};
                
            \draw[dashed, very thin] (0.5,1/1.7) -- +(0,-1/1.7) 
                node[below] {$R$};
        \end{scope}
    \end{tikzpicture}
}

\newcommand{\tfighysterese}{
    \begin{tikzpicture}[scale=2]

        \coordinate (saturation) at (1,1);
        \coordinate (inv saturation) at (-1,-1);

        % Axis
        \draw[name path=xaxis,arr] (-1.4,0) -- (1.4,0) node[anchor=north west] {$H$};
        \draw[name path=yaxis,arr] (0,-1.4) -- (0,1.4) node[anchor=south east] {$B$};

        % Curve (1) (magnetize)
        \draw[midarrow,dashed,color=blue!50!black]
        (0,0) .. controls (0.3,.2) and (0.1,1) .. (saturation)
        node[midway,right,xshift=-1.5mm,yshift=-2mm] {(1)};
        % Curve (2)  (change field to negative)
        \draw[name path=dcurve, midarrow,color=blue!50!green]
        (saturation) .. controls (-0.8,1) and (-0.1,-1) .. (inv saturation)
        node[near end,left] {(2)};
        % Curve (2)  (change field to positive)
        \draw[midarrow,color=blue!60!white!70!black]
        (inv saturation) .. controls (0.8,-1) and (0.1,1) .. (saturation)
        node[near start,right,xshift=1mm,yshift=1mm] {(3)};
        % Coordinates of Saturation
        \draw[dashed] (saturation) -- (1,0)
        node[below] {$H_\mathrm{S}$};
        \draw[dashed] (saturation) -- (0,1)
        node[left] {$B_\mathrm{S}$};

        % Remanence and coercitive field
        \path[name intersections={of=yaxis and dcurve, by=remanenz}];
        \path[name intersections={of=xaxis and dcurve, by=coercitive}];
        \draw ($(remanenz)+(1pt,0)$) -- ($(remanenz)-(1pt,0)$)
        node[left] {$B_\mathrm{R}$};
        \draw ($(coercitive)+(0,1pt)$) -- ($(coercitive)-(0,1pt)$)
        node[anchor=north west,xshift=-2mm] {$H_\mathrm{C}$};

    \end{tikzpicture}
}

\newcommand{\tfigMagneticRefraction}{
    \begin{tikzpicture}[
            scale=2,
            incolor/.style = {red!50!black},
            outcolor/.style = {blue!50!black}
        ]
        \coordinate (Start) at (-2,-0.5);
        \coordinate (StartFoot) at ($(Start) + (0,.2)$);
        \coordinate (Middle) at (0,-0.3);
        \coordinate (End) at (0.4,0.2);
        \coordinate (EndFoot) at (0,0.2);

        \draw (0,1) -- (0,-.6);
        \node[circle] at (-1,.9) {$\mu_1$};
        \node[circle] at (.5,.9) {$\mu_2\gg\mu_1$};

        \draw[arr, incolor] (Start) -- (Middle)
        node[midway, below] {$\vec H^{(1)}$};
        \draw[arr, outcolor] (Middle) -- (End)
        node[midway, right] {$\vec H^{(2)}$};
        \draw[dashed, incolor] (Start) -- (StartFoot)
        node[midway,left] {$H_\parallel^{(1)}$}
        -- (Middle) node[midway, above] {$H_\perp^{(1)}$};
        \draw[dashed, outcolor] (EndFoot)
        -- (End) node[midway, above] {$H_\perp^{(2)}$};
    \end{tikzpicture}
}


\newcommand{\tfigConductorLoopWithRefPoint}{
    \begin{tikzpicture}[scale=1]
        \coordinate (O) at (0,0);
        \coordinate (P) at (1,2);
        \draw[midarrow=.75] (O) ellipse [x radius=1, y radius=.5]
        node[] {$F$};
        \node[below] at (0,-.5) {$I$};
        \foreach \angle in{0, 55, ..., 360}
        \draw[gray] ($cos(\angle)*(1,0)+sin(\angle)*(0,.5)$) -- (P);
        \node[point] at (P) [label=P]{};
    \end{tikzpicture}
}

% \newcommand{\tfigMagneticFeldHomogenousBall}{
%     \begin{tikzpicture}[
%             scale=.7,
%             mcolor/.style=legreen,
%             pics/graph/.style={background code={
%                             \coordinate (O) at (0,0);
%                             \coordinate (P) at (1,2);
%                             \draw (O) circle [radius=1];
%                             \foreach \vzx in {1,-1}{
%                                     \draw[midarrow] (\vzx*0.5,0.866) ..
%                                     controls ($1.2*(\vzx*0.5,0.866)$)
%                                     and (\vzx*1.2, 2.4) .. (\vzx*2.5,3);
%                                     \draw[midarrow] (\vzx*2.5,-3) ..
%                                     controls (\vzx*1.2,-2.4)
%                                     and ($1.2*(\vzx*0.5,-0.866)$) .. (\vzx*0.5,-0.866);
%                                     \draw[midarrow=.51] (\vzx*0.866,0.5)
%                                     .. controls +($.8*(\vzx*0.866,0.5)$)
%                                     and (\vzx*3,1)
%                                     .. (\vzx*3,0)
%                                     .. controls (\vzx*3,-1) and ($1.6*(\vzx*0.866,-0.5)$)
%                                     .. (\vzx*0.866,-0.5) ;
%                                     \draw[arr] (\vzx*0.866,#1*-0.4) -- (\vzx*0.866,#1*0.4);
%                                     \draw[arr] (\vzx*0.5,#1*-0.8) -- (\vzx*0.5,#1*0.8);
%                                     \draw[arr, mcolor] (\vzx*0.7,-0.4) -- +(0,0.8);
%                                     \draw[arr, mcolor] (\vzx*0.25,-0.4) -- +(0,0.8);
%                                 }
%                             \draw[midarrow=.6] (0,1) -- (0,3.2);
%                             \draw[midarrow=.4] (0,-3.2) -- (0,-1);
%                             \draw[arr] (0,#1*-.9) -- (0,#1*0.9);
%                         }}
%         ]

%         %\pic=1 at (0,0) [transform shape]{graph};
%         %\pic at (7,0) [transform shape]{graph};

%         \draw (0,0) pic[transform shape] {graph=1} node[outer sep=70,above] {$\vec B$-Feld};
%         \draw (7,0) pic[transform shape] {graph=-1} node[outer sep=70,above] {$\vec H$-Feld};
%         \node[mcolor] at (-1.5,0) {$\vec M$};
%         \node[mcolor] at (-1.5+7,0) {$\vec M$};

%     \end{tikzpicture}
% }


\newcommand{\tfigCoil}{
    \begin{tikzpicture}[scale=1]
        \draw[arr] (0,0) -- (0,4.5)
        node[anchor=south east] {$z$};

        \draw[rmidarrow=.10] (-1,0)
        \foreach \a in {1,...,6}{
                .. controls +(0,.6) and +(0,.6)
                .. +(2,.35)
                .. controls +(0,-.6) and +(0,-.6)
                .. ++(0,.7)
            };
        \node[anchor=north west] at (.5,0) {$I$};
        \draw[legreen, rounded corners=6pt, midarrow=.7]
        (.8,.7) rectangle (1.2,3.5);
        \draw[legreen, |-|] (1.4,.7) -- +(0,2.8)
        node[midway, right] {$L$}
        node[near start, right] {$C=\partial F$};
    \end{tikzpicture}
}

\newcommand{\tfigInductionA}{
    \begin{tikzpicture}[scale=1]
        \draw[midarrow=.5] (0,0)
        ellipse[x radius=1, y radius=2];

        \draw (4, -.5) -- ++(-2,0)
        -- ++(0,1) node[midway, right] {N}
        -- +(2,0);
        \draw[arr] (1.7,-.19) -- ++(-1.1,0) node[near start, below] {$\vec v$};
        \node[below] at (0,-2) {$C=\partial F$};
        \node[anchor=south east] at (-1,0) {$I$};
        \draw[legreen,arr] (2,0.25) .. controls +(-1.5,0) and +(1,-1) .. (-2,1.6);
        \draw[legreen,arr] (2,-0.25) .. controls +(-1.5,0) and +(1,1) .. (-2,-1.6);
        \draw[legreen,arr] (2,0) -- (-2,0);
    \end{tikzpicture}
}

\newcommand{\tfigInductionB}{
    \begin{tikzpicture}[scale=1]
        \draw[midarrow=.5] (0,0)
        ellipse[x radius=1, y radius=2];

        \draw (4, -.5) -- ++(-2,0)
        -- ++(0,1) node[midway, right] {N}
        -- +(2,0);
        \draw[arr] (1.2,-1) -- ++(1.1,0) node[near start, below] {$\vec v$};
        \node[below] at (0,-2) {$C=\partial F$};
        \node[anchor=south east] at (-1,0) {$I$};
        \draw[legreen,arr] (2,0.25) .. controls +(-1.5,0) and +(1,-1) .. (-2,1.6);
        \draw[legreen,arr] (2,-0.25) .. controls +(-1.5,0) and +(1,1) .. (-2,-1.6);
        \draw[legreen,arr] (2,0) -- (-2,0);
    \end{tikzpicture}
}


\newcommand{\tfigcylindricalConductorAndCoil}{
    \begin{tikzpicture}[scale=1]
        \coordinate (bottom) at (0,0);
        \coordinate (top) at (0,3);
        \coordinate (offset) at (1,0);

        % cylindrical conductor
        \draw (top)
        ellipse[x radius=1, y radius=.5];
        \draw
        ($(bottom) + (offset)$)
        arc[start angle=0, end angle=-180, x radius=1, y radius=.5];
        \draw
        ($(bottom) + (offset)$) -- ($(top) + (offset)$)
        ($(bottom) - (offset)$) -- ($(top) - (offset)$);
        \draw
        (top) -- ($(top) + (offset)$)
        node[midway, below] {$a$};
        \draw[distance marker]
        ($(bottom) + (offset) + (.2, 0)$) -- +(top)
        node[midway, right] {$h$};
        \node at ($(bottom) + (-2,.75)$) {$\mu$};
        % coil
        \begin{scope}[shift={(4,0)}]
            \draw (0,-.2)
            \foreach \a in {1,...,5}{
                .. controls +(0,.6) and +(0,.6)
                .. +(2,.35)
                .. controls +(0,-.6) and +(0,-.6)
                .. ++(0,.7)
            };
            \node at (1,0.1) {$\mu$};
            \draw[distance marker]
                (2.2, 0) -- +(0,3)
                node[midway, right] {$h$};
        \end{scope}
    \end{tikzpicture}
}


\newcommand{\tfigTwoInductors}{
    \begin{tikzpicture}[scale=1]
        \draw[rotate=20, midarrow=0.75]
        (0,0) ellipse [x radius=1, y radius=.5]
        node {$C_1$}
        node[xshift=13, yshift=-20] {$I_1$};
        \draw[rotate=-40, midarrow=0.75]
        (3,0) ellipse [x radius=1, y radius=.5]
        node {$C_2$}
        node[xshift=-13, yshift=-20] {$I_2$};
        \node at (0,-2) {$\mu$};
        \node at (3,0) {$\ldots$};
    \end{tikzpicture}
}

\newcommand{\tfigGrenzflaecheMagnetic}{
    \begin{tikzpicture}[
            scale=1,
            dosenfarbe/.style={blue!50!black!80!white}
        ]
        \draw
        (0,0) .. controls (1.8,.6) and +(-3,0) .. (13,.3)
        node[pos=.17, above] {$\vec B^{(2)},\vec H^{(2)},\vec D^{(2)}$}
        node[pos=.22, below] {$\vec B^{(1)},\vec H^{(1)},\vec D^{(1)}$}
        node[very near end, above] {$\vec k\ldots$ Flächenstrom}
        node[very near end, below] {Grenzfläche};

        \coordinate (P) at (5.5, .37);

        \node[point] at (P) {};
        \draw[arr]
        (P) -- +(0:1)
        node[midway, below] {$\hat{\vec t}$};
        \draw[arr]
        (P) -- +(90:1)
        node[midway, left] {$\hat{\vec n}$};

        \node at (5.2,.1) {$\otimes$}
        node at(4.7,-.2) {$\hat{\vec m}=\hat{\vec n}\times\hat{\vec t}$};

        \draw [dosenfarbe]
        (7,.6) -- +(0,-.4)
        arc[start angle=-180, end angle=0,x radius=.5, y radius=.15]
        -- +(0,.4)
        arc[start angle=0, end angle=180,x radius=.5, y radius=.15]
        node[midway, above] {Dose}
        arc[start angle=-180, end angle=0,x radius=.5, y radius=.15];
        \draw[dosenfarbe]
        (8.5,.55) rectangle +(1,-0.4) node[midway,above,yshift=10] {Schleife};
    \end{tikzpicture}
}


\newcommand{\tfigBewegungDurchLichtaetherA}{
    \begin{tikzpicture}[scale=3]
        \begin{scope}
            \draw[arr] (0,0) -- (0,1)
                node[anchor=south east] {$\vec e_2$};
            \draw[arr] (0,0) -- (1,0)
                node[anchor=north west] {$\vec e_1$};
            \node at (.5,.5) {Äther};
        \end{scope}
        \begin{scope}[shift={(2cm,0)}]
            \draw[arr] (0,0) -- (0,1)
                node[anchor=south east] {$\vec e_2$};
            \draw[arr] (0,0) -- (1,0)
                node[anchor=north west] {$\vec e_1$};
            \node at (.5,.5) {Erde};

            \draw[arr] (.2,.2) -- +(.6,0) node[anchor=west] {$\vec v=v \vec e_1$};            
        \end{scope}
    \end{tikzpicture}
}

\newcommand{\tfigBewegungDurchLichtaetherB}{
    \begin{tikzpicture}[scale=1.5]
        \begin{scope}
            \draw[arr] (0,0) -- +(1,0)
                node[midway, below] {$v\vec e_1$};
            \draw[arr] (1,0) -- +(1,0)
                node[midway, below] {$\vec c'$};
            \draw[arr] (0,.1) -- +(2,0)
                node[midway, above] {$c \vec n$};
        \end{scope}

        \begin{scope}[shift={(3cm,-0.5cm)}]
            \draw[arr] (0,0) -- (0,1)
                node[midway, left] {$\vec c'$};
            \draw[arr] (0,1) -- +(1,0)
                node[midway, above] {$v\vec e_1$};
            \draw[arr] (0,0) -- +(1,1)
                node[midway, anchor=north west] {$c\vec n$};

        \end{scope}
    \end{tikzpicture}
}

\newcommand{\tfigMichelsonInterferometer}{
    \begin{tikzpicture}[scale=2]
        \coordinate (ST) at (0,0);
        \coordinate (L) at (-1,0);
        \coordinate (S1) at (0,1);
        \coordinate (S2) at (1,0);
        \coordinate (D) at (0,-1);

        \draw[midarrow, red laser] (L) -- (ST);
        \draw[midarrow=.3, rmidarrow=.6, red laser] (ST) -- (S1);
        \draw[midarrow=.3, rmidarrow=.6, red laser] (ST) -- (S2);
        \draw[midarrow, red laser] (ST) -- (D);

        \draw[dashed, very thick] ($(ST) + (45:.3)$) -- ($(ST) - (45:.3)$);
        \draw[very thick] ($(S1) + (.3,0)$) -- ($(S1) - (.3,0)$)
            node[midway, above] {S1};
        \draw[very thick] ($(S2) + (0,.3)$) -- ($(S2) - (0,.3)$)
            node[midway, right] {S2};
        
        \draw[fill=black] ($(L) - (.4, .1)$) rectangle ($(L) + (0, .1)$);
        \draw[fill=black] ($(D) - (.3, .05)$) rectangle ($(D) + (.3, 0)$)
            node[midway, below, yshift=-1.2mm] {D};

        \draw[|-|] let \p1 = (S1), \p2 = (S2) in 
            (\x2 + .4cm,\y1) -- (\x2 +.4cm, 0) 
            node[midway, right] {$l_1$};
        \draw[|-|] let \p1 = (S1), \p2 = (S2) in 
            (0,\y1 +.4cm) -- (\x2,\y1 +.4cm) 
            node[midway, above] {$l_2$};

        \begin{scope}[shift={(-2cm, -1cm)}, scale=.5]
            \draw[arr] (0,0) -- (0,1) node[anchor=south east] {$\vec e_2$};
            \draw[arr] (0,0) -- (1,0) node[anchor=north west] {$\vec e_1$};
        \end{scope}
    \end{tikzpicture}
}


\newcommand{\tfigSRTGedankenExperimentLichtkegel}{
    \begin{tikzpicture}[scale=2.5]

        \coordinate (O) at (0,0);
        \coordinate (O2) at (2.5,0.3);
        
        \draw[arr] (0,0) -- (0,1) node[anchor=south west] {$z$} node[midway, left] {$\Sigma$};
        \draw[arr] (0,0) -- (1,0) node[anchor=south west] {$x$};
        \draw[arr] (0,0) -- (45:0.8) node[anchor=north west] {$y$};

        \foreach \angle in {0,45,...,360}{
            \draw[arr, red laser] (O) -- (\angle:0.3);
        }
        \draw[arr] ($.7*(O2)$) -- ($0.9*(O2)$) node[midway, above] {$\vec v$};

        \begin{scope}[shift={(O2)}, moving system color]
            \draw[arr] (0,0) -- (0,1) node[anchor=south west] {$z'$} node[midway, left] {$\Sigma'$};
            \draw[arr] (0,0) -- (1,0) node[anchor=south west] {$x'$};
            \draw[arr] (0,0) -- (45:0.8) node[anchor=north west] {$y'$};
        \end{scope}
    \end{tikzpicture}
}
\renewcommand{\tfigSRTGedankenExperimentLichtkegel}{
    \begin{tikzpicture}[scale=3]
        \begin{scope}          
            \draw[arr] (0,0) -- (0,1) node[anchor=south west] {$y,\textcolor{blue!60!black!70!white}{y'}$} 
            node[midway, left] {$\Sigma,\textcolor{blue!60!black!70!white}{\Sigma'}$};
            \draw[arr] (0,0) -- (1,0) node[anchor=north west] {$x,\textcolor{blue!60!black!70!white}{x'}$};
            \node at (.5,-.2) {$t_0=0$};
        \end{scope}
        \begin{scope}[xshift=2cm]
            \node at (.5,-.2) {$t_0+\diff t$};

            \coordinate (O) at (0,0);
            \coordinate (O2) at (0.6,0.2);
            \coordinate (L) at (.9,.8);

            \draw[arr, red laser] (O) -- (L);
            \draw[arr, red laser] (O2) -- (L);

            \draw[gray,decorate, decoration = {brace}] let \p1 = (L) in 
                (0,\y1) -- (\x1,\y1) node[midway, above] {$\diff x$};
            \draw[gray,decorate, decoration = {brace}] let \p1 = (L) in 
                (\x1,0) -- (\x1,\y1) node[midway, left] {$\diff y$};
            
            \draw[arr] (0,0) -- (0,1) node[anchor=south west] {$y$} node[midway, left] {$\Sigma$};
            \draw[arr] (0,0) -- (1,0) node[anchor=north west] {$x$};
            %\draw[arr] (0,0) -- (45:0.8) node[anchor=north west] {$y$};

            \draw[arr] (O) -- (O2) node[midway, above] {$\vec v$};

            \begin{scope}[shift={(O2)}, moving system color]
                \draw[arr] (0,0) -- (0,1) node[anchor=south west] {$y'$} node[midway, left] {$\Sigma'$};
                \draw[arr] (0,0) -- (1,0) node[anchor=north west] {$x'$};
                %\draw[arr] (0,0) -- (45:0.8) node[anchor=north west] {$y'$};
                \draw[blue!50!gray,decorate,decoration = {brace}] let \p1 = (L) in 
                    (0,\y1+1) -- +(\x1,0) node[midway, above] {$\diff x'$};
                \draw[blue!50!gray,decorate, decoration = {brace,mirror}] let \p1 = (L) in 
                    (\x1+1,0) -- +(0,\y1) node[midway, right] {$\diff y'$};
            \end{scope}
        \end{scope}
    \end{tikzpicture}
}

\newcommand{\tfigSRTLichtkegel}{
    \begin{tikzpicture}[scale=2.5]
        \draw[arr] (0,-1) -- (0,1) node[anchor=south west] {$ct$};
        \draw[arr] (-1,0) -- (1,0) node[anchor=north west] {$x$};

        \draw[color=blue!50!green] (-.9,.9) -- (.9,-.9) (-.9,-.9) -- (.9,.9) 
        node[right] {$-(ct)^2+x^2=0$};
    \end{tikzpicture}
}



\newcommand{\tfigGeschwindigkeitsaddition}{
    \begin{tikzpicture}[scale=2]
        \begin{scope}
            \draw[arr] (0,0) -- (0,1)
                node[midway, left] {$\Sigma$}
                node[anchor=south east] {$y$};
            \draw[arr] (0,0) -- (1,0)
                node[anchor=north west] {$x$};
        \end{scope}
        \begin{scope}[shift={(2cm,0)}]
            \draw[arr] (0,0) -- (0,1)
                node[midway, left] {$\Sigma'$}
                node[anchor=south east] {$y'$};
            \draw[arr] (0,0) -- (1,0)
                node[anchor=north west] {$x'$};

            \draw[arr] (.2,.2) -- +(.6,0) node[anchor=west] {$v$};
            \node (U) at (.9,.6) {$u$};
            \node[point] at (.4,.6) {} edge[arr] (U);
        \end{scope}
    \end{tikzpicture}
}

\newcommand{\tfigMinkowskiDiagramA}{
    \begin{tikzpicture}[scale=3.5]

        \coordinate (O) at (0,0);
        \coordinate (C1) at (.9,.9);
        \coordinate (C2) at (-.9,.9);
        \coordinate (C3) at (-.9,-.9);
        \coordinate (C4) at (.9,-.9);

        \fill[blue!2!white] (C2) -- (C4) -- (C3) -- (C1);
        \fill[red!2!white] (C2) -- (C4) -- (C1) -- (C3);

        \draw[arr] (-1,0) -- (1,0) node[anchor=north west] {$x$};
        \draw[arr] (0,-1) -- (0,1) node[anchor=south east] {$ct$};
        \draw[color=violet] (-1,1) -- (1,-1) (-1,-1) -- (1,1) 
            node[right] {Lichtkegel $-(ct)^2+x^2=0$};
        
        \node[fill=blue!2!white] at (0,.7) {Zukunft};
        \node[fill=blue!2!white] at (0,-.7) {Vergangenheit};
        \node[red!40!white!70!black] at (2,-.5) {raumartige Ereignisse} edge[arr, very thin] (.8,-.4);
        \node[blue!40!white!70!black] at (2,.5) {zeitartige Ereignisse} edge[arr, very thin] (.55,.7);
        
        \draw[color=green!30!black, name path={weltlinie}] (-115:1.1) .. controls +(72:.8) and +(-108:.8) ..  (65:1.1)
            node[anchor=south] {Weltlinie};
        \draw[transparent, name path={x parallel}] (0,.6) -- +(.5,0);
        \draw[name intersections={of=weltlinie and x parallel,by={P1}},dashed] 
            ($(P1)+(.1,.1)$) -- ($(P1)+(-.1,-.1)$) 
            ($(P1)+(-.1,.1)$) -- ($(P1)+(.1,-.1)$)
            node[point] at (P1) {};
    \end{tikzpicture}
}
\newcommand{\tfigMinkowskiDiagramB}{
    \begin{tikzpicture}[scale=1.5,clr/.style={legreen}]

        \coordinate (O) at (0,0);
    
        \draw[arr] (-3,0) -- (3,0) node[anchor=north west] {$x$};
        \draw[arr] (0,-3) -- (0,3) node[anchor=south east] {$ct$};
        \draw[color=violet] let \n1={3} in (-\n1,\n1) -- (\n1,-\n1) (-\n1,-\n1) -- (\n1,\n1);
    
        \draw[clr,arr,name path global/.expanded={ctp axis}] (-110:3) -- (70:3) node[anchor=south east] {$ct'$};
        \draw[clr,arr,name path global/.expanded={xp axis}] (-160:3) -- (20:3) node[anchor=north west] {$x'$};
        \draw[clr,arr,dashed] (-2.9,.4) -- +(20:6) node[anchor=north west] {$ct'=\mathrm{const}$};
        \draw[clr,arr,dashed] (-2.9,-2.85) -- +(70:6) node[anchor=south east] {$x'=\mathrm{const}$};
        
        \draw[name path global/.expanded={hyp top}] plot[domain=-3:3,samples=30] ({\x},{sqrt(1+\x*\x)});
        \draw[name path global/.expanded={hyp right}] plot[domain=1:3,samples=50] ({\x},{sqrt(\x*\x-1)});
        \draw plot[domain=1:3,samples=50] ({\x},{-sqrt(\x*\x-1)});
        
        \draw[clr] (0,0) -- (1.5,0) arc [start angle=0, end angle=20, radius=1.5];
        \draw (10:1.3) node[clr] {$\delta$};
        \draw[clr] (0,0) -- (0,1.5) arc [start angle=90, end angle=70, radius=1.5];
        \draw (80:1.3) node[clr] {$\delta$};
    
        \node[point] at (1,0) [label={south west}:1] {};
        \node[point] at (0,1) [label={south west}:1] {};
        
        \path [name intersections={of=ctp axis and hyp top,by={P1}}]
            (P1) node [point,clr,label={south east,clr}:1] {};
        \path [name intersections={of=xp axis and hyp right,by={P1}}]
            (P1) node [point,clr,label={north west,clr}:1] {};
    \end{tikzpicture}
}

\usepackage{hyperref} % include hyperref as last package!



\title{\includegraphics{images/Titel.pdf}}
\subtitle{Wintersemester 2021/2022}
\date{}
\author{von Kyano Levi\\bei Professor Holger Stark}

% -----------------------------------
% -----------------------------------
% -----------------------------------

\begin{document}

\frontmatter
\maketitle


\tableofcontents

% -----------------------------------
% -----------------------------------

\mainmatter
% !TEX root = Theo_III.tex

\chapter{Einleitung\label{einleitung}}

\section{Geschichte}

\begin{itemize}
	\item \textbf{1785} \textendash{} Charles Augustin de Coulomb: Entdeckung des Coulombsches Gesetzes.

	\item \textbf{1800} \textendash{} Alessandro Volta: Erfindung der erstern Batterie, die Voltasche Säule.

	\item \textbf{1820} \textendash{}  Hans Christian \O{}rsted: Das \O{}rstedsche Gesetz beschreibt, dass elektrische Ströme ein Magnetfeld erzeugen.

	\item \textbf{1820-25} \textendash{}   André-Marie Ampère: Entdeckung der Grundlagen der Magnetostatik durch Messungen.

	\item \textbf{1831} \textendash{}  Michael Faraday: Beschreibung der magnetischen Induktion.

	\item \textbf{1852} \textendash{}  Michael Faraday: Formulierung des Nahwirkungsstandpunktes (Beschreibung elektrischer Phänomene über Felder statt Kräfte).

	\item \textbf{1864} \textendash{} James Clerk Maxwell: Formulierung der Maxwell-Gleichungen als fundamentale Feldgleichungen des elektromagnetischen Feldes und Nutzung von elektrischen und magnetischen Hilfsfeldern für die physikalische Beschreibung in Materie sowie Äußerung der Vermutung, dass Licht eine elektromagnetische Welle ist.

	\item \textbf{1886} \textendash{} Heinrich Hertz: Nachweis elektromagnetischer Wellen und Postulierung eines Äthers als hypothetisches Ausbreitungsmedium.

	\item \textbf{1881} \textendash{} Michelson-Morley-Experiment: Konstanz der Lichtgeschwindigkeit unabhängig von Beobachter und Quelle ${\Rightarrow}$ ein absolutes Bezugssystem Äther existiert nicht.

	\item \textbf{1905} \textendash{} Albert Einstein: spezielle Relativitätstheorie.

\end{itemize}





\section{Inhalt}

Der Inhalt dieser Vorlesung gliedert sich in folgende Abschnitte:

\begin{itemize}
	\item Einleitung

	\item Elemente der Vektoranalysis

	\item Elektrostatik

	\item Elektrische Felder in Materie

	\item Magnetostatik

	\item Grundgleichungen der Elektrodynamik: Die Maxwellschen Gleichungen

	\item Spezielle Relativitätstheorie

	\item Ebene elektromagnetische Wellen

	\item Elektromagnetische Felder bei vorgegebenen Ladungen und Strömen
\end{itemize}



\section{Grundlegende Konstanten der Elektrodynamik}

Für Konstanten deren Wert per Definition festgelegt wurde, wird ein $\equiv $-Zeichen verwendet.


\begin{table}[H]
	\centering
	\begin{tabular}{|l|l|} \hline
		\textbf{Konstante}         & \textbf{Wert}                                                     \\\hline
		Vakuumlichtgeschwindigkeit & \centering\arraybackslash{}$c_{0}\equiv \SI{299792458}{\m\per\s}$ \\
		Elektrische Feldkonstante  & $\varepsilon _{0}=\SI{8,8541878128e-12}{\A\s\per\V\per\m}$        \\
		Magnetische Feldkonstante  & $\mu _{0}=\SI{1,25663706212e-6}{\N\per\square\A}$                 \\
		\hline
	\end{tabular}
\end{table}




\section{Grundlegende Formeln der Elektrodynamik}

Maxwellsche Feldgleichungen:
\begin{align*}
	\nabla \cdot \vec {D}                        & =\rho _{f}                                                       &
	\nabla \cdot \vec {B}                        & =0                                                               &
	\nabla \times \vec {E}                       & =-\frac{\partial \vec {B}}{\partial t}                           &
	\nabla \times \vec {H}                       & =\vec {j}_{f}+\frac{\partial \vec {D}}{\partial t}                 \\
	\oint_A D \cdot\diffa{\vec{A}}               & =Q                                                               &
	\oint_A B \cdot\diffa{\vec{A}}               & =0                                                               &
	\oint_{\partial A}\vec E\cdot \diffa{\vec s} & =    -\int_A\frac{\partial\vec B}{\partial t}\cdot\diffa{\vec A} &
	\oint_{\partial A}\vec H\cdot \diffa{\vec s} & = I + \int_A\frac{\partial\vec D}{\partial t}\cdot\diffa{\vec A}
\end{align*}
% \begin{table}[H]
% 	\centering\begin{tabular}{llll}
% 		$\displaystyle \nabla \times \vec {E}=-\frac{\partial \vec {B}}{\partial t} $                                                  &
% 		$\displaystyle \nabla \cdot \vec {B}=0 $                                                                                       &
% 		$\displaystyle \nabla \times \vec {E}=-\frac{\partial \vec {B}}{\partial t} $                                                  &
% 		$\displaystyle \nabla \times \vec {H}=\vec {j}_{f}+\frac{\partial \vec {D}}{\partial t} $                                        \\ \\
% 		$\displaystyle \oint_A D \cdot\diffa{\vec{A}}=Q $                                                                              &
% 		$\displaystyle \oint_A B \cdot\diffa{\vec{A}}=0 $                                                                              &
% 		$\displaystyle \oint_{\partial A}\vec E\cdot \diffa{\vec s} =    -\int_A\frac{\partial\vec B}{\partial t}\cdot\diffa{\vec A} $ &
% 		$\displaystyle \oint_{\partial A}\vec H\cdot \diffa{\vec s} = I + \int_A\frac{\partial\vec D}{\partial t}\cdot\diffa{\vec A} $
% 	\end{tabular}
% 	\label{<label>}
% \end{table}

Materialgleichungen (allgemein und in linearen, isotropen Medien (Vakuum: $\epsilon_r=\mu_r=1$):
\begin{align*}
	\vec {D} & =\varepsilon _{0}\vec {E}+\vec {P}        & \vec {B} & =\mu _{0}\left(\vec {H}+\vec {M}\right) \\
	\vec {D} & =\varepsilon _{0}\varepsilon _{r}\vec {E} & \vec {B} & =\mu _{0}\mu _{r}\vec {H}
\end{align*}

Statische Potentiale:
\begin{align*}
	\phi(\vec r)   & =\frac{1}{4\pi\epsilon_0}\int \frac{\rho(\vec r')}{\left|\vec r-\vec r'\right|}\diffa[3]{\vec r'} &
	\vec A(\vec r) & =\frac{\mu_0}{4\pi}\int \frac{\vec j(\vec r')}{\left|\vec r-\vec r'\right|}\diffa[3]{\vec r'}       \\
	\vec E         & =-\nabla \phi                                                                                     &
	\vec B         & =\nabla \times\vec A                                                                              &
\end{align*}

Biot-Savart-Gesetz historisch (für unendlich lange, gerade Leiter) und modern:
\begin{align*}
	B(\rho) = \frac{\mu_0}{2\pi}\frac{I}{\rho} \qquad B(\vec r)=\frac{\mu_0}{4\pi}\int\vec j(\vec r')\times \frac{\vec r-\vec r'}{\left|\vec r-\vec r'\right|^3}\diffa[3]{\vec r'}
\end{align*}

Feldenergie und Feldenergiedichte:
\begin{align*}
	u(\vec r) & =\frac{1}{2}\vec E\cdot \vec D                      & u(\vec r) & =\frac{1}{2}\vec H\cdot \vec B                       \\
	U(\vec r) & =\frac{1}{2}\int\vec E\cdot \vec D\diffa[3]{\vec r} & U(\vec r) & =\frac{1}{2}\int\vec H\cdot \vec B \diffa[3]{\vec r}
\end{align*}


Multipole:
\begin{align*}
	p            & = \int\rho(\vec r)\vec r'\diffa[3]{\vec r'}                                                                                   & m                            & =\frac{1}{2}\int \vec r'\times\vec j(\vec r)	\diffa[3]{\vec r'} \\
	Q_{ij}       & =\int (3x_i'x_j'-\delta_{ij} {\vec r'}^2)\diffa[3]{\vec r'}                                                                   &                              &                                                                \\
	\phi(\vec r) & = \frac{1}{4\pi\epsilon_0}\left(\frac{q}{r}+\frac{\vec p\cdot \vec r}{r^3}+\frac{1}{2}Q_{ij} \frac{x_ix_j}{r^5}+\ldots\right) & \vec A_\text{Dipol} (\vec r) & =\frac{\mu_0}{4\pi} \frac{\vec m\times \vec r}{r^3}
\end{align*}

Randbedingungen an Grenzflächen:
\begin{align*}
	\hat{\vec n}\times(\vec E^{(1)}-\vec E^{(2)}) & = 0      & \hat{\vec n}\times(\vec H^{(1)}-\vec H^{(2)}) & = \vec k\cdot \vec m \\
	\hat{\vec n}\cdot(\vec D^{(1)}-\vec D^{(2)})  & = \sigma & \hat{\vec n}\cdot(\vec B^{(1)}-\vec B^{(2)})  & = 0
\end{align*}

Lorentzkraft und Lorentzkraft-Dichte:
\begin{align*}
	\vec F_\text{L} = q(\vec E+\vec v\times \vec B) = \int \vec f_\text{L}\diffa[3]{\vec r}, \qquad \vec f_\text{L}=\rho\vec E+\vec j\times \vec B
\end{align*}

Kontinuitätsgleichung
\begin{align*}
	\frac{\partial w}{\partial t}+\nabla\cdot \vec S & = -\vec j\cdot \vec E & w & =\frac{1}{2}\left(\vec E\cdot \vec D+\vec H\cdot \vec B\right) &
	\vec S                                           & =\vec E\times \vec H
\end{align*}

Relationen von Lichtgeschwindigkeit, Feldkonstanten und Brechungsindex:
\begin{align*}
	c_{0} & =\frac{1}{\sqrt{\varepsilon _{0}\mu _{0}}} & n & =\frac{c_{0}}{c} & \omega^2 & =c^2 k^2
\end{align*}
In linearen, isotropen Medien gilt
\begin{equation*}
	n=\sqrt{\varepsilon\mu}
\end{equation*}

Vektoranalysis
\begin{align*}
	\nabla\cdot (\vec a\times \vec b)   & =\vec b\cdot(\nabla\times\vec a)-\vec a\cdot(\nabla\times\vec b)                                             \\
	\nabla\times (\vec a\times  \vec b) & =\vec a(\nabla\cdot \vec b)-(\vec a\cdot \nabla)\vec b+(\vec b\cdot \nabla)\vec a-\vec b(\nabla\cdot \vec a) \\
	\nabla \times (\nabla\times \vec a) & =\nabla(\nabla\cdot \vec a) - \Delta\vec a                                                                   \\
	\nabla\times(\vec a\times \vec b f) & = f\nabla\times(\vec a\times  \vec b)-(\vec a\times  \vec b)\times \nabla f                                  \\
	\nabla\cdot(f\vec a)                & = \vec a\cdot (\nabla f)+f\nabla \cdot \vec a
\end{align*}
\begin{align*}
	\nabla \frac{1}{r} & =-\frac{\vec {r}}{r^{3}} & \Delta \frac{1}{r} & =-4\pi\delta(\vec r)
\end{align*}

Operatoren in Kugelkoordinaten
\begin{align*}
	\nabla            & =\vec e_r \frac{\partial}{\partial r}+\frac{1}{r}\vec e_\theta \frac{\partial}{\partial \theta} + \frac{1}{r\sin\theta}\vec e_\varphi \frac{\partial}{\partial\varphi}                                                                                                                    \\
	\nabla\cdot\vec F & = \frac{1}{r^2} \frac{\partial}{\partial r} (r^2 F_r)+ \frac{1}{r \sin\theta} \frac{\partial}{\partial\theta}(\sin\theta F_\theta)+\frac{1}{r\sin\theta} \frac{\partial}{\partial\varphi}F_\varphi                                                                                        \\
	\nabla^2          & =\frac{1}{r^2} \frac{\partial}{\partial r}\left(r^2 \frac{\partial}{\partial r}\right) +\frac{1}{r^2}\left[ \frac{1}{\sin\theta} \frac{\partial}{\partial\theta}\left(\sin\theta \frac{\partial}{\partial\theta}\right)+\frac{1}{\sin^2\theta}\frac{\partial^2}{\partial\varphi^2}\right]
\end{align*}

Operatoren in Zylinderkoordinaten
\begin{align*}
	\nabla             & = \vec e_\rho \frac{\partial}{\partial\rho} + \frac{1}{\rho}\vec e_\varphi \frac{\partial}{\partial\varphi}+\vec e_z \frac{\partial}{\partial z}                                    \\
	\nabla\cdot \vec F & = \frac{1}{\rho} \frac{\partial}{\partial\rho} \left(\rho F_\rho\right) + \frac{1}{\rho} \frac{\partial}{\partial\varphi}F_\varphi + \frac{\partial}{\partial z} F_z                \\
	\Delta             & = \frac{1}{\rho} \frac{\partial}{\partial\rho} \left(\rho \frac{\partial}{\partial\rho}\right)+\frac{1}{\rho^2}\frac{\partial^2}{\partial\varphi^2}+\frac{\partial^2}{\partial z^2}
\end{align*}

% !TEX root = Theo_III.tex


\chapter{Elemente der Vektoranalysis\label{elemente_der_vektoranalysis}}

\section{Vektoranalysis}

\subsection{Gradient und Nabla-Operator}

Der Gradient eines skalaren Feldes $U$ ist definiert über das totale Differential:
\begin{equation*}
	\diff U=\grad U\cdot \diff \vec {r}=\nabla U\cdot \diff \vec {r}
\end{equation*}
Der Gradient steht senkrecht auf den Äquipotentiallinien. Für kartesische Koordinaten gilt
\begin{equation*}
	\nabla =\vec {e}_{x}\frac{\partial }{\partial x}+\vec {e}_{y}\frac{\partial }{\partial y}+\vec {e}_{z}\frac{\partial }{\partial z}=\sum _{i}\vec {e}_{i}\nabla _{i},
\end{equation*}
während für krummlinige allgemein gilt, dass
\begin{equation*}
	\nabla =\sum _{i}\vec {e}_{i}\frac{1}{\left| \partial \vec {r}/\partial x_{i}\right| }\frac{\partial }{\partial x_{i}}.
\end{equation*}
\subsection{Divergenz eines Vektorfeldes\label{ref-007}}

Die Divergenz eines Vektorfeldes $\vec {a}$ wird beschrieben durch
\begin{equation*}
	\divg \vec {a}\left(\vec {r}\right)=\nabla \cdot \vec {a}\left(\vec {r}\right).
\end{equation*}
Sie gibt die Quellenhaftigkeit von $\vec {a}$ an. In kartesischen Koordinaten ist $\divg \vec {a}=\sum _{i}\nabla _{i}a_{i}$.



\begin{figure}[htb]
	\centering
	\includegraphics{vecanalysis_flow.pdf}
	\caption{}
	\label{fig:vecanalysis_flow}
\end{figure}

Betrachte zum Verständnis ein kleines Volumen $\Delta  V$ bei $\vec {r}$. Die Normalen $\hat{\vec {\nu }}$ zeigen überall nach außen. Der Fluss aus $\Delta  V$ heraus ist
\begin{equation*}
	q\left(r\right)\Delta  V,
\end{equation*}
wobei $q\left(r\right)=\divg \vec {a}$. Wir können sagen, dass
\begin{align*}
	q\left(\vec {r}\right)=\divg \vec {a}=\begin{cases} >0, & \text{Quelle von }\vec {a}                     \\
              <0, & \text{Senke von }\vec {a}                      \\
              =0, & \text{was reinflie\ss t},\text{flie\ss t raus}
	                                      \end{cases} .
\end{align*}



\subsection{Rotation eines Vektorfeldes}

Die Rotation eines Vektorfeldes $\vec {a}$ ist definiert als
\begin{equation*}
	\rot \vec {a}\left(\vec {r}\right)=\nabla \times \vec {a}\left(\vec {r}\right)
\end{equation*}


\begin{figure}[htb]
	\centering
	\includegraphics{vecanaylsis_curl.pdf}
	\caption{}
	\label{fig:vecanaylsis_curl}
\end{figure}

und es wird auch als das Wirbelfeld von $\vec {a}$ bezeichnet. Wieder ist die Darstellung in kartesischen Koordinaten einfach: $\left(\rot \vec {a}\right)_{i}=\varepsilon _{ijk}\partial _{{x_{j}}}a_{k}$.

Wir schauen uns ein kleines orientiertes Flächenelement $\Delta  f$ an. Dann ist die Verwirbelung/Zirkulation um $\Delta  f$
\begin{equation*}
	\sum _{C}\vec {a}\cdot \hat{\vec {t}}_{C}\Delta  r_{C}=\rot \vec {a}\cdot \Delta  f
\end{equation*}
mit Tangentialvektor $\hat{\vec {t}}_{C}$ und Parallelkomponente $\vec {a}\cdot \hat{\vec {t}}_{C}$. Wir bezeichnen $\vec {\omega }=\rot \vec {a}$ als lokale Wirbelstärke.

Allgemein gilt, dass Gradientenfelder wirbelfrei sind,
\begin{equation*}
	\vec {a}=\grad U\equivalence \rot \vec {a}=0\quad\mathrm{bzw.}\quad  \rot \left(\grad U\right)=0
\end{equation*}
und Wirbelfelder quellenfrei sind,
\begin{equation*}
	\divg \vec {B}=0\equivalence \vec {B}=\rot \vec {A}\quad\mathrm{bzw.}\quad  \divg \left(\rot \vec {A}\right)=0.
\end{equation*}
Wir definieren ferner den Laplace-Operator als
\begin{equation*}
	\Delta  \equiv \nabla ^{2}\equiv \nabla \cdot \nabla ,
\end{equation*}
für den in kartesischen Koordinaten gilt:
\begin{equation*}
	\nabla ^{2}=\partial _{x}^{2}+\partial _{y}^{2}+\partial _{z}^{2}.
\end{equation*}
\subsection{Fundamentalsatz der Vektoranalysis (Helmholtz-Theorem)\label{ref-009}}

Das Helmholtz-Theorem besagt, dass Quellen und Wirbel ein Vektorfeld $\vec {a}\left(\vec {r}\right)$ eindeutig bestimmen. Ein Vektorfeld kann also in ein Rotationsfeld und ein Wirbelfeld aufgeteilt werden:
\begin{equation*}
	\vec {a}=\underset{\substack{
			\omega =\rot \vec {a}=\rot \vec {a}_{t} \\
			\divg \vec {a}_{t}=0 \\
			\text{Wirbel}!
		}}{\underbrace{\vec {a}_{t}}}+\underset{\substack{
			\rho =\divg \vec {a}=\divg \vec {a}_{l} \\
			\rot \vec {a}_{l}=0                     \\
			\text{Quellen}!
		}}{\underbrace{\vec {a}_{l}}}+\underset{\substack{
			\rot \vec {a}_{r}=\divg \vec {a}_{r}=0 \\
			\text{Randbedingungen}                 \\
			\hat{\vec {\nu }}\cdot \vec {a}=f\left(\vec {r}\right),\vec {r}\in \partial V
		}}{\underbrace{\vec {a}_{r}}}
\end{equation*}
Eine zusätzliche, sowohl quellen- als auch wirbelfreie Komponente kann vorkommen, um Randbedingungen zu erfüllen oder einen konstanten Untergrund zu addieren.

Ebene Transversalwellen ($e^{i\vec {k}\cdot \vec {r}}\perp \vec {k}$) sind zum Beispiel quellenfrei, ebene Longitudinalwellen ($e^{i\vec {k}\cdot \vec {r}}\parallel \vec {k}$) sind dagegen wirbelfrei, denn
\begin{equation*}
	\divg \left(e^{i\vec {k}\cdot \vec {r}}\right)=i\vec {k}\cdot e^{i\vec {k}\cdot \vec {r}},\quad \rot \left(e^{i\vec {k}\cdot \vec {r}}\right)=i\vec {k}\times e^{i\vec {k}\cdot \vec {r}}.
\end{equation*}




\section{Integration von Feldern}


\subsection{Linienintegrale}


\begin{equation*}
	\int _{C}\vec {a}\left(\vec {r}\right)\cdot \diff \vec {r}=\int _{C}\vec {a}\left(\vec {r}\left(s\right)\right)\cdot \frac{\diff \vec {r}}{\diff s}\diff s
\end{equation*}


\begin{figure}[htb]
	\centering
	\includegraphics{vecanaylsis_curve.pdf}
	\caption{}
	\label{fig:vecanaylsis_curve}
\end{figure}

Parameterdarstellung: $\vec {r}=\vec {r}\left(s\right)\rightarrow \diff \vec {r}=\frac{\diff \vec {r}}{\diff s}\diff s$ mit der Bogenlänge $s$.

Für rotationsfreie (Einschränkung, siehe Satz von Poincaré) Felder ist das Linienintegral zwischen zwei Punkten wegunabhängig:
\begin{equation*}
	\oint \vec {a}\cdot \diff \vec {r}=0\equivalence \vec {a}\left(\vec {r}\right)=\nabla \varphi \equivalence \rot \vec {a}=\vec {0}.
\end{equation*}


\subsection{Satz von Stokes}

\begin{equation*}
	\underset{\text{Fluss von }\rot \vec {a}~\text{durch} F}{\underbrace{ \int _{F}\rot \vec {a}\cdot \diff f}}=\underset{\text{Zirkulation von }\vec {a}~\text{entlang}~ C=\partial F}{\underbrace{\oint _{C=\partial F}\vec {a}\cdot \diff \vec {r}}}
\end{equation*}

\begin{figure}[htb]
	\centering
	\includegraphics{vecanaylsis_normal.pdf}
	\caption{}
	\label{fig:vecanaylsis_normal}
\end{figure}

Die Kurve $C$ ist dabei stets so orientiert, dass sie der Rechte-Hand-Regel folgt. Von außen (die Seite, nach der der Normalenvektor $\diff \vec {f}$ zeigt) betrachtet geht die Kurve gegen den Uhrzeigersinn.


\subsection{Satz von Gauß}

\begin{equation*}
	\underset{\text{Quellen von }\vec {a}~\mathrm{in} V}{\underbrace{ \int _{V}\divg \vec {a}\cdot \mathrm{dV}}}=\underset{\substack{
			\text{Fluss von }\vec {a}~\text{durch } \\
			\partial V \mathrm{aus}~V \text{heraus}
		}}{\underbrace{\int _{\partial V}\vec {a}\cdot \diff f}}
\end{equation*}
Aus den Satz von Gauß abgeleiteten Formen:
\begin{itemize}
	\item $\vec {a}=g\vec {e}_{i}\rightarrow \int _{V}\frac{\partial }{\partial x_{i}}g\diff V=\int _{\partial V}g\diff f_{i}$

	\item $g=a_{j}\rightarrow \int _{V}\rot \vec {a}\diff V=\int _{\partial V}\diff \vec {f}\times \vec {a}$

	\item Greensche Identitäten (diese finden ihre Anwendung in der Potentialtheorie, hierzu wird $\nabla ^{2}\varphi $ verwendet). $\vec {a}_{1}=\varphi \nabla \psi , \vec {a}_{2}=\psi \nabla \varphi $.

	      \begin{enumerate}[a]
		      \setcounter{enumii}{14}

		      \item[o] 1. Identität: $\int \nabla \cdot \vec {a}_{1}\diff V$
			      \begin{equation*}
				      \int _{V}\left(\nabla \varphi \cdot \nabla \psi +\varphi \nabla ^{2}\psi \right)\diff V=\int _{\partial V}\varphi \nabla \psi \cdot \diff f
			      \end{equation*}
		      \item[o] 2. Identität: $\int \left(\nabla \cdot \vec {a}_{1}-\nabla \cdot \vec {a}_{2}\right)\diff V$ (Greenscher Satz)
			      \begin{equation*}
				      \int _{V}\left(\varphi \nabla ^{2}\psi -\psi \nabla ^{2}\varphi \right)\diff V=\int _{\partial V}\left(\varphi \nabla \psi -\psi \nabla \varphi \right)\cdot \diff f
			      \end{equation*}
	      \end{enumerate}
\end{itemize}

% !TEX root = Theo_III.tex

\chapter{Elektrostatik}

Die Elektrostatik behandelt elektrische Felder ruhender oder langsam bewegter elektrischer Ladungen. In den folgenden Kapiteln werden die Grundgesetze der Elektrostatik aus dem Coulomb-Gesetz abgeleitet.

\section{Bemerkungen zur elektrischen Ladung}

Es gibt zwei Arten von Ladungen: positive und negative Ladung. Die Ladung ist eine diskrete Größe und nimmt stets ein ganzzahliges Vielfaches der sogenannten Elementarladung $e_{0}$ an:
\begin{equation*}
	e_{0}=\SI{1.602176624e-19}{\coulomb}
\end{equation*}
Diese wurde zuerst bei dem Millikan-Versuch bestimmt. So trägt zum Beispiel das Proton die Ladung $+e_{0}$ und das Elektron die Ladung $-e_{0}$. Quarks haben zwar Bruchteile der Elementarladung, treten aber nie frei, sondern nur in Kombinationen auf, die ein Vielfaches der Elementarladung bilden.

Es gilt strenge Ladungserhaltung:
\begin{formal}
	In einem abgeschlossenen System bleibt die Summe aller Ladungen konstant.
\end{formal}
Eine Ladung auf einem infinitesimalen Raum wird als Punktladung bezeichnet. In der Elektrostatik und der Elektrodynamik wird häufig mit der Ladungsdichte $\rho $ gerechnet. Für eine einzige Punktladung $q$ (zum Beispiel ein Proton oder Elektron) am Ort $\vec {r}_{0}$ gilt für die Ladungsdichteverteilung
\begin{equation*}
	\rho \left(\vec {r}\right)=q\delta \left(\vec {r}-\vec {r}_{0}\right).
\end{equation*}
Daraus lässt sich die Ladungsdichte für viele Punktladungen $q_{i}$ an Orten $\vec {r}_{i}$ verallgemeinern:
\begin{equation*}
	\rho \left(\vec {r}\right)=\sum _{i}q_{i}\delta \left(\vec {r}-\vec {r}_{i}\right)
\end{equation*}
Im Grenzwert für kleinste Abstände kann man schließlich auch mit kontinuierlichen Ladungsdichten rechnen:
\begin{equation*}
	\rho \left(\vec {r}\right)=\frac{\diff Q}{\diff V}
\end{equation*}
Die gesamte Ladung in einem Volumen $V$ ist also
\begin{equation*}
	Q=\int _{V}\diffa[3]{\vec{r}}\rho \left(\vec {r}\right).
\end{equation*}



\section{Coulombsches Gesetz und elektrisches Feld}

Im Alltag machen wir die Erfahrung, dass sich gleichnamige (also zum Beispiel zwei positive) Ladungen abstoßen, während zwischen ungleichnamigen Ladungen eine anziehende Kraft wirkt. Diese Kraft ist ein Vektor im Sinne der Newtonschen Mechanik und unterliegt also dem Superpositionsprinzip.


\subsection{Coulombsches Gesetz}

Die Kraft, die eine Ladung $q_{2}$ am Ort $\vec {r}_{2}$ auf eine Ladung $q_{1}$ am Ort $\vec {r}_{1}$ ausübt (siehe \Abbref{fig:coulomb_point_charges}), berechnet sich durch
\begin{equation}
	\label{3.1}
	\boxed{\vec {F}_{1}=kq_{1}q_{2}\frac{\vec {r}_{1}-\vec {r}_{2}}{\left| \vec {r}_{1}-\vec {r}_{2}\right| ^{3}}=-\vec {F}_{2}.}
\end{equation}
Dieser Zusammenhang ist als Coulombsches Gesetz bekannt und wurde experimentell gefunden. Die Proportionalitätskonstante $k$ ist dabei
\begin{equation*}
	k=\frac{1}{4\pi \varepsilon _{0}}.
\end{equation*}

\begin{figure}[htb]
	\centering
	\tfigCoulombPointCharges
	\caption{Die Kraft auf zwei Punktladungen $q_1$ und $q_2$ an Orten $\vec r_1$ und $\vec r_2$ wird durch das Coulombsche Gesetz beschrieben und ist invers proportional zum Qudrat des Abstands $\left|\vec r_1-\vec r_2\right|$. }
	\label{fig:coulomb_point_charges}
\end{figure}

mit Dielektrizitätskonstante $\varepsilon _{0}=\SI{8,8541878128e-12}{\farad\per\m}$. Das Coulombsche Gesetz hat die gleiche Form wie das Newtonsche Gravitationsgesetz, aber hier kann die Kraft auch abstoßend wirken, weil die Ladung anders als die Masse negativ sein kann. Genauso wie beim Gravitationsgesetz ist die Kraft antiproportional zum Quadrat des Abstands der Ladungen.\footnote{Es ist möglich, dass die Proportionalität nicht exakt $\vec {F}\propto r^{-2}$ ist, aber es ist durch Experimente bestätigt worden, dass für einen Ansatz $F\propto r^{-2-\varepsilon }$ zumindest $\varepsilon <3\cdot 10^{-16}$ ist und für einen Ansatz $F\propto e^{-\frac{r}{\xi }}r^{-2}$ (siehe sogenanntes Yukawa-Potential) wenigstens $\xi >\SI{1e8}{\m}$. }

Mithilfe des Coulombschen Gesetzes können wir nach dem Superpositionsprinzip die Kraft auf eine Testladung $q_{0}$ am Ort $\vec {r}_{0}$ durch mehrere Ladungen $q_{i}$ bestimmen:
\begin{equation}
	\label{3.2}
	\vec {F}=\frac{q_{0}}{4\pi \varepsilon _{0}}\sum _{i}q_{i}\frac{\vec {r}_{0}-\vec {r}_{i}}{\left| \vec {r}_{0}-\vec {r}_{i}\right| ^{3}}
\end{equation}
Dieser Ansatz ist der Fernwirkungsstandpunkt (die Kraft wirkt über die Ferne hinweg). Seit Veröffentlichung der Relativitätstheorie ist aber bekannt, dass sich nichts schneller als mit Vakuumlichtgeschwindigkeit bewegen kann \textendash{} also auch keine Kraftwirkung.

Daher führt man den sogenannten Nahwirkungsstandpunkt ein, bei dem man man ein elektrisches Feld $\vec {E}$ betrachtet, das durch Ladungen $q_{i}$ erzeugt wird ($\vec {E}$ zeigt weg von positiven Ladungen und hin zu den negativen):
\begin{equation}
	\vec {E}\left(\vec {r}\right)=\frac{1}{4\pi \varepsilon _{0}}\sum _{i}q_{i}\frac{\vec {r}_{0}-\vec {r}_{i}}{\left| \vec {r}_{0}-\vec {r}_{i}\right| ^{3}},\quad \vec {E}\left(\vec {r}\right)=\frac{1}{4\pi \varepsilon _{0}}\int \rho \left(\vec {r}'\right)\frac{\vec {r}-\vec {r}'}{\left| \vec {r}-\vec {r}'\right| ^{3}}\diffa[3]{\vec{r}'}
\end{equation}
Damit ergibt sich die folgende Kraft auf eine Testladung $q_{0}$:
\begin{equation}
	\vec {F}=q_{0}\vec {E}\left(\vec {r}_{0}\right)
\end{equation}


\section{Feldgleichungen der Elektrostatik}

\subsection{Grundlagen}

Wir definieren zunächst das elektrostatische Potential:
\begin{equation}
	\label{3.3}
	\boxed{\vec {E}=-\nabla \phi ,\quad \phi \left(\vec {r}\right)=\frac{1}{4\pi \varepsilon _{0}}\int \frac{\rho \left(\vec {r}'\right)}{\left| \vec {r}-\vec {r}'\right| }\diffa[3]{\vec{r}'}}
\end{equation}
Weil $\vec {E}$ ein Potentialfeld ist, ist $\rot \vec {E}=0$. Das elektrostatische Feld ist also wirbelfrei.

Zum Beispiel ist das Potential einer Punktladung $\rho \left(\vec {r}\right)=q\delta \left(\vec {r}-\vec {r}_{0}\right)$ nach obiger Formel\footnote{Hinweis: Es gilt
	\begin{equation*}
		\nabla \frac{1}{\left| \vec {r}-\vec {r}\mathrm{'}\right| }=-\frac{1}{\left| \vec {r}-\vec {r}\mathrm{'}\right| ^{2}}\nabla \left| \vec {r}-\vec {r}\mathrm{'}\right| \overset{\nabla r=\frac{\vec {r}}{r}=\hat{\vec {r}}}{=}-\frac{\vec {r}-\vec {r}\mathrm{'}}{\left| \vec {r}-\vec {r}\mathrm{'}\right| ^{3}}
	\end{equation*}
}:
\begin{equation*}
	\phi \left(\vec {r}\right)=\frac{1}{4\pi \varepsilon _{0}}\frac{q}{\left| \vec {r}-\vec {r}_{0}\right| }
\end{equation*}
Die Quellen des elektrischen Feldes werden durch die Divergenz von $\vec E$ beschrieben,
\begin{equation*}
	\divg \vec {E}=-\nabla ^{2}\phi =-\frac{1}{4\pi \varepsilon _{0}}\int \rho \left(\vec {r}\mathrm{'}\right)\underset{=-4\pi \delta \left(\vec {r}-\vec {r}\mathrm{'}\right)}{\underbrace{\nabla ^{2}\frac{1}{\left| \vec {r}-\vec {r}\mathrm{'}\right| }}}\diff ^{3}r\mathrm{'}=\frac{1}{\varepsilon _{0}}\rho \left(\vec {r}\right).
\end{equation*}


\subsection{Feldgleichungen der Elektrostatik}

Die soeben gefundenen Zusammenhänge werden als Feldgleichungen der Elektrostatik bezeichnet:
\begin{align}
	\label{3.4}
	\Aboxed{\divg \vec {E} & =\frac{1}{\varepsilon _{0}}\rho \left(\vec {r}\right)} \\
	\label{3.5}
	\Aboxed{\rot \vec {E}  & =0}
\end{align}
Die erste Gleichung wird als Gaußsches Gesetz bezeichnet und beschreibt die elektrische Ladung als Quelle des elektrischen Feldes. Die zweite beschreibt die Wirbelfreiheit des elektrostatischen Feldes.

Mit $\vec {D}=\varepsilon _{0}\vec {E}$ im Vakuum kann man das Gaußsche Gesetz auch umformulieren zu
\begin{equation}
	\label{3.6}
	\divg \vec {D}=\rho \left(\vec {r}\right).
\end{equation}
Zu beiden Feldgleichungen gibt es integrale Formulierungen:
\begin{equation*}
	\int _{V}\diff ^{3}\vec r\divg \vec {E}=\int _{\partial V}\vec {E}\cdot \diff \vec {f}=\frac{1}{\varepsilon _{0}}\int \diff ^{3}\vec r\rho \left(\vec {r}\right)\implication \boxed{\int _{\partial V}\vec {E}\cdot \diff \vec f=\frac{1}{\varepsilon _{0}}Q}
\end{equation*}
\begin{formal}
	Der Fluss aus einem Volumen $V$ heraus ist proportional zu der Gesamtladung. Die elektrische Ladungsdichte ist die Quelle des elektrischen Feldes.
\end{formal}
Betrachte als Beispiel eine Punktladung, die das Feld
\begin{align*}
	\vec {E}=\frac{1}{4\pi \varepsilon _{0}}q\frac{\vec {r}-\vec {r}_{0}}{\left| \vec {r}-\vec {r}_{0}\right| ^{3}}=\frac{1}{4\pi \varepsilon _{0}}q\frac{\hat{\vec {R}}}{R^{2}}
\end{align*}
erzeugt:
\begin{equation*}
	\int _{\partial V_{K}}\vec {E}\cdot \diff \vec {f}=\frac{q}{4\pi \varepsilon _{0}}\int _{\partial V_{K}}\frac{\hat{\vec {R}}}{R^{2}}\cdot \hat{\vec {R}}R^{2}\diff \Omega  =\frac{q}{4\pi \varepsilon _{0}}\int _{\partial V_{K}}\diff \Omega  =\frac{q}{\varepsilon _{0}}
\end{equation*}
Für die andere Feldgleichung betrachten wir das Arbeitsintegral, also die von einer Punktladung $q$ verrichtete Arbeit gegen die elektrische Kraft $\vec F_{\mathrm{el}}=q\vec {E}$,
\begin{equation*}
	W=-q\int _{C}\vec {E}\cdot \diff \vec {r}=q\int _{C}\nabla \phi \cdot \diff \vec {r}=q\int _{C}\diff \phi =q\left[\phi \left(2\right)-\phi \left(1\right)\right].
\end{equation*}
Insbesondere gilt
\begin{equation}
	\label{3.7}
	\oint \vec {E}\cdot \diff \vec {r}=0.
\end{equation}

\begin{formal}
	Die Feldlinien des elektrostatischen Feldes sind nicht geschlossen, es gibt keine Zirkulation in der Elektrostatik, $\rot \vec {E}=0$.
\end{formal}



\subsection{Potentialgleichung}

Aus dem Gaußschen Gesetz können wir die folgende Poisson-Gleichung ableiten, die für $\rho =0$ zu einer Laplace-Gleichung wird:
\begin{equation}
	\label{3.8}
	\boxed{\nabla ^{2}\phi =-\frac{1}{\varepsilon _{0}}\rho }
\end{equation}
Zur Lösung einer linearen Differentialgleichung können wir die Methode der Greenschen Funktion verwenden. Dabei drücken wir die Lösung allgemein als Faltung der Ladungsdichte mit einer sogenannten Greenschen Funktion aus,
\begin{equation}
	\label{3.9}
	\phi \left(\vec {r}\right)=\int G\left(\vec {r}-\vec {r}'\right)\rho \left(\vec {r}'\right)\diffa[3]{\vec{r}'}.
\end{equation}
Durch Vergleich mit der Bestimmungsgleichung des elektrischen Potentials können wir die Greensche Funktion für diese Differentialgleichung ablesen:
\begin{equation}
	\label{3.10}
	G\left(\vec {r}-\vec {r}'\right)=\frac{1}{4\pi \varepsilon _{0}}\frac{1}{\left| \vec {r}-\vec {r}'\right| }
\end{equation}
Insbesondere gilt für eine Punktladung $\rho \left(\vec {r}\right)=q\delta \left(\vec {r}-\vec {r}_{0}\right)$, dass $\phi \left(\vec {r}\right)=qG\left(\vec {r}-\vec {r}_{0}\right)$ und damit, dass
\begin{equation*}
	\nabla ^{2}\frac{1}{\left| \vec {r}-\vec {r}_{0}\right| }=-4\pi \delta \left(\vec {r}-\vec {r}_{0}\right).
\end{equation*}



\subsection{Feldlinien}

\begin{figure}[htb]
	\centering
	\tfigDipoleFieldPotential
	\caption{Äquipotentiallinien (gestrichelt) und elektrische Feldlinien (durchgezogen) von zwei ungleichnamigen Ladungen. }
	\label{fig:dipole_field_potential}
\end{figure}

Als Äquipotentiallinien bzw. -flächen werden die Linien/Flächen gleichen Potentials, $\phi =\text{const}$ bezeichnet. Die Feldlinien stehen senkrecht auf den Äquipotentialflächen, weil $\vec {E}\left(\vec {r}\right)=-\nabla \phi $. In \Abbref{fig:dipole_field_potential} sind die Äquipotentiallinien und die Feldlinien einer positiven und einer negativen Ladung dargestellt.



\subsection{Elektrostatische Energie}

Die potentielle Energie einer Ladung $q$ am Ort $\vec {r}$ im Feld $\vec {E}=-\nabla \phi $ ist definiert über einen Referenzpunkt $\phi _{1}=0$, der zum Beispiel im Unendlichen liegt (aber je nach Anwendung auch an anderen Punkten liegen kann):
\begin{equation}
	\label{3.11}
	U\left(\vec {r}\right)=q\phi \left(\vec {r}\right)
\end{equation}
Zum Beispiel ist die potentielle Energie von zwei Punktladungen
\begin{equation*}
	U=q_{1}\phi _{2}=\frac{1}{4\pi \varepsilon _{0}}\frac{q_{1}q_{2}}{\left| \vec {r}_{1}-\vec {r}_{2}\right| }.
\end{equation*}
Die elektrostatische Energie $U$ von $N$ Punktladungen im eigenen Feld kann dann in zwei Schritten bestimmt werden.
\begin{itemize}
	\item Bestimme die Energie von $q_{i}$ im Feld von $q_{j}$ ($j=1,\ldots ,i-1$):
	      \begin{equation*}
		      U_{i}\left(\vec {r}_{i}\right)=q_{i}\sum _{j=1}^{i-1}\phi _{j}=\frac{q_{i}}{4\pi \varepsilon _{0}}\sum _{j=1}^{i-1}\frac{q_{j}}{\left| \vec {r}_{i}-\vec {r}_{j}\right| }
	      \end{equation*}
	\item Bringe $N$ Ladungen sukzessive an ihren Ort:
	      \begin{equation*}
		      U=\sum _{i=2}^{N}U_{i}\left(\vec {r}_{i}\right)=\frac{1}{4\pi \varepsilon _{0}}\sum _{i=2}^{N}\sum _{j=1}^{i-1}\frac{q_{i}q_{j}}{\left| \vec {r}_{i}-\vec {r}_{j}\right| }=\frac{1}{8\pi \varepsilon _{0}}\sum _{i\neq j}^{N}\frac{q_{i}q_{j}}{\left| \vec {r}_{i}-\vec {r}_{j}\right| }
	      \end{equation*}

\end{itemize}
Für eine kontinuierliche Ladungsverteilung ergibt sich
\begin{equation*}
	U=\frac{1}{8\pi \varepsilon _{0}}\int \diffa[3]{\vec{r}}\diffa[3]{\vec{r}'}\frac{\rho \left(\vec {r}\right)\rho \left(\vec {r}'\right)}{\left| \vec {r}-\vec {r}'\right| }=\frac{1}{2}\int \diffa[3]{\vec{r}}\rho \left(\vec {r}\right)\phi \left(\vec {r}\right).
\end{equation*}
Bemerkungen:
\begin{itemize}
	\item Den zusätzlichen Faktor von $1/2$ erhält man, weil $\phi \left(\vec {r}\right)$ von $\rho \left(\vec {r}\right)$ selbst erzeugt wird und die gegenseitige Wirkung von je zwei Ladungen die gleiche ist.

	\item Für beschränkte $\rho $ ist $\vec {r}\rightarrow \vec {r}'$ wohldefiniert, da $\diff ^{3}r=r^{2}\diff r\diff \Omega  $.


\end{itemize}
Es ist auch möglich, die Energie durch das Feld $\vec {E}\left(\vec {r}\right)$ auszudrücken:
\begin{equation*}
	U=\frac{\varepsilon _{0}}{2}\int \diffa[3]{\vec{r}}\phi \nabla ^{2}\phi \overset{\text{partielle Int}.}{=}\frac{\varepsilon _{0}}{2}\int \diffa[3]{\vec{r}}\nabla \phi \cdot \nabla \phi =\frac{\varepsilon _{0}}{2}\int \diffa[3]{\vec{r}}\left| \vec {E}\right| ^{2}
\end{equation*}
Damit lässt sich die Energiedichte in der Elektrostatik folglich schreiben als
\begin{equation*}
	u\left(\vec {r}\right)=\frac{\varepsilon _{0}}{2}\left| \vec {E}\left(\vec {r}\right)\right| ^{2}=\frac{1}{2}\vec {E}\cdot \vec {D}.
\end{equation*}



\subsection{Homogen geladene Kugel}

\begin{figure}[htb]
	\centering
	\tfigEfieldAndPotLinesAndChargeDensitityHomoChargedSphere
	\caption{Links: Äquipotentiallinien und elektrische Feldlinien einer homogen geladenen Kugel mit Radius $R$. Rechts: Die Ladungsdichte ist konstant $\rho_0$ innerhalb ($r<R$) und gleich 0 außerhalb der Kugel. }
	\label{fig:homogenously_charged_ball}
\end{figure}

Auf der homogen geladenen Kugel $V_{r}$ ist die Ladungsdichte $\rho \left(\vec {r}\right)$ innerhalb der Kugel konstant $\rho _{0}$ und außerhalb der Kugel gleich $0$ (siehe \Abbref{fig:homogenously_charged_ball}). Das Problem ist kugelsymmetrisch und hängt nur von der Radialrichtung ab.



\begin{figure}[htb]
	\centering
	\tfigEfieldAndPotentialHomoChargedSphere
	\caption{Links: Elektrisches Feld einer homogen geladenen Kugel. Das Feld steigt im Inneren linear an und fällt im Äußeren mit $r^{-2}$ ab. Rechts: Das Potential fällt im Äußeren genauso ab wie das Potential einer Punktladung. }
	\label{fig:homogenously_charged_ball_field_potential}
\end{figure}

Feld und Potential können zum Beispiel über das Gaußsche Gesetz berechnet werden:
\begin{align*}
	\int _{{V_{r}}}\diff ^{3}r'\frac{\rho \left(r\right)}{\varepsilon _{0}} & =\int _{{V_{r}}}\diff ^{3}r'\divg \vec {E}=\int _{\partial {V_{r}}}\vec {E}\cdot \diff \vec {f}\implication 4\pi r^{2}E\left(r\right)=\frac{1}{\varepsilon _{0}}\int _{0}^{r}\diff r'r'^{2}\rho \left(r'\right) \\
	\Rightarrow E\left(r\right)                                             & =\frac{Q}{4\pi \varepsilon _{0}}
	\begin{cases} \frac{r}{R^{3}}, & r<R     \\
              \frac{1}{r^{2}}, & r\geq R
	\end{cases} \quad\xrightarrow{\text{Integration}}\quad \phi \left(r\right)=\frac{Q}{4\pi \varepsilon _{0}}
	\begin{cases} \frac{3}{2R}-\frac{r^{2}}{2R^{3}}, & r<R     \\
              \frac{1}{r},                       & r\geq R
	\end{cases}
\end{align*}
Beide Größen sind in \Abbref{fig:homogenously_charged_ball_field_potential} dargestellt. Bemerkenswert ist, dass für $r\geq R$ das elektrische Feld $\vec {E}\left(\vec {r}\right)=E\left(r\right)\cdot \vec {e}_{r}$ gerade dem Feld einer Punktladung $Q$ im Mittelpunkt der Kugel entspricht.

Es soll nun die Energiedichte $u\left(\vec {r}\right)$ für die homogen geladene Kugel berechnet werden:
\begin{align*}
	u\left(\vec {r}\right)=\frac{\varepsilon _{0}}{2}\left| \vec {E}\right| ^{2}=\frac{Q^{2}}{32\pi ^{2}\varepsilon _{0}}\begin{cases} \frac{r^{2}}{R^{6}}, & r<R     \\
              \frac{1}{r^{4}},     & r\geq R
	                                                                                                                     \end{cases}
\end{align*}
Daraus ergibt sich die elektrostatische Energie ("`Selbstenergie`` einer homogen geladenen Kugel)
\begin{equation*}
	U=4\pi \int _{0}^{\infty }\diffa{r}u\left(r\right)r^{2}=\frac{1}{4\pi \varepsilon _{0}}\frac{3}{5}\frac{Q^{2}}{R}.
\end{equation*}
Diese Rechnung lässt über die Ruheenergie eines Elektrons eine Abschätzung für den Elektronenradius zu (sogenannter klassischer Elektronenradius):
\begin{equation*}
	U\overset{!}{=} m_{e}c^{2}\approx \SI{0.5}{\mega\eV}\implication R_{e}= \SI{1,7e-15}{\m}
\end{equation*}
Allerdings liegt die Compton-Wellenlänge $\lambda _{e}=h/({m_{e}c})=\SI{2e-12}{\m}$ schon weit über diesem Radius, sodass Quanteneffekte hier nicht vernachlässigbar sind.



\subsection{Extremalprinzip und Kapazitäten}

\begin{formal}
	In der Elektrostatik sind Leiter stets Äquipotentialflächen, d.h. $\phi =\text{const}$ und daher $\vec {E}=-\nabla \phi =\vec {0}$ entlang des Leiters. Sonst würde ein Strom fließen, weil sich die freien Elektronen im Leiter aufgrund des nicht-verschwindenden Feldes bewegen würden.
\end{formal}

\begin{figure}[htb]
	\centering
	\tfigThreeConductors
	\caption{Anordnung von elektrischen Leitern $L_i$ mit Ladungen $Q_i$ und Potentialen $\phi_i$. }
	\label{fig:three_conductors}
\end{figure}

Betrachte den Fall von $n$ Leitern mit Volumen $L_{i}$, je der Ladung $Q_{i}$ und dem Potential $\phi _{i}$, wie schematisch in \Abbref{fig:three_conductors} gezeigt.

Es soll untersucht werden, wie aus der Ladungsverteilung die Potentiale und das elektrische Feld bestimmt werden können.
\begin{formal}
	\textbf{Theorem von Thomson:}

	Die Ladungsdichten $\rho _{i}\left(\vec {r}\right)$ in Leitern $i$ stellen sich so ein, dass die Gesamtenergie minimal wird.
\end{formal}
Für den Beweis wird die Gesamtenergie betrachtet:
\begin{equation*}
	U=\frac{1}{8\pi \varepsilon _{0}}\sum _{ij}\int _{L_{i}}\diffa[3]{\vec {r}_{i}}\int _{L_{j}}\diffa[3]{\vec {r}_{i}}\frac{\rho _{i}\left(\vec {r}_{i}\right)\rho _{j}\left(\vec {r}_{j}\right)}{\left| \vec {r}_{i}-\vec {r}_{j}\right| }
\end{equation*}
Minimierung unter der Nebenbedingung $\int _{{L_{i}}}\diffa[3]{\vec {r}}\rho _{i}\left(\vec {r}\right)=Q_{i}$ führt auf
\begin{equation*}
	\frac{\partial }{\partial \rho _{k}\left(\vec {r}\right)}\left(U-\sum _{i}\phi _{i}\int _{L_{i}}\diffa[3]{\vec {r}_{i}}\rho _{i}\left(\vec {r}\right)\right)=0,
\end{equation*}
wobei in Voraussicht die Lagrange-Parameter als $\phi _{i}$ bezeichnet werden, weil sich mit
\begin{equation*}
	\partial _{{\rho _{k}}\left(\vec {r}\right)}\sum _{i}\int \rho _{i}\left(\vec {r}\right)f_{i}\left(\vec {r}\right)\diff ^{3}r=f_{k}\left(\vec {r}\right)
\end{equation*}
ergibt, dass
\begin{equation*}
	\phi _{k}=\frac{1}{4\pi \varepsilon _{0}}\sum _{j}\int _{L_{j}}\diffa[3]{\vec {r}_{j}}\frac{\rho _{j}\left(\vec {r}_{j}\right)}{\left| \vec {r}_{k}-\vec {r}_{j}\right| },\quad \vec {r}_{k}\in L_{k}
\end{equation*}
was gerade der Bestimmungsgleichung für das Potential $\phi _{k}$ als Potential von $L_{k}$ entspricht. Da das Vorgehen der Minimierung der Gesamtenergie auf das richtige Potential führt, ist das Theorem bestätigt.


Die Potentiale $\phi _{i}$ lassen sich linear über die Ladungen $Q_{i}$ zerlegen,
\begin{equation*}
	\phi _{i}=\sum _{j}p_{ij}Q_{j},
\end{equation*}
weil einerseits gilt, dass $\nabla ^{2}\phi =-\rho /\varepsilon _{0}$ und andererseits $\phi $ linear in $\rho $ ist. Dieser Zusammenhang lässt sich invertieren,
\begin{equation*}
	Q_{i}=\sum _{j}C_{ij}\phi _{j},
\end{equation*}
wobei dann die Vorfaktoren $C_{ij}$ als Kapazitäten mit der Einheit $\left[C_{ij}\right]=\SI{1}{\coulomb\per\V}=\SI{1}{\farad}$ definiert werden. Aus dem Ausdruck für die elektrostatische Energie
\begin{equation*}
	U=\frac{1}{2}\sum _{i}\underset{Q_{i}=\sum _{j}C_{ij}\phi _{j}}{\underbrace{\int _{L_{i}}\diffa[3]{\vec {r}_{i}}\rho _{i}\left(\vec {r}_{i}\right)\phi _{i}}}=\frac{1}{2}\sum _{ij}\phi _{i}C_{ij}\phi _{j}
\end{equation*}
folgt die Symmetrie $C_{ij}=C_{ji}$.

So gilt zum Beispiel für einen Plattenkondensator allgemein
\begin{equation*}
	C=\frac{Q}{V},\quad U=\frac{1}{2}CV^{2}=\frac{1}{2}QV,
\end{equation*}
für einen Plattenkondensator mit parallelen Platten der Fläche $A$ und Abstand $d$
\begin{equation*}
	C=\varepsilon _{0}\varepsilon _{r}\frac{A}{d},
\end{equation*}
für einen Zylinderkondensator der Länge $L$ und mit Radien $r_{1}<r_{2}$
\begin{equation*}
	C=2\pi \varepsilon _{0}\varepsilon _{r}\frac{L}{\ln \frac{r_{2}}{r_{1}}}
\end{equation*}
und schließlich für einen Kugelkondensator mit Radien $r_{1}<r_{2}$
\begin{equation*}
	C=4\pi \varepsilon _{0}\varepsilon _{r}\left(\frac{1}{r_{1}}-\frac{1}{r_{2}}\right)^{-1}=4\pi \varepsilon _{0}\varepsilon _{r}\frac{r_{1}r_{2}}{d}.
\end{equation*}
\Abbref{fig:capacitors} bildet diese drei einfachen Kondensatorgeometrien ab.


\begin{figure}[htb]
	\centering
	\includegraphics{capacitors.pdf}
	\caption{Schematische Darstellung eines Plattenkondensators mit parallelen, ebenen Platten (links), eines Zylinderkondensators (Mitte) und eines Kugelkondensators (rechts). Auf den Platten ist das Potential jeweils $\phi_1$ und $\phi_2$. }
	\label{fig:capacitors}
\end{figure}



\subsection{Maxwellscher Spannungstensor\label{sec:maxwellscher_spannungstensor}}

\section{Randbedingungen des elektrischen Feldes auf Grenzflächen\label{sec:randbedingungen_auf_grenzflaechen}}

 {\ldots}



\section{Randwertprobleme der Elektrostatik}

Meist sind bei der Lösung von elektrostatischen Problemen der Poisson-Gleichung
\begin{equation*}
	\nabla ^{2}\phi =-\frac{\rho }{\varepsilon _{0}}
\end{equation*}
in einem Volumen $V$ noch die Randbedingungen auf $\partial V$ zu berücksichtigen.



\subsection{Eindeutigkeit der Lösung}

Allgemein sind drei verschiedene Arten von Randbedingungen möglich.

\begin{enumerate}
	\item Dirichlet-Randbedingung: Das Potential ist auf dem Rand vorgegeben, $\left.\hspace{0pt}\phi \right| _{\partial V}$.

	\item Neumann-Bedingung: Die Normalenableitung der Lösung wird auf dem Rand vorgegeben,
	      \begin{align*}
		      \vec {n}\cdot \left.\nabla \phi \right| _{\partial V}=\left.\frac{\partial \phi }{\partial n}\right| _{\partial V}.
	      \end{align*}


	\item Cauchy-Bedingung: $a\left(1\right)+b\left(2\right)$ ist vorgegeben.
\end{enumerate}

Zum Beispiel kommt die Dirichlet-Randbedingung bei Oberflächen von Leitern vor, von denen wir ja bereits wissen, dass dort das Potential konstant gleich $0$ ist.
\begin{formal}
	Für Dirichlet- und Neumann-Randbedingungen ist die Lösung der Poisson-Gleichung eindeutig.
\end{formal}
Der Beweis ist einfach, denn seien $\phi _{1}$ und $\phi _{2}$ zwei unterschiedliche Lösungen, dann erfüllt $\phi _{d}=\phi _{1}-\phi _{2}$ die Gleichung $\nabla ^{2}\phi _{2}=0$ mit der Randbedingung
\begin{align*}
	\begin{cases} \left.\phi _{d}\right| _{\partial V}                             & =0 \\
              \left.\frac{\partial \phi _{d}}{\partial n}\right| _{\partial V} & =0
	\end{cases} ,
\end{align*}
da $\phi _{1,2}$ die gleichen Randbedingungen erfüllen. Mit der zweiten Greenschen Identität folgt
\begin{equation*}
	\int _{V}\left(\varphi \nabla ^{2}\psi +\nabla \varphi \cdot \nabla \psi \right)\diff V=\int _{\partial V}\varphi \nabla \psi \cdot \diff \vec {f}.
\end{equation*}
Setze nun $\varphi =\psi =\phi _{d}$,
\begin{equation*}
	\int _{V}\left(\nabla \phi _{d}\right)^{2}\diff V=0.
\end{equation*}
Da nun aber der Integrand stets positiv ist, folgt $\nabla \phi _{d}=0$ und also ohne Beschränkung der Allgemeinheit $\phi _{d}=\text{const}=0$.


\subsection{Methode der Greenschen Funktion}

\subsection{Aussagen zur Potentialtheorie}

\subsection{Lösungen zur Laplace-Gleichung in Kugelkoordinaten}

Die Laplace-Gleichung ist eine zentrale Gleichung in der Physik. In der Elektrostatik gilt sie zum Beispiel im ladungsfreien Raum, aber sie spielt auch für viele andere Modelle eine große Rolle. In kartesischen Koordinaten nimmt die Gleichung die Form
\begin{equation*}
	\nabla ^{2}\phi =\left(\frac{\partial ^{2}}{\partial x^{2}}+\frac{\partial ^{2}}{\partial y^{2}}+\frac{\partial ^{2}}{\partial z^{2}}\right)\phi =0
\end{equation*}
an. Die Lösung lässt sich in Eigenfunktionen des Laplace-Operators zerlegen. Diese sind zum Beispiel für den kartesischen Fall ebene Wellen.

Für kugelsymmetrische Problem bietet es sich an in Kugelkoordinaten zu rechnen. In Kugelkoordinaten lässt sich der Laplace-Operator in Radial- und Winkelanteil zerlegen:
\begin{align*}
	\nabla ^{2}\phi & =\nabla _{r}^{2}\phi +\frac{1}{r^{2}}\nabla _{\varphi ,\vartheta }^{2}\phi \\&=\frac{1}{r^{2}}\frac{\partial }{\partial r}r^{2}\frac{\partial }{\partial r}\phi +\frac{1}{r^{2}\sin \vartheta }\frac{\partial }{\partial \vartheta }\sin \varphi \frac{\partial }{\partial \vartheta }\phi -\frac{1}{r^{2}\sin ^{2} \vartheta }\frac{\partial ^{2}}{\partial \varphi ^{2}}\phi
\end{align*}
Zur Lösung wird ein Produktansatz gemacht,
\begin{equation*}
	\phi \left(r,\varphi ,\vartheta \right)=R\left(r\right)Y\left(\varphi ,\vartheta \right).
\end{equation*}
Eingesetzt in die Laplace-Gleichung ergibt sich
\begin{equation*}
	Y\nabla _{r}^{2}R+\frac{R}{r^{2}}\nabla _{\varphi ,\vartheta }^{2}Y=0 \equivalence\frac{r^{2}}{R}\nabla _{r}^{2}R=-\frac{1}{Y}\nabla _{\varphi ,\vartheta }^{2}Y=\text{const}
\end{equation*}
und hieraus erhält man separat die radialen Eigenfunktionen
\begin{equation*}
	R\left(r\right)=\alpha r^{l}+\beta r^{-\left(l+1\right)}
\end{equation*}
und die bereits aus der Quantenmechanik bekannten Kugelflächenfunktionen $Y_{lm}\left(\varphi ,\vartheta \right)$ für den Winkelanteil nach der Eigenwertgleichung
\begin{equation*}
	\nabla _{\varphi ,\vartheta }^{2}Y_{lm}=-l\left(l+1\right)Y_{lm}.
\end{equation*}
Die Gesamtlösung setzt sich dann zusammen aus dem Radial- und Winkelanteil:
\begin{equation*}
	\phi \left(r,\varphi ,\vartheta \right)=\sum _{l=0}^{\infty }\sum _{m=-l}^{l}\underset{\text{Radialanteil}}{\underbrace{\left(\alpha _{lm}r^{l}+\beta _{lm}r^{-\left(l+1\right)}\right)}}\underset{\text{Winkelanteil}}{\underbrace{Y_{lm}\left(\varphi ,\vartheta \right)}}
\end{equation*}
Für zylindersymmetrische Probleme ist die $\varphi $-Abhängigkeit aufgehoben und es brauchen nur Funktionen mit $m=0$ betrachtet zu werden.

Für die Kugelflächenfunktionen von zwei Vektoren $\vec {r}_{1}=\left(r_{1},\varphi _{1},\vartheta _{1}\right)$ und $\vec {r}_{2}=\left(r_{2},\varphi _{2},\vartheta _{2}\right)$ gilt das folgende Additionstheorem:
\begin{equation*}
	\sum _{m=-l}^{l}Y_{lm}\left(\varphi _{1},\vartheta _{1}\right)Y_{lm}^{*}\left(\varphi _{2},\vartheta _{2}\right)=\frac{2l+1}{4\pi }P_{l}\left(\cos \angle \left(\vec {r}_{1},\vec {r}_{2}\right)\right)
\end{equation*}
Die Greensche Funktion kann mit diesem Additionstheorem nach den Kugelflächenfunktionen entwickelt werden,
\begin{equation*}
	\frac{1}{\left| \vec {r}-\vec {r}'\right| }=\frac{1}{\sqrt{r^{2}+r'^{2}-2rr'\cos \vartheta }}=\sum _{l}\frac{r_{<}^{l}}{r_{>}^{l+1}}P_{l}\left(\cos \vartheta \right),\quad
	r_{>}=\max \left(r,r'\right) ,\quad
	r_{<}=\min \left(r,r'\right).
\end{equation*}


\begin{figure}[htb]
	\centering
	\tfigComplexProblemsConformalMap
	\caption{Für Probleme mit komplexen Geometrien kann mithilfe einer konformen Abbildung $z$ die Lösung aus der Lösung für den kugelsymmetrischen Fall abgeleitet werden. }
	\label{fig:complex_problems_conformal_map}
\end{figure}

Obwohl die Wahl der Kugelkoordinaten nur für wenige Probleme sinnvoll ist, kann die Lösung für das kugelförmige Problem durch konforme Abbildungen auf komplexere Geometrien angewandt werden (\Abbref{fig:complex_problems_conformal_map}).

\section{Multipolentwicklung}

Bei der Multipolentwicklung klassifiziert man bestimmte Ladungsverteilungen nach sogenannten Momenten (Dipolmoment, Quadrupolmoment, {\ldots}). Zum Beispiel beschreibt das Dipolmoment zwei räumlich voneinander getrennte Ladungen unterschiedlichen Vorzeichens. Auch ein nach außen insgesamt elektrisch neutraler Körper kann ein Dipolmoment aufweisen, nämlich wenn die Schwerpunkte von der positiven und negativen Ladung nicht zusammenfallen. Ein prominentes mikroskopisches Beispiel ist das Wassermolekül (\Abbref{fig:water_molecule}), bei dem das Sauerstoffatom eine bedeutend größere Elektronegativität besitzt als die Wasserstoffatome und dadurch eine Ladungsverschiebung der gebundenen Elektronen zum Sauerstoffatom hin bewirkt. Dadurch besitzt dieses lokal eine Ladung von $\SI{-0.8}{\eV}$, während die Wasserstoffatome eine Ladung von je $\SI{+0.4}{\eV}$ tragen.

Das Dipolmoment $\vec {p}$ ist ein Vektor und per Definition von der negativen Ladung zur positiven gerichtet. Die Einheit des Dipolmoments ist $\SI{1}{Debye}=\SI{3,34e-30}{\coulomb\m}$.



\begin{figure}[htb]
	\centering
	\tfigWatermolecule
	\caption{Das Wassermolekül ist ein Dipol, bei dem das Sauerstoffatom eine negative Partialladung trägt, während die Wasserstoffatome aufgrund ihrer geringeren Elektronegativität entsprechend positiv geladen sind. }
	\label{fig:water_molecule}
\end{figure}

Für die Multipolentwicklung wird das Potential
\begin{equation*}
	\phi \left(\vec {r}\right)=\frac{1}{4\pi \varepsilon _{0}}\int \diffa[3]{\vec{r}'}\frac{\rho \left(\vec {r}\right)}{\left| \vec {r}-\vec {r}'\right| }
\end{equation*}
nach den sogenannten Momenten der Ladungsverteilung entwickelt. Dazu wird zunächst eine Taylorentwicklung für den Ausdruck $1/\left| \vec {r}-\vec {r}'\right| $ um $\vec {r}$ durchgeführt ($\left| \vec {r}\right| =r$, Einsteinsche Summenkonvention):
\begin{equation*}
	\frac{1}{\left| \vec {r}-\vec {r}'\right| }=\frac{1}{r}-x_{i}'\partial _{i}\frac{1}{r}+\frac{1}{2}x_{i}'x_{j}'\partial _{i}\partial _{j}\frac{1}{r}+\ldots +\frac{\left(-1\right)^{n}}{n!}x_{i_{1}}'\ldots x_{i_{n}}'\partial _{{i_{1}}}\ldots \partial _{{i_{n}}}\frac{1}{r}
\end{equation*}
Damit lässt sich das Potential nähern als
\begin{align*}
	\phi \left(\vec {r}\right) & \approx \frac{1}{4\pi \varepsilon _{0}}\left(\int \diffa[3]{\vec{r}'}\frac{\rho \left(\vec {r}\right)}{r}-\int \diffa[3]{\vec{r}'}\rho \left(\vec {r}\right)x_{i}'\partial _{i}\frac{1}{r}+\frac{1}{2}\int \diffa[3]{\vec{r}'}\rho \left(\vec {r}\right)x_{i}'x_{j}'\partial _{i}\partial _{j}\frac{1}{r}+\ldots \right.                                                                                                                      \\
	                           & \quad\quad\left. +\frac{\left(-1\right)^{n}}{n!}\int \diffa[3]{\vec{r}'}\rho \left(\vec {r}\right)x_{i_{1}}'\ldots  x_{i_{n}}'\partial _{{i_{1}}}\ldots \partial _{{i_{n}}}\frac{1}{r}\right)                                                                                                                                                                                                                                                 \\
	                           & =\frac{1}{4\pi \varepsilon _{0}}\left(\frac{1}{r}\underset{q}{\underbrace{\int \diffa[3]{\vec{r}'}\rho \left(\vec {r}\right)}}-\partial _{i}\frac{1}{r}\underset{p_{i}}{\underbrace{\int \diffa[3]{\vec{r}'}\rho \left(\vec {r}\right)x_{i}'}}+\frac{1}{6}\partial _{i}\partial _{j}\frac{1}{r}\underset{Q_{ij}}{\underbrace{\int \diffa[3]{\vec{r}'}\rho \left(\vec {r}\right)\left(3x_{i}'x_{j}'-r^{'2}\delta _{ij}\right)}}+\ldots \right. \\
	                           & \quad\quad\left.  +\,\partial _{{i_{1}}}\ldots \partial _{{i_{n}}}\frac{1}{r}\underset{M_{{i_{1}}\ldots {i_{n}}}}{\underbrace{\frac{\left(-1\right)^{n}}{n!}\int \diffa[3]{\vec{r}'}\rho \left(\vec {r}\right)x_{i_{1}}'\ldots x_{i_{n}}'}}\right)                                                                                                                                                                                            \\
	                           & =\frac{1}{4\pi \varepsilon _{0}}\left(\frac{q}{r}-p_{i}\partial _{i}\frac{1}{r}+\frac{1}{6}Q_{ij}\partial _{i}\partial _{j}\frac{1}{r}+\ldots +M_{{i_{1}}\ldots {i_{n}}}\partial _{{i_{1}}}\ldots \partial _{{i_{n}}}\frac{1}{r}\right).
\end{align*}
Dabei identifizieren wir das Dipolmoment als Tensor erster Stufe
\begin{equation*}
	p_{i}=\int \diffa[3]{\vec{r}'}\rho \left(\vec {r}\right)x_{i}',
\end{equation*}
das Quadrupolmoment als Tensor zweiter Stufe
\begin{equation*}
	Q_{ij}=\int \diffa[3]{\vec{r}'}\left(3x_{i}'x_{j}'-{r'}^{2}\delta _{ij}\right)\rho \left(\vec {r}\right),
\end{equation*}
bei dem standardmäßig noch der Term $-{r'}^{2}\delta _{ij}$ hinzugefügt wird, welcher aber nicht zu $\phi(\vec r) $ beiträgt, weil
\begin{equation*}
	\delta _{ij}\partial _{i}\partial _{j}\frac{1}{r}=\sum _{i}\partial _{i}^{2}\frac{1}{r}=\nabla ^{2}\frac{1}{r}=0
\end{equation*}
und schließlich das $n$-te Multipolmoment
\begin{equation*}
	M_{{i_{1}}\ldots {i_{n}}}\propto \int \diffa[3]{\vec{r}'}\rho \left(\vec {r}\right)x_{i_{1}}'\ldots x_{i_{n}}'.
\end{equation*}
Mit den Identitäten
\begin{equation*}
	\partial _{i}\frac{1}{r}=-\frac{1}{r^{2}}\partial _{i}r=-\frac{x_{i}}{r^{3}},\quad \partial _{i}\partial _{j}\frac{1}{r}=\frac{3x_{i}x_{j}}{r^{5}}-\frac{\delta _{ij}}{r^{3}},
\end{equation*}
(wobei der Term $\delta_{ij}/r^3$ irrelevant ist wegen $\delta _{ij}Q_{ij}=Q_{ii}=0$) erhält man dann für das Potential
\begin{equation*}
	\phi \left(\vec {r}\right)=\frac{1}{4\pi \varepsilon _{0}}\left(\frac{q}{r}+\frac{\vec {p}\cdot \vec {r}}{r^{3}}+\frac{1}{2}Q_{ij}\frac{x_{i}x_{j}}{r^{5}}+\ldots \right).
\end{equation*}



\subsection{Diskussion der Multipolmomente}

\begin{enumerate}
	\item Monopol (Potential/Feld einer Punktladung $\rho _{m}=q\delta \left(\vec {r}\right)$):
	      \begin{equation*}
		      \phi _{\mathrm{m}}\left(\vec {r}\right)=\frac{1}{4\pi \varepsilon _{0}}\frac{q}{r}\rightarrow \vec {E}(\vec r)=\frac{q}{4\pi \varepsilon _{0}}\frac{\vec {r}}{r^{3}}
	      \end{equation*}
	\item Dipol:
	      \begin{align*}
		      \phi _{\diff } & =-\frac{1}{4\pi \varepsilon _{0}}\vec {p}\cdot \nabla \frac{1}{r}=\frac{1}{4\pi \varepsilon _{0}}\frac{\vec {p}\cdot \vec {r}}{r^{3}}\propto \frac{1}{r^{2}}                                        \\
		      E_{i}          & =\frac{1}{4\pi \varepsilon _{0}}p_{j}\nabla _{i}\nabla _{j}\frac{1}{r}=\frac{1}{4\pi \varepsilon _{0}}\frac{p_{j}}{r^{3}}\left(\frac{3x_{i}x_{j}}{r^{2}}-\delta _{ij}\right)\propto \frac{1}{r^{3}}
	      \end{align*}
	      Wir sehen, dass das Feld eines Dipols mit $r^{-3}$ abnimmt, während dasjenige eines Monopols nur mit $r^{-2}$ abfällt. Die Felder der einzelnen Ladungen heben sich im Fernfeld zum Teil auf.

	      Die Ladungsdichte eines elementaren Dipols ist
	      \begin{equation*}
		      \rho _{\diff }\left(\vec {r}\right)=q\left(\delta \left(\vec {r}-\frac{\vec {d}}{2}\right)-\delta \left(\vec {r}+\frac{\vec {d}}{2}\right)\right),
	      \end{equation*}
	      woraus sich ein Dipolmoment von
	      \begin{equation*}
		      \vec {p}=q\vec {d}\parallel \vec {d}
	      \end{equation*}
	      ergibt.

	      Man kann auch einen sogenannten Punktdipol betrachten \textendash{} ein idealisiertes Objekt, bei dem der Abstand $\vec {d}$ gegen $0$ geht:
	      \begin{equation*}
		      \vec {p}=\lim _{\substack{
				      d\rightarrow 0 \\
				      qd<\infty}} q\vec {d},\quad \rho _{\diff }\left(\vec {r}\right)=-\vec {p}\cdot \nabla \delta \left(\vec {r}\right), \quad\phi _{\diff }=-\frac{1}{4\pi \varepsilon _{0}}\vec {p}\cdot \nabla \frac{1}{r}
	      \end{equation*}
	\item Quadrupol:
	      \begin{align*}
		      \phi _{\mathrm{Q}} & =\frac{1}{4\pi \varepsilon _{0}}\frac{1}{6}Q_{kl}\nabla _{k}\nabla _{l}\frac{1}{r}=\frac{1}{4\pi \varepsilon _{0}}\frac{1}{2}Q_{kl}\frac{x_{k}x_{l}}{r^{5}}\propto \frac{1}{r^{3}}                                                                   \\
		      E_{i}              & =-\frac{1}{4\pi \varepsilon _{0}}\frac{1}{6}Q_{kl}\nabla _{i}\nabla _{k}\nabla _{l}\frac{1}{r}=\frac{1}{4\pi \varepsilon _{0}}\frac{Q_{kl}}{2}\frac{5x_{i}x_{k}x_{l}-r^{2}\left(\delta _{kl}x_{i}+\delta _{il}x_{k}+\delta _{ik}x_{l}\right)}{r^{7}}
	      \end{align*}


	      \begin{figure}[htb]
		      \centering
		      \tfigElementalQuadrupoles
		      \caption{Elementare Quadrupole können in zwei verschiedenen Konfigurationen auftreten. }
		      \label{fig:elemental_quadrupoles}
	      \end{figure}

	      Es gibt zwei elementare Quadrupole (mit $\vec {p}=0$), die in \Abbref{fig:elemental_quadrupoles} zu sehen sind.

	      Wird der Bezugspunkt/Aufpunkt verschoben, so ändern sich im Allgemeinen die Multipolmomente, aber das erste Moment, das bei der Multipolentwicklung einer Ladungsverteilung ungleich $0$ ist, bleibt unverändert.


	      \begin{formal}
		      Das niedrigste, nicht-verschwindende Multipolmoment in der Entwicklung ist unabhängig vom Bezugspunkt.
	      \end{formal}
	      Man kann auch sphärische Multipolmomente mithilfe von Kugelflächenfunktionen ausdrücken.
\end{enumerate}


\subsection{Energie von Multipolen im äußeren Feld}

Die Energie von Multipolen in einem externen Potential $\phi _{e}\left(\vec {r}\right)$ kann aus der bereits bekannten Formel für die Energie einer beliebigen Ladungsverteilung $\rho \left(\vec {r}\right)$ abgeleitet werden,
\begin{equation*}
	U=\int _{V}\diffa[3]{\vec{r}}\rho \left(\vec {r}\right)\phi _{e}\left(\vec {r}\right).
\end{equation*}
Wir nehmen an, dass die Änderung von $\phi _{e}$ in $V$ nur klein ist und erhalten durch eine Taylor-Entwicklung
\begin{align*}
	U & =\int \diffa[3]{\vec{r}}\rho \left(\vec {r}\right)\left[\phi _{e}\left(0\right)+\vec {r}\nabla \phi _{e}\left(0\right)+\frac{1}{2}x_{i}x_{j}\nabla _{i}\nabla _{j}\phi _{e}\left(0\right)+\ldots \right] \\&=q\phi \left(0\right)-\vec {p}\cdot E_{e}\left(0\right)-\frac{1}{6}Q_{ij}\nabla _{j}E_{e}^{\left(i\right)}\left(0\right)+\ldots
\end{align*}
Die Energie der Multipole ist also durch die $n$-fache Ableitung des Potentials $\nabla ^{n}\phi $ bestimmt. Diese Rechnung erlaubt wegen der Taylor-Entwicklung eine beliebige Wahl des Bezugspunkts.

\begin{figure}[htb]
	\centering
	\includegraphics{dipoles.pdf}
	\caption{Links: Schematische Darstellung zweier Dipole $\vec p_1$, $\vec p_2$ mit Abstand $\vec r$. Weitere Abbildungen: Spezielle Anordnungen zweier Dipole, für die die Energie extremal wird. }
	\label{fig:dipoles}
\end{figure}

Als typisches Beispiel soll die Wechselwirkung zweier Dipole betrachtet werden. Die potentielle Energie kann berechnet werden, indem der Dipol $\vec {p}_{2}$ wie oben beschrieben in das Feld des Dipols $\vec {p}_{1}$ gesetzt wird (oder umgekehrt):
\begin{equation*}
	U_{\mathrm{DD}}=-\vec {p}_{2}\cdot \vec {E}_{1}\left(\vec {r}\right)=-\frac{1}{4\pi \varepsilon _{0}}\frac{1}{r^{3}}\left(\frac{3\left(\vec {r}\cdot \vec {p}_{1}\right)\left(\vec {r}\cdot \vec {p}_{2}\right)}{r^{2}}-\vec {p}_{1}\cdot \vec {p}_{2}\right)
\end{equation*}
Diese wird minimal für $\vec {p}_{1}\parallel \vec {p}_{2}\parallel \vec {r}$ und maximal für $\vec {p}_{1}\parallel \vec {p}_{2}\perp \vec {r}$. Diese Konfigurationen sind in \Abbref{fig:dipoles} zusammengestellt. Für antiparallele $\vec {p}_{1}$ und $\vec {p}_{2}$ und $\vec {p}_{1},\vec {p}_{2}\perp \vec {r}$ wird außerdem ein lokales Minimum erreicht. Aus diesem Grund bilden Dipolmoleküle auch häufig Molekülketten.

Zuletzt sollen noch Drehmomente auf Multipole diskutiert werden. Es ist
\begin{align*}
	\vec {M}           & =\int \diffa[3]{\vec{r}}\vec {r}\times \underset{\text{Kraftdichte}}{\underbrace{\rho \left(\vec {r}\right)\vec {E}_{e}\left(\vec {r}\right)}}\implication M_{i}=\int \diffa[3]{\vec{r}}\varepsilon _{ijk}x_{j}\rho \left(\vec {r}\right)\underset{\approx E_{e}^{\left(k\right)}\left(0\right)+x_{l}\nabla _{l}E_{e}^{\left(k\right)}\left(0\right)}{\underbrace{E_{e}^{\left(k\right)}\left(\vec {r}\right)}} \\
	\implication M_{i} & =\left(\vec {p}\times \vec {E}_{e}\right)_{i}+\frac{1}{3}\varepsilon _{ijk}Q_{jl}\nabla _{l}E_{e}^{\left(k\right)}
\end{align*}
Insbesondere dreht das Drehmoment $\vec {M}$ einen Dipol parallel zu $\vec {E}$, da $\vec {M}=0$ für $\vec {p}\parallel \vec {E}_{e}$ und
\begin{equation*}
	U=-\vec {p}\cdot \vec {E}_{e}=-pE_{e}\cos \vartheta \implication M=-\frac{\partial U}{\partial \vartheta }=-pE_{e}\sin \vartheta =-\left| \vec {p}\times \vec {E}_{e}\right| .
\end{equation*}
\chapter{Elektrische Felder in Materie}

In diesem Kapitel werden die makroskopischen Gleichungen der Elektrostatik in Materie beschrieben und erläutert.

\section{Mikroskopische Gleichungen der Elektrostatik und Mittelung}

Bis jetzt haben wir nur freie Ladungen betrachtet. Die Ladungsdichte $\rho \left(\vec {r}\right)$ erzeugt ein elektrisches Feld $\vec {E}(\vec r)$. In Materie sind zusätzlich auch gebundene Ladungen vorhanden, die mit dem Feld wechselwirken. Das können (nach außen hin elektrisch neutrale) Atome, geladenen Ionen, permanente Dipole (oder Multipole) sein (z.B. $\mathrm{H}_{2}\mathrm{O}$), sowie Dipole sein, die durch ein äußeres elektrisches Feld induziert werden.

Um diese Wechselwirkung zu beschreiben, wird eine Mittelung der mikroskopischen Gleichungen
\begin{equation*}
	\divg \vec {e}=\frac{1}{\varepsilon _{0}}\rho \left(\vec {r}\right),\quad \rot \vec {e}=0
\end{equation*}
durchgeführt. Wir haben bisher einzelne Ladungen durch $\delta $-Funktionen in der Ladungsdichte beschrieben. Dadurch kommt es zu starken räumlichen Ladungsschwankungen. Für eine makroskopische Betrachtung in Größenordnungen von Nanometern wenden wir eine räumliche Mittelung bzw. Glättungsfunktion auf die Ladungsdichteverteilung an.


\subsection{Glättungsfunktion}

Um eine stark variierende Funktion $F\left(\vec {r},t\right)$ zu mitteln, wird sie mit einer sogenannten Glättungsfunktion $f$ gefaltet. Dabei kann es sich z.B. um eine Gauß-Funktion handeln:
\begin{equation*}
	F\left(\vec {r},t\right) \xrightarrow{\text{Mittelung}} \left\langle F\left(\vec {r},t\right)\right\rangle =\int f\left(\left| \vec {r}-\vec {r}'\right| \right)F\left(\vec {r}',t\right)\diff ^{3}\vec {r}'
\end{equation*}
Für eine Punktladung $F\left(\vec {r}\right)=F_{0}\delta \left(\vec {r}-\vec {r}_{0}\right)$ ist dann zum Beispiel $\left\langle F\right\rangle =F_{0}f\left(\vec {r}-\vec {r}_{0}\right)$.

Für die Mittelung gelten die folgenden Eigenschaften:\begin{enumerate}
	\item Die Mittelung der konstanten Funktion $F=1$ ist genau dann konstant $1$, wenn die Glättungsfunktion über den gesamten Raum auf $1$ normiert ist,
		\begin{equation*}
			\left\langle 1\right\rangle =1\equivalence \int \diff ^{3}\vec {r}f=1.
		\end{equation*}
	\item $\partial _{i}\left\langle F\right\rangle =\left\langle \partial _{i}F\right\rangle $.
\end{enumerate}

\section{Makroskopische Gleichungen der Elektrostatik}

Mithilfe der Glättung kann man das makroskopische $\vec {E}$-Feld als
\begin{equation*}
	\vec {E}\left(\vec {r},t\right)=\left\langle \vec {e}\left(\vec {r},t\right)\right\rangle
\end{equation*}
schreiben. Für die Ladungsdichte erhält man
\begin{equation*}
	\left\langle \rho \left(\vec {r}\right)\right\rangle =\left\langle \rho _{f}\left(\vec {r}\right)+\rho _{b}\left(\vec {r}\right)\right\rangle =\left\langle \rho _{f}\left(\vec {r}\right)\right\rangle +\left\langle \rho _{b}\left(\vec {r}\right)\right\rangle \equiv \rho _{F}+\rho _{B}.
\end{equation*}
Die gebundenen Ladungen werden als Summe der Ladungsdichten einzelner Moleküle geschrieben:
\begin{equation*}
	\rho _{b}\left(\vec {r}\right)=\sum _{n}\rho _{n}\left(\vec {r}\right), \rho _{n}\left(\vec {r}\right)=\sum _{i}q_{i}\delta \left(\vec {r}-\vec {r}_{i}\right)=\sum _{i}q_{i}\delta \left(\vec {r}-\left(\vec {r}_{n}+\vec {r}_{ni}\right)\right)
\end{equation*}
mit neuen Bezugspunkten $\vec {r}_{n}$ für die einzelnen Moleküle. Für die Mittelung wird dann eine Taylor-Entwicklung um diese neuen Bezugspunkte $\vec {r}_{n}$ durchgeführt:
\begin{align*}
	\left\langle \rho _{n}\left(\vec {r}\right)\right\rangle &=\sum _{i}q_{i}f\left(\vec {r}-\left(\vec {r}_{n}+\vec {r}_{ni}\right)\right)\\&=\sum _{i}q_{i}\left[f\left(\vec {r}-\vec {r}_{n}\right)-\vec {r}_{ni}\cdot \nabla f\left(\vec {r}-\vec {r}_{n}\right)+\frac{1}{2}\left(\vec {r}_{ni}\right)_{k}\left(\vec {r}_{ni}\right)_{l}\nabla _{k}\nabla _{l}f\left(\vec {r}-\vec {r}_{n}\right)+\ldots \right]
\end{align*}
Daraus können die molekularen Dipolmomente bestimmt werden:
\begin{align*}
		q_{n}                            & =\sum _{i}q_{i} &\text{(Molekulare Ladung)}     \\
		\vec {p}_{n}                     & =\sum _{i}q_{i}\vec {r}_{ni} &\text{(Molekulares Dipolmoment)}    \\
		\left(\mathrm{Q}_{n}\right)_{kl} & =3\sum _{i}q_{i}\left(\vec {r}_{ni}\right)_{k}\left(\vec {r}_{ni}\right)_{l} &\text{(Molekulares Quadrupolmoment)}
\end{align*}
(vgl. Multipolmomente einer kontinuierlichen Ladungsverteilung, aber hier jetzt diskret). Insgesamt ergibt sich eine Verschmierung punktförmiger molekularer Multipole:
\begin{align*}
	\left\langle \rho _{n}\left(\vec {r}\right)\right\rangle &=q_{n}f\left(\vec {r}-\vec {r}_{n}\right)-\vec {p}_{n}\cdot \nabla f\left(\vec {r}-\vec {r}_{n}\right)+\frac{1}{6}\left(\mathrm{Q}_{n}\right)_{kl}\nabla _{k}\nabla _{l}f\left(\vec {r}-\vec {r}_{n}\right) \\&=\left\langle q_{n}\delta \left(\vec {r}-\vec {r}_{n}\right)\right\rangle -\nabla \cdot \left\langle p_{n}\delta \left(\vec {r}-\vec {r}_{n}\right)\right\rangle +\frac{1}{6}\nabla _{k}\nabla _{l}\left\langle \left(\mathrm{Q}_{n}\right)_{kl}\delta \left(\vec {r}-\vec {r}_{n}\right)\right\rangle
\end{align*}
und für die gemittelte gebundene Ladungsdichte:
\begin{equation*}
	\left\langle \rho _{b}\left(\vec {r}\right)\right\rangle =\rho _{\mathrm{m}}\left(\vec {r}\right)-\nabla \cdot \vec {P}\left(\vec {r}\right)+\nabla _{k}\nabla _{l}{Q}_{\mathrm{kl}}+\ldots
\end{equation*}
mit der makroskopischen Ladungsdichte (Monopoldichte)
\begin{equation*}
	\rho _{\mathrm{m}}\left(\vec {r}\right)=\left\langle \sum _{n}q_{n}\delta \left(\vec {r}-\vec {r}_{n}\right)\right\rangle ,
\end{equation*}
der Polarisation (Dipolmomentdichte)
\begin{equation*}
	\vec {P}\left(\vec {r}\right)=\left\langle \sum _{n}\vec {p}_{n}\delta \left(\vec {r}-\vec {r}_{n}\right)\right\rangle
\end{equation*}
und so weiter.

% !TEX root = Theo_III.tex
\chapter{Magnetostatik}


In den vorigen Kapiteln haben wir uns mit der Elektrostatik beschäftigt und gesehen, wie Ladungen dem Coulombgesetz gehorchen und ein elektrisches Feld hervorrufen, das die Grundgleichungen $\rot \vec {E}=0$ und $\divg \vec {D}=0$ erfüllt.

In der Magnetostatik betrachten wir jetzt stationäre (also nicht zeitabhängige) Ströme und die Kraftwirkung, die sie hervorrufen. Es wird ein Magnetfeld und die Stromdichte eingeführt, was schließlich auf eine integrale und differentielle Formulierung der Grundgesetze der Magnetostatik führt.

Es gibt Ähnlichkeiten zwischen der Elektro- und Magnetostatik, wie zum Beispiel das Abstandsverhalten und die Symmetrie einiger Formeln, aber auch wesentliche Unterschiede, unter Anderem in den Kraftrichtungen und Potentialen.

\section{Strom, Stromdichte und Kontinuitätsgleichung}

Der elektrische Strom $I$ ist als zeitliche Änderung der Ladung definiert,
\begin{equation*}
	I=\frac{\diff q}{\diff t}.
\end{equation*}
Die Einheit ist der Ampère. Aus der Ladungserhaltung folgt, dass der Strom konstant entlang eines Drahts ist.

Außerdem wird die Stromdichte als Strom pro Querschnittsfläche $A$
\begin{equation*}
	\vec {j}=\frac{\text{Strom}}{\text{Fläche}}=\frac{I}{A}\frac{\diff \vec {\vec {r}}}{\diff s}\:\xrightarrow{\Delta  f\diff s=\diff ^{3}r}\: \vec {j}\diff ^{3}\vec {r}=I\diff \vec {r}
\end{equation*}
definiert. Dabei ist $\diff\vec r$ als Leiterelement und $I\diff\vec r$ als gerichtetes Stromelement zu verstehen. Es gilt also 
\begin{equation*}
    I=\int_A \vec j \cdot \diff \vec A,
\end{equation*}
bzw. für eine gleichmäßig auf $A$ verteilte Stromdichte
\begin{equation*}
    I=\vec j\cdot \vec A.
\end{equation*}

Die Stromdichte lässt sich auch ausdrücken durch das Produkt aus Ladungsdichte und Geschwindigkeit,
\begin{equation*}
	\vec {j}=\rho \vec {v},
\end{equation*}
was eine mikroskopische Definition analog zu der der Ladungsdichte erlaubt\footnote{Zur Erinnerung: $\rho =\sum _{i}q_{i}\delta \left(\vec {r}-\vec {r}_{i}\right)$.}:
\begin{equation*}
	\vec {j}=\sum _{i}q_{i}\vec {v}_{i}\delta \left(\vec {r}-\vec {r}_{i}\right)
\end{equation*}
Die Stromdichte zeigt damit in dieselbe Richtung wie der Geschwindigkeitsvektor einer positiven Ladung. 

Zur Herleitung der Kontinuitätsgleichung betrachten wir die zeitliche Änderung der Ladung in einem Volumen $V$:
\begin{equation*}
	\frac{\diff Q}{\diff t}=\frac{\diff }{\diff t}\int _{V}\diff^{3}\vec {r}\rho \left(\vec {r},t\right)=\int _{V}\diff ^{3}\vec {r}\frac{\partial }{\partial t}\rho \left(\vec {r},t\right)
\end{equation*}
Wegen der Ladungserhaltung entspricht dies gerade dem Fluss der Stromdichte aus der Volumenoberfläche $\partial V$ heraus:
\begin{equation*}
	\int _{V}\diff ^{3}\vec {r}\frac{\partial }{\partial t}\rho \left(\vec {r},t\right)=-\int _{\partial V}\vec {j}\cdot \diff \vec {f}=-\int _{V}\diff ^{3}\vec {r}\divg \vec {j},
\end{equation*}
woraus sich die Kontinuitätsgleichung ergibt:
\begin{equation*}
	\frac{\partial }{\partial t}\rho +\divg \vec {j}=0
\end{equation*}
In der Magnetostatik ist $\partial _{t}\rho =0$ und damit
\begin{equation*}
	\divg \vec {j}=0.
\end{equation*}
\section{Leiter und Magnetfeld}

Auf Erde kommen im Wesentlichen zwei natürliche bekannte Magnetfelder vor: dasjenige der Erde und das Magnetfeld von speziellen Mineralen, wie zum Beispiel Magnetit. Hans Christian \O{}rsted entdeckte im 19. Jahrhundert, dass auch stromdurchflossene Leiter ein Magnetfeld erzeugen und André-Marie Ampère entdeckte fast zeitgleich, dass ein Magnetfeld eine Kraftwirkung auf Leiter hervorruft.

Auf ein stromdurchflossenes Volumenelement $\diff V=\diff^3 \vec r$ in einem Magnetfeld $\vec B$ wirkt nach dem folgenden Gesetz eine Kraft\footnote{vergleichlich mit $\vec {F}=q\vec {E}$, also Produkt aus Quelle und Feld in der Elektrostatik. }:
\begin{equation*}
	\diff \vec {F}=(\vec j\times\vec B)\diff V
\end{equation*}
Die Gesamtkraft auf einen ausgedehnten Leiter $V$ ergibt sich durch Integration:
\begin{equation*}
	\vec {F}=\int _{V}\vec {j}\times \vec {B}\diff V
\end{equation*}
Im Spezialfall für einen dünnen Leiter $C$ und ein Leiterelement $\diff\vec r$ ($\diff V = \diff A\diff\vec r$) gilt
\begin{align}
    \label{eq:def_magn_flussdichte}
    \diff \vec F &=I\diff\vec r\times\vec B \\
	\vec {F}&=\int _{V}\vec {j}\times \vec {B}\,\diff A \,\diff\vec r = \int_{A}j\diff A \cdot \int_C \diff\vec r\times \vec B \nonumber\\& = I\int_C \diff\vec r\times \vec B .\nonumber
\end{align}

\begin{figure}[htb]
	\centering
	\includegraphics{magnetic_field_conductor_diff_view.pdf}
	\caption{Stromelement $I\diff\vec r$ und Magnetfeld $\vec B$ eines Leiterelements $\diff \vec r$. }
	\label{fig:magnetic_field_conductor_diff_view}
\end{figure}
Wir untersuchen zunächst die magnetische Flussdichte einiger speziellen geometrischen Anordnungen. Es gilt für ein Leiterelement $\diff \vec {r}'$ (dargestellt in \Abbref{fig:magnetic_field_conductor_diff_view}):
\begin{equation*}
	\diff \vec {B}=\frac{\mu _{0}}{4\pi }I\diff \vec {r}'\times \frac{\vec {r}-\vec {r}'}{\left| \vec {r}-\vec {r}'\right| ^{3}}
\end{equation*}
Dieser Zusammenhang ist als Biot-Savartsches Gesetz für Leiter bekannt und ergibt sich aus den Betrachtungen $\left| \diff \vec {B}\right| \propto I\left| \diff \vec {r}'\right| , \left| \vec {r}-\vec {r}'\right| ^{-2}$ und $\diff \vec {B}\perp \diff \vec {r}',\vec {r}-\vec {r}'$. Hier wird außerdem die magnetische Feldkonstante $\mu _{0}\approx 4\pi \cdot 10^{-7}\,\si{\newton\per\square\ampere}$ eingeführt\footnote{Über die Kraft zwischen zwei stromdurchflossene parallele Leiter wurde früher die Einheit Ampère definiert und dabei festgelegt, dass $\frac{\mu _{0}}{4\pi }=\SI{1e-7}{\newton\per\square\ampere}$, aber die magnetische Feldkonstante wurde 2019 umdefiniert auf Basis der Elementarladung und Sekunde, wobei aber die Abweichung extrem gering ist. Damit ist die magnetische Feldkonstante eine experimentell zu ermittelnde Größe geworden. }. Die Flussdichte ist zwar proportional zu $r^{-2}$ wie beim elektrischen Feld einer Punktladung, aber im Gegensatz können isolierte Stromelemente $I\diff \vec {r}$ nicht existieren.

Für das Feld eines unendlich langen, geraden Leiters gilt
\begin{equation*}
    \label{eq:biot_savart}
	B\left(\rho \right)=\frac{\mu _{0}}{4\pi }I\rho \int _{-\infty }^{\infty }\frac{\diff z}{\left(\rho ^{2}+z^{2}\right)^{\frac{3}{2}}}=\frac{\mu _{0}}{4\pi }\frac{I}{\rho },
\end{equation*}


\begin{figure}[htb]
	\centering
	\includegraphics{force_parallel_conductors.pdf}
	\caption{Die Kraft auf parallele, stromdurchflossene Leiter ist anziehend, wenn der Stromfluss in verschiedene Richtungen geht und abstoßend, wenn der Strom in beiden Leiter in der gleichen Richtung fließt. }
	\label{fig:force_parallel_conductors}
\end{figure}

Diese Gleichung beschreibt das historische Biot-Savartsche Gesetz.

Verwendet man Gleichungen \eqref{eq:def_magn_flussdichte} und \eqref{eq:biot_savart}, so erhält man die Kraft zwischen zwei parallelen Leitern mit Abstand $d$ (siehe \Abbref{fig:force_parallel_conductors})
\begin{equation*}
    \frac{\diff\vec F}{\diff z} = \frac{\mu_0}{2\pi} \frac{I_1I_2}d,
\end{equation*}
die orthogonal zum Leiter ist.

Bis zum 20.5.2019 war der Ampère definiert als der Strom, der durch zwei parallele Leiter der Länge $\SI{1}{\m}$ mit $\SI{1}{\m}$ Abstand in gleicher Richtung fließt und eine Anziehungskraft von $\SI{1e-7}{\newton}$ bewirkt. 

Heute gilt
\begin{equation*}
	\SI{1}{\ampere}\equiv \frac{\SI{1}{\coulomb}}{\SI{1}{\s}}.
\end{equation*}
Zwischen beliebigen Leiterschleifen $C_{1},C_{2}$ wirkt eine Kraft von
\begin{equation*}
	\vec {F}_{21}=-\vec {F}_{12}=-\frac{\mu _{0}}{4\pi }I_{1}I_{2}\oint _{C_{1}}\oint _{C_{2}}\frac{\vec {r}_{1}-\vec {r}_{2}}{\left| \vec {r}_{1}-\vec {r}_{2}\right| ^{3}}\diff \vec {r}_{1}\cdot \diff \vec {r}_{2}.
\end{equation*}



\section{Grundgleichungen der Magnetostatik}

Um die Grundgleichungen der Magnetostatik herzuleiten, gehen wir zuerst von dem Stromelement $I\diff \vec {r}$ über in die Stromdichte $\vec {j}\left(\vec {r}\right)$ mit $I\diff \vec {r}=\vec {j}\left(\vec {r}\right)\diff^3 \vec {r}$.

Die Kraft auf ein Stromgebiet ist das Integral der Kraftdichte
\begin{equation*}
	\vec {F}=\int f\left(\vec {r}\right)\diff ^{3}\vec {r}=\int \vec {j}\left(\vec {r}\right)\times \vec {B}\left(\vec {r}\right)\diff ^{3}\vec {r}.
\end{equation*}
Für eine Punktladung $q$, die sich mit Geschwindigkeit $\vec {v}$ bewegt ($\vec {j}=q\vec {v}\delta \left(\vec {r}-\vec {r}\left(t\right)\right)$) gilt speziell
\begin{equation*}
	\vec F\left(\vec {r}\right)=q\vec {v}\times \vec {B}\left(\vec {r},t\right)
\end{equation*}
bzw. allgemein mit einem zusätzlichen elektrischen Feld
\begin{equation*}
	\vec F\left(\vec {r}\right)=q\left(\vec {v}\times \vec {B}\left(\vec {r},t\right)+\vec {E}\left(\vec {r},t\right)\right).
\end{equation*}
Diese Gesamtkraft ist die sogenannte Lorentzkraft.

Integration des Biot-Savartschen Gesetzes für Leiter führt auf\footnote{vergleichlich mit $\vec {E}(\vec r)=\frac{1}{4\pi \varepsilon _{0}}\int \rho \left(\vec {r}\right)\frac{\vec {r}-\vec {r}'}{\left| \vec {r}-\vec {r}'\right| ^{3}}\diff ^{3}\vec {r}'$.}
\begin{equation*}
	\vec {B}\left(\vec {r}\right)=\frac{\mu _{0}}{4\pi }\int \vec {j}\left(\vec {r'}\right)\times \frac{\vec {r}-\vec {r}'}{\left| \vec {r}-\vec {r}'\right| ^{3}}\diff ^{3}\vec {r}'.
\end{equation*}
Wie für das elektrische Feld können wir ein Potential einführen \textendash{} allerdings ist $\vec {B}$ kein Potentialfeld und daher ist das magnetische Potential ein Vektorpotential:
\begin{equation*}
	\vec {B}\left(\vec {r}\right)=\nabla \times \vec {A}\left(\vec {r}\right), \quad\vec {A}\left(\vec {r}\right)=\frac{\mu _{0}}{4\pi }\int \frac{\vec {j}\left(\vec {r}'\right)}{\left| \vec {r}-\vec {r}'\right| }\diff ^{3}\vec {r}'
\end{equation*}
Die Divergenz von $\vec {B}$ verschwindet, weil stets gilt, dass $\nabla\cdot(\nabla\times \vec A) = 0$.
\begin{formal}
		Das Magnetfeld hat keine Quellen, $\divg \vec {B}=0$.
\end{formal}
In der integralen Formulierung,
\begin{equation*}
	\int _{V}\divg \vec {B}\,\diff ^{3}\vec {r}=\int _{V}\vec {B}\cdot \diff \vec {f}=0,
\end{equation*}
bedeutet das, dass die magnetischen Feldlinien geschlossen sind. Es gibt folglich keine magnetischen Ladungen, wo die Feldlinien beginnen oder enden.

Für die Rotation der elektrischen Flussdichte gilt
\begin{align}
\begin{split}
    \label{eq:durchflutungsgesetz_magnetostatik}
	\rot \vec {B}&=\nabla \times \left(\nabla \times \vec {A}\right)=\nabla \left(\nabla \cdot \vec {A}\right)-\nabla ^{2}\vec {A}\\
	&=\frac{\mu _{0}}{4\pi }\nabla \int \vec {j}\left(\vec {r}'\right)\cdot \nabla \frac{1}{\left| \vec {r}-\vec {r}'\right| }\diff ^{3}\vec {r}'-\frac{\mu _{0}}{4\pi }\int \vec {j}\left(\vec {r}'\right)\nabla ^{2}\frac{1}{\left| \vec {r}-\vec {r}'\right| }\diff ^{3}\vec {r}'\\
	&=0+\mu _{0}\int \vec {j}\left(\vec {r}'\right)\delta \left(\vec {r}-\vec {r}'\right)\diff ^{3}\vec {r}'\\&=\mu _{0}\vec {j}\left(\vec {r}\right).
\end{split}
\end{align}

\begin{formal}
	Elektrische Ströme rufen Wirbel in der magnetischen Flussdichte hervor, $\rot \vec {B}=\mu _{0}\vec {j}\left(\vec {r}\right)$.
\end{formal}
Alternativ kann man sagen, dass die Zirkulation entlang der Oberfläche eines Volumens einem Strom durch das Volumen entspricht,
\begin{equation*}
	\boxed{\int _{\partial F}\vec {B}\cdot \diff \vec {r}=\mu _{0}I.}
\end{equation*}
Diese Gleichung ist als Ampèresches Gesetz bekannt. Mit seiner Nutzung kann man leicht das Magnetfeld von einem homogen stromdurchflossenen, zylindrischen Draht mit Radius $R$ bestimmen, wie in Abbildung \Abbref{fig:cylinder_conductor_with_diagrams} gezeigt. Wähle dazu eine kreisförmige Kurve $C$ mit Radius $r$ um die $z$-Achse herum. Aufgrund der Zylindersymmetrie ist das Magnetfeld nur vom Abstand $r$ der $z$-Achse abhängig und es gilt mit dem Ampèreschen Gesetz
\begin{align*}
		B\left(r\right)\oint _{C}1\diff s&=2\pi rB\left(r\right)=\mu _{0}I\left(r\right) \\
		\equivalence B\left(r\right)&=\frac{\mu _{0}I\left(r\right)}{2\pi r}.
\end{align*}


\begin{figure}[htb]
	\centering
	\includegraphics{cylinder_conductor_with_diagrams.pdf}
	\caption{Ein stromdurchflossener, zylindrischer Leiter mit Radius $R$ und Stromdichte $r$ erzeugt ein magnetisches Wirbelfeld. Links: schematische Darstellung, Mitte links: Querschnitt mit gedachter kreisförmiger Kurve $C$ mit Radius $r$ um den Leiter herum, Mitte rechts: Die Stromdichte ist konstant im Leiter und fällt außerhalb auf $0$ ab, rechts: im Leiter steigt der Betrag des Magnetfelds linear mit dem Abstand an und fällt außerhalb ab mit $r^{-1}$.  }
	\label{fig:cylinder_conductor_with_diagrams}
\end{figure}

Der Strom $I\left(r\right)$ enthält nur den Strom, der innerhalb der Kurve $C$ fließt. Innerhalb des zylindrischen Leiters ($r\leq R$) ist die eingeschlossene Fläche gerade $\pi r^{2}$ und damit
\begin{equation*}
	I\left(r\right)=j_{0}\pi r^{2}.
\end{equation*}


\begin{figure}[htb]
	\centering
	\includegraphics{long_coil_scheme.pdf}
	\caption{Magnetfeld einer lange, stromdurchflossenen Spule. }
	\label{fig:long_coil_scheme}
\end{figure}

Außerhalb des Leiters ist $I$ konstant $j_{0}\pi R^{2}$, weil sich der Leiter und damit der Stromfluss nur bis $r=R$ erstreckt. Das Magnetfeld ist also
\begin{align*}
	B\left(r\right)=\frac{\mu _{0}}{2}j_{0}\begin{cases} r,               & r\leq R \\
              \frac{R^{2}}{r}, & r>R
	                                       \end{cases}
\end{align*}
und es ist kreisförmig um die $z$-Achse gerichtet, $\vec {B}\left(r,\varphi \right)=B\left(r\right)\vec {e}_{\varphi }$.

Genauso lässt sich das Feld einer unendlich langen Spule berechnen (\Abbref{fig:long_coil_scheme}):
\begin{align*}
		\oint \vec {B}\cdot \diff \vec {r}&=LB_{0}\overset{!}{=}\mu _{0}NI \\
		\implication B_{0}&=\mu _{0}\frac{N}{L}I, \vec {B}=B_{0}\vec {e}_{z}
\end{align*}
Zwischen den Windungen hebt sich das magnetische Feld weg.

Wie aus der Gleichung \eqref{eq:durchflutungsgesetz_magnetostatik} hervorgeht, lässt sich für das Vektorpotential $\vec {A}$ wie in der Elektrostatik eine Poissongleichung formulieren,
\begin{equation*}
	\nabla ^{2}\vec {A}=-\mu _{0}\vec {j}
\end{equation*}
und im Potential $\vec {A}\left(\vec {r}\right)$ findet sich wieder eine Greenschen Funktion
\begin{equation*}
	\vec {A}\left(\vec {r}\right)=\int \diff ^{3}\vec {r}'G\left(\vec {r}-\vec {r}'\right)\vec {j}\left(\vec {r}'\right),\quad G\left(\vec {r}-\vec {r}'\right)=\frac{\mu _{0}}{4\pi }\frac{1}{\left| \vec {r}-\vec {r}'\right| }.
\end{equation*}
\section{Kleine Stromverteilungen: Der magnetische Dipol\label{mark-5.4}}

\subsection{Felder kleiner Stromverteilungen\label{mark-5.4.1}}

Wir führen zunächst wieder eine Entwicklung des magnetischen Potentials nach Momenten der Stromverteilung durch:
\begin{equation}
    \label{eq:entwicklung_magn_potential}
	\vec {A}\left(\vec {r}\right)=\frac{\mu _{0}}{4\pi }\int \diff ^{3}\vec {r}'\frac{\vec {j}\left(\vec {r}'\right)}{\left| \vec {r}-\vec {r}'\right| }\approx \frac{\mu _{0}}{4\pi }\left(\frac{1}{r}\int \diff ^{3}\vec {r}'\vec {j}\left(\vec {r}'\right)+\frac{\vec {r}}{r^{3}}\cdot \int \diff ^{3}\vec {r}'\vec {j}\left(\vec {r}'\right)\vec {r}'+\ldots \right)
\end{equation}
Die Auswertung ist allerdings viel komplizierter als für das elektrische Potential und daher beschränken wir uns hier auf die Dipole und vernachlässigen höhere Multipolmomente. Außerdem verwenden wir als Hilfsatz folgende Gleichung, die für beliebige skalare Felder $g$ und $f$ sowie für ein quellenfreies Vektorfeld $\vec {j}$ gilt\footnote{Beweis: Betrachte $\int _{V}\nabla \left(gf\vec {j}\right)\diff ^{3}\vec {r}$. Einerseits ist dieser Ausdruck mit dem Satz von Gauß gleich $\int _{\partial V}gf\vec {j}\cdot \diff \vec {f}$, was gleich $0$ ist, wenn man das Volumen groß genug wählt, dass auf dem Rand $\vec {j}\left(\vec {r}\right)=\vec {0}$ ist. Außerdem findet man mithilfe der Produktregel, dass $\int _{V}\nabla \left(gf\vec {j}\right)\diff ^{3}\vec {r}=\int _{V}\vec {j}\cdot \nabla \left(gf\right)\diff ^{3}\vec {r}+\int _{V}gf\nabla \cdot \vec {j}\diff ^{3}\vec {r}$, wobei der letzte Term aufgrund der Bedingung $\divg \vec {j}=\vec {0}$ verschwindet, q.e.d.}:
\begin{equation}
    \label{eq:hilfssatz}
	\int \vec {j}\cdot \nabla \left(gf\right)\diff ^{3}\vec {r}=0.
\end{equation}
Ferner können wir zeigen\footnote{Beweis: Mit der vorigen Formel $\int \vec {j}\cdot \nabla \left(gf\right)\diff ^{3}\vec {r}=0$ ist mit $g=1$ und $f=x_{i}$ offensichtlicherweise $\int j_{k}\cdot \nabla _{k}x_{i}\diff ^{3}\vec {r}=\int j_{i}\diff ^{3}\vec {r}=0$, q.e.d. }, dass
\begin{equation*}
	\int \vec {j}\left(\vec {r}\right)\diff ^{3}\vec {r}=0,
\end{equation*}
es also keine Strommonopole in $\vec {A}\left(\vec {r}\right)$ gibt. Als letzte Vorbereitung werten wir den zweiten Term in Gleichung \eqref{eq:entwicklung_magn_potential} aus
\begin{equation*}
	\vec {r}\cdot \int \diff ^{3}\vec {r}'\vec {j}\left(\vec {r}'\right)\vec {r}'=x_{i}\int \diff ^{3}\vec {r}'x_{i}^{'}j_{k}\left(\vec {r}'\right).
\end{equation*}
Das Produkt $T=x_{i}^{'}j_{k}$ ist ein Tensor zweiter Stufe und lässt sich zerlegen in einen symmetrischen und einen antisymmetrischen Teil, $T_{ik}=\frac{1}{2}\left(T_{ik}+T_{ki}\right)+\frac{1}{2}\left(T_{ik}-T_{ki}\right)$, also
\begin{equation*}
	\vec {r}\cdot \int \diff ^{3}\vec {r}'\vec {j}\left(\vec {r}'\right)\vec {r}'=x_{i}\int \diff ^{3}\vec {r}'\left(\frac{1}{2}\left(x_{i}^{'}j_{k}+x_{k}^{'}j_{i}\right)+\frac{1}{2}\left(x_{i}^{'}j_{k}-x_{k}^{'}j_{i}\right)\right).
\end{equation*}
Der erste Summand im Integranden (symmetrischer Teil des Tensors) verschwindet nach dem Hilfssatz \eqref{eq:hilfssatz} mit $g=x_{i}^{'},f=x_{k}^{'}$ und der zweite wird zu
\begin{equation*}
	\frac{1}{2}\int \diff ^{3}\vec {r}'\left(\left(\vec {r}\cdot \vec {r}'\right)\vec {j}-\left(\vec {r}\cdot \vec {j}\right)\vec {r}'\right)=\frac{1}{2}\int \diff ^{3}\vec {r}'\left(\vec {r}'\times \vec {j}\right)\times \vec {r}
\end{equation*}
evaluiert.

Nun können wir das magnetische Dipolmoment
\begin{equation*}
	\vec {m}=\frac{1}{2}\int \vec {r}'\times \vec {j}\left(\vec {r}\right)\diff ^{3}\vec {r}'
\end{equation*}
und das Vektorpotential des magnetischen Dipols
\begin{equation*}
	\vec {A}\left(\vec {r}\right)=\frac{\mu _{0}}{4\pi }\frac{\vec {m}\times \vec {r}}{r^{3}}
\end{equation*}
einführen. Daraus ergibt sich (mit weiteren Umformungen) schließlich das Magnetfeld eines Dipols
\begin{equation*}
	\vec {B}\left(\vec {r}\right)=\rot \vec {A}=-\frac{\mu _{0}}{4\pi }\nabla \frac{\vec {m}\cdot \vec {r}}{r^{3}}=\frac{\mu _{0}}{4\pi }\frac{3\left(\vec {m}\cdot \hat{\vec {r}}\right)\hat{\vec {r}}-\vec {m}}{r^{3}},
\end{equation*}
was wieder völlig analog zum elektrischen Dipol ist.

Wir wollen im Folgenden einmal das Dipolmoment zweier einfacher Geometrien berechnen. Das Dipolmoment einer Stromschleife (\Abbref{fig:magnetic_dipole_field_circle_conductor_ellipsoid}, links) ist einfach (verwende $\vec {j}\diff ^{3}\vec {r}=I\diff \vec {r}$)
\begin{equation*}
	\vec {m}=\frac{I}{2}\oint \vec {r}\times \diff \vec {r}.
\end{equation*}
Für eine ebene Schleife mit Fläche $F$ und Normalenvektor $\vec {n}$ (\Abbref{fig:magnetic_dipole_field_circle_conductor_ellipsoid}, Mitte) vereinfacht sich dieser Ausdruck zu
\begin{equation*}
	\vec {m}=IF\vec {n}.
\end{equation*}


\begin{figure}[htb]
	\centering
	\includegraphics{magnetic_dipole_field_circle_conductor_ellipsoid.pdf}
	\caption{Links: Magnetfeld eines magnetischen Dipols, Mitte: Ein Kreisstrom erzeugt ein magnetisches Dipolmoment, rechts: starrer Körper, der mit $\vec\omega$ rotiert und einer Ladungsverteilung $\rho(\vec r)$ besitzt. }
	\label{fig:magnetic_dipole_field_circle_conductor_ellipsoid}
\end{figure}

Zuletzt betrachten wir einen starren geladenen Körper, der um eine Rotationsachse $\vec {\omega }$ rotiert (\Abbref{fig:magnetic_dipole_field_circle_conductor_ellipsoid}, rechts). Die lokale Geschwindigkeit für eine Ladung in diesem Körper ist $\vec {v}\left(\vec {r}\right)=\vec {\omega }\times \vec {r}$. Die rotierenden Ladungen resultieren in einer Stromdichte $\vec {j}=\rho \vec {v}$, sodass ein Dipolmoment induziert wird:
\begin{equation*}
	\vec {m}=\frac{1}{2}\int d^{3}\vec {r}\,\rho \left(\vec {r}\right)\vec {r}\times \vec {v}.
\end{equation*}
Gleichzeitig besitzt der Körper einen Drehimpuls
\begin{equation*}
	\vec {L}=\int \diff ^{3}\vec {r}\rho _{m}\left(\vec {r}\right)\vec {r}\times \vec {v}.
\end{equation*}
Trifft man jetzt die Annahme, dass die Verteilungen von Ladung und Masse im Körper gleich sind,
\begin{equation*}
	\frac{\rho \left(\vec {r}\right)}{\rho _{m}\left(\vec {r}\right)}=\frac{q_{\mathrm{ges}}}{m_{\mathrm{ges}}}\equiv \frac{q}{M},
\end{equation*}
dann findet man eine Proportionalität von $\vec {m}$ und $\vec {L}$:
\begin{equation*}
	\vec {m}=\frac{q}{2M}\vec {L}
\end{equation*}
Der Proportionalitätsfaktor $\left| \vec {m}\right| /\left| \vec {L}\right| $ wird als gyromagnetisches Verhältnis bezeichnet. Für den Spin gibt es allerdings eine Abweichung, die aus der Relativitätstheorie hervorgeht und es gilt eigentlich
\begin{equation*}
	\vec {m}=g\frac{q}{2M}\vec {L}
\end{equation*}
mit einem zusätzlichen Landé-Faktor (oder g-Faktor), dessen Wert von der Teilchensorte abhängt. Für Elektronen ist $g\approx 2$, für Protonen $g\approx 2\cdot 2,79$ und für Neutronen $g\approx 2\cdot \left(-1,91\right)$.

Auch die Erde ist ein magnetischer Dipol, der durch Ströme im flüssigen äußeren Erdkern angetrieben wird. Bemerkenswerterweise ist die Dipolachse leicht gegen die Erdachse geneigt und die Polarität dreht sich rund alle 200.000 Jahre um. Der Mechanismus ist allerdings bis heute nicht genau verstanden.

\subsection{Kraft, Drehmoment und Energie\label{mark-5.4.2}}

Die bereits bekannte Kraft auf ein Volumen $V$ mit Stromverteilung $\vec {j}\left(\vec {r}\right)$
\begin{equation*}
	\vec {F}=\int \vec {j}\left(\vec {r}\right)\times \vec {B}\left(\vec {r}\right)\diff ^{3}\vec {r}
\end{equation*}
lässt sich nach $\vec {B}$ um $\vec {r}=\vec {0}$ Taylor-entwickeln:
\begin{equation*}
	\vec {F}\approx \int \vec {j}\left(\vec {r}\right)\times \left(\vec {B}\left(0\right)+\vec {r}\cdot \nabla \vec {B}\left(0\right)+\ldots \right)\diff ^{3}\vec {r}=\left(\vec {m}\times \nabla \right)\times \vec {B}\left(0\right)=\nabla \left(\vec {m}\cdot \vec {B}\right)
\end{equation*}
Die potentielle Energie eines Dipols $\vec {m}$ im magnetischen Feld $\vec {B}$ ergibt sich daraus durch Integration
\begin{equation*}
	F=-\nabla U\implication U=-\vec {m}\cdot \vec {B}.
\end{equation*}
Das Minimum wird für $\vec {m}\upuparrows \vec {B}$ erreicht\footnote{vgl. Funktionsweise von einem Kompass. }. $U$ enthält aber nicht die Energie um den Dipol $\vec {m}$ aufrechtzuerhalten!

Das Drehmoment hat die gleiche Form wie für elektrische Dipole,
\begin{equation*}
	\vec {M}=\vec {m}\times \vec {B}
\end{equation*}
und auch die Wechselwirkungsenergie für einen Dipol im Feld eines zweiten Dipols ist analog
\begin{equation*}
	U=-\frac{\mu _{0}}{4\pi }\frac{3\left(\vec {m}_{1}\cdot \hat{\vec {r}}\right)\left(\vec {m}_{2}\cdot \hat{\vec {r}}\right)-\vec {m}_{1}\cdot \vec {m}_{2}}{r^{3}}.
\end{equation*}



\section{Magnetische Felder in Materie}

Nun behandeln wir magnetische Felder in Materie. Die Vorgehensweise erfolgt analog zu den Kapiteln \ref{sec:mikroskopische_gleichungen_der_elektrostatik} und \ref{sec:Makroskopische_Gleichungen_der_Elektrostatik}. Die freien Ladungen bilden freie Ströme $j_{\mathrm{f}}\left(\vec {r},t\right)$ und gebundene Ladungen fügen jetzt noch gebundene Ströme hinzu $\vec {j}_{\mathrm{b}}\left(\vec {r},t\right)$. Außerdem werden auch magnetische Dipole durch den intrinsischen Spin der Teilchen hinzuaddiert. Die letzten beiden Quellen werden dann gemittelt über eine neue Größe beschrieben: die Magnetisierung.

\subsection{Einführung der Vakuumsverschiebungsstromdichte}

In der Magnetostatik haben wir bisher nur
\begin{equation*}
	\nabla \times \vec {B}=\mu _{0}\vec {j},\quad\nabla \cdot \vec {j}=0
\end{equation*}
gehabt. Aber mit der vollständigen Kontinuitätsgleichung
\begin{equation*}
	\frac{\partial }{\partial t}\rho +\nabla \cdot \vec {j}=0
\end{equation*}
brauchen wir eine Verallgemeinerung, wenn $\nabla \cdot \vec {j}\neq 0$:
\begin{equation*}
	\nabla \times \vec {B}=\mu _{0}\vec {j}+\frac{1}{c^{2}}\frac{\partial \vec {E}}{\partial t}=\mu _{0}\left(\vec {j}+\varepsilon _{0}\frac{\partial \vec {E}}{\partial t}\right)
\end{equation*}
Das ist das Ampèresche Gesetz mit der Maxwellschen Verallgemeinerung.

\subsection{Einführung der Magnetisierung}

Wie für die Elektrostatik in Kapitel \ref{sec:mikroskopische_gleichungen_der_elektrostatik} und \ref{sec:Makroskopische_Gleichungen_der_Elektrostatik} beginnen wir, indem wir die mikroskopischen Gleichungen
\begin{equation*}
	\nabla\times \vec {b}=\mu _{0}\left(\vec {j}+\varepsilon _{0}\frac{\partial \vec {e}}{\partial t}\right), \quad\nabla\cdot \vec {b}=0
\end{equation*}
mitteln,
\begin{equation*}
	\vec {B}\left(\vec {r},t\right)=\left\langle \vec {b}\left(\vec {r},t\right)\right\rangle ,\quad \vec {E}\left(\vec {r},t\right)=\left\langle \vec {e}\left(\vec {r},t\right)\right\rangle .
\end{equation*}
Die Stromdichte
\begin{equation*}
	\vec {j}\left(\vec {r},t\right)=\vec {j}_{\mathrm{f}}\left(\vec {r},t\right)+\vec {j}_{\mathrm{b}}\left(\vec {r},t\right)
\end{equation*}
mit "`freien`` Strömen durch einzelne Ladungsträger mit Ladung $q_{i}$ und Geschwindigkeit $v_{i }$
\begin{equation*}
	\vec {j}_{\mathrm{f}}\left(\vec {r},t\right)=\sum _{i\left(f\right)}q_{i}\vec {v}_{i}\delta \left(\vec {r}-\vec {r}_{i}\left(t\right)\right)\implication \vec {j}_{\mathrm{F}}=\left\langle \vec {j}_{\mathrm{f}}\right\rangle .
\end{equation*}
Die gebundenen Ströme werden über alle Moleküle summiert, $\vec {j}_{\mathrm{b}}\left(\vec {r},t\right)=\sum _{n}\vec {j}_{n}\left(\vec {r},t\right)$ mit
\begin{equation*}
	\vec {j}_{n}\left(\vec {r},t\right)=\sum _{i\left(n\right)}q_{i}\vec {v}_{i}\delta \left(\vec {r}-\vec {r}_{i}\right)=\sum _{i\left(n\right)}q_{i}\left(\vec {v}_{n}+\vec {v}_{ni}\right)\delta \left(\vec {r}-\left(\vec {r}_{n}+\vec {r}_{ni}\right)\right)
\end{equation*}
und die Mittelung für das $n$-te Molekül ist (verwende wieder eine Glättungsfunktion $f$)
\begin{equation*}
	\left\langle \vec {j}_{n}\left(\vec {r},t\right)\right\rangle =\sum _{i\left(n\right)}q_{i}\left(\vec {v}_{n}+\vec {v}_{ni}\right)f\left(\vec {r}-\left(\vec {r}_{n}+\vec {r}_{ni}\right)\right).
\end{equation*}
Eine Taylor-Entwicklung um $\vec {r}_{n}$ liefert
\begin{equation*}
	\left\langle \vec {j}_{n}\left(\vec {r},t\right)\right\rangle =\sum _{i\left(n\right)}q_{i}\left(\vec {v}_{n}+\vec {v}_{ni}\right)\left[f\left(\vec {r}-\vec {r}_{n}\right)-\vec {r}_{ni}\cdot \nabla f\left(\vec {r}-\vec {r}_{n}\right)+\ldots \right].
\end{equation*}
Betrachte zunächst die Terme mit $\sum _{i\left(n\right)}q_{i}\vec {v}_{ni}$, die die molekularen Dipolmomente liefern. Der erste Term kann durch ein elektrisches Dipolmoment beschrieben werden:
\begin{equation*}
	\sum _{i\left(n\right)}q_{i}\vec {v}_{ni}=\sum _{i\left(n\right)}q_{i}\frac{\diff \vec {r}_{ni}}{\diff t}=\frac{\diff }{\diff t}\vec {p}_{n}
\end{equation*}
Der zweite Term enthält ein magnetisches Moment (verwende wieder die Tensorzerlegung in einen symmetrischen und einen asymmetrischen Anteil):
\begin{equation*}
	\sum _{i\left(n\right)}q_{i}\left(\vec {r}_{ni}\right)_{\beta }\left(\vec {v}_{ni}\right)_{\alpha }=\sum _{i\left(n\right)}\frac{1}{2}\left[q_{i}\left(\vec {r}_{ni}\right)_{\beta }\left(\vec {v}_{ni}\right)_{\alpha }-q_{i}\left(\vec {r}_{ni}\right)_{\alpha }\left(\vec {v}_{ni}\right)_{\beta }\right]+\sum _{i\left(n\right)}\frac{1}{2}\left[q_{i}\left(\vec {r}_{ni}\right)_{\beta }\left(\vec {v}_{ni}\right)_{\alpha }+q_{i}\left(\vec {r}_{ni}\right)_{\alpha }\left(\vec {v}_{ni}\right)_{\beta }\right],
\end{equation*}
wobei die Summanden der zweiten Summe verschwinden, weil diese Ströme jeweils auf ein Molekül beschränkt sind ($\nabla\cdot \vec {j}_{\mathrm{b}}=0$). Die erste Summe ist asymmetrisch und lässt sich als Kreuzprodukt schreiben,
\begin{equation*}
	\sum _{i\left(n\right)}q_{i}\left(\vec {r}_{ni}\right)_{\beta }\left(\vec {v}_{ni}\right)_{\alpha }=\frac{1}{2}\varepsilon _{\alpha \beta \gamma }\left(\sum _{i\left(n\right)}q_{i}\left(\vec {r}_{ni}\times \vec {v}_{ni}\right)\right)_{\gamma },
\end{equation*}
sodass sich ein sinnvolles molekulares magnetisches Dipolmoment ergibt,
\begin{equation*}
	\vec {m}_{n}=\frac{1}{2}\sum _{i\left(n\right)}q_{i}\left(\vec {r}_{ni}\times \vec {v}_{ni}\right).
\end{equation*}
Sammle nun alle bisher ausgewerteten Ausdrücke:
\begin{align*}
	\left\langle \left(\vec {j}_{n}\left(\vec {r},t\right)\right)_{\alpha }\right\rangle &=\left(\left(\sum _{i\left(n\right)}q_{i}\left(\vec {v}_{n}\right)_{\alpha }\right)+\frac{\diff }{\diff t}\left(\vec {p}_{n}\right)_{\alpha }\right)f\left(\vec {r}-\vec {r}_{n}\right)\\
	&\qquad-\left(\left(\sum _{i\left(n\right)}q_{i}\left(\vec {v}_{n}\right)_{\alpha }\left(\vec {r}_{ni}\right)_{\beta }\right)-\varepsilon _{\alpha \beta \gamma }\left(\vec {m}_{n}\right)_{\gamma }\right)\nabla _{\beta }f\left(\vec {r}-\vec {r}_{n}\right) \\
	&=\left\langle q_{n}\left(\vec {v}_{n}\right)_{\alpha }\delta \left(\vec {r}-\vec {r}_{n}\right)\right\rangle +\frac{\partial }{\partial t}\left\langle \left(\vec {p}_{n}\right)_{\alpha }\delta \left(\vec {r}-\vec {r}_{n}\right)\right\rangle +\varepsilon _{\alpha \beta \gamma }\nabla _{\beta }\left\langle \left(\vec {m}_{n}\right)_{\gamma }\delta \left(\vec {r}-\vec {r}\right)\right\rangle\\
	&\qquad -\nabla _{\beta }\left\langle \left(\vec {v}_{n}\right)_{\alpha }\left(\vec {p}_{n}\right)_{\beta }\delta \left(\vec {r}-\vec {r}_{n}\right)\right\rangle
\end{align*}
Den letzten Term vernachlässigen wir im Folgenden als Term höherer Ordnung (Details sind im Buch von Jackson, Kap. 6.6, zu finden).

Mit diesen Ergebnissen erhalten wir jetzt schlussendlich die gemittelte Stromdichte der gebundenen Ladungen,
\begin{equation*}
	\left\langle \vec {j}_{\mathrm{b}}\left(\vec {r},t\right)\right\rangle =\sum _{n}\left\langle \vec {j}_{n}\left(\vec {r},t\right)\right\rangle =j_{\mathrm{M}}\left(\vec {r},t\right)+\frac{\partial }{\partial t}\vec {P}\left(\vec {r},t\right)+\nabla \times \vec {M}\left(\vec {r},t\right)
\end{equation*}
mit der Stromdichte der gebundenen Ladungen
\begin{equation*}
	\vec {j}_{\mathrm{M}}\left(\vec {r},t\right)=\left\langle \sum _{n}q_{n}\vec {v}_{n}\delta \left(\vec {r}-\vec {r}_{n}\right)\right\rangle ,
\end{equation*}
der makroskopischen Polarisation
\begin{equation*}
	\vec {P}\left(r,t\right)=\left\langle \sum _{n}\vec {p}_{n}\delta \left(\vec {r}-\vec {r}_{n}\right)\right\rangle
\end{equation*}
und der neu eingeführten Magnetisierung (Dichte der magnetischen Dipole)
\begin{equation*}
	\vec {M}\left(\vec {r},t\right)=\left\langle \sum _{n}\vec {m}_{n}\delta \left(\vec {r}-\vec {r}_{n}\right)\right\rangle .
\end{equation*}
Daraus ergibt sich die die gemittelte Stromdichte als Summe der makroskopischen und mikroskopischen Stromdichte,
\begin{equation*}
	\left\langle \vec {j}\left(\vec {r},t\right)\right\rangle =\left\langle \vec {j}_{\mathrm{f}}\left(\vec {r},t\right)\right\rangle +\left\langle \vec {j}_{\mathrm{b}}\left(\vec {r},t\right)\right\rangle =\underset{\vec {j}_{\mathrm{Ma}}}{\underbrace{\vec {j}_{\mathrm{F}}\left(\vec {r},t\right)+\vec {j}_{\mathrm{M}}\left(\vec {r},t\right)}}+\underset{\vec {j}_{\mathrm{Mi}}\left(\vec {r},t\right)}{\underbrace{\frac{\partial }{\partial t}\vec {P}\left(\vec {r},t\right)+\nabla \times \vec {M}\left(\vec {r},t\right)}}
\end{equation*}
und wir können die makroskopischen Feldgleichungen formulieren:
\begin{align*}
	\divg \vec {B}&=0 \\
	\rot \vec {B}&=\mu _{0}\left(\vec {j}_{\mathrm{Ma}}\left(\vec {r},t\right)+\vec {j}_{\mathrm{Mi}}\left(\vec {r},t\right)+\varepsilon _{0}\frac{\partial }{\partial t}\vec {E}\left(\vec {r},t\right)\right)\\
	&=\mu _{0}\left(\vec {j}_{\mathrm{Ma}}\left(\vec {r},t\right)+\frac{\partial }{\partial t}\vec {D}\left(\vec {r},t\right)+\nabla \times \vec {M}\left(\vec {r},t\right)\right)
\end{align*}
Nun führen wir noch das magnetische Feld $\vec {H}$ ein,
\begin{equation*} 
	\vec {H}\left(\vec {r},t\right)=\frac{1}{\mu _{0}}\vec {B}\left(\vec {r},t\right)-M\left(\vec {r},t\right)\equivalence \vec {B}\left(\vec {r},t\right)=\mu _{0}\left(H\left(\vec {r},t\right)+M\left(\vec {r},t\right)\right)
\end{equation*}
Die Magnetisierung $\vec {M}$ wird also wieder in ein Hilfsfeld integriert, was einen einfacheren Ausdruck für die zweite Feldgleichung erlaubt:
\begin{equation*}
	\boxed{\rot \vec {H}=\vec {j}_{\mathrm{Ma}}+\frac{\partial }{\partial t}\vec {D}}
\end{equation*}

\begin{formal}
		Die Wirbel des Magnetfelds $\vec {H}$ sind die makroskopische Stromdichte $\vec {j}_{\mathrm{Ma}}$ und der dielektrische Verschiebungsstrom $\partial _{t}\vec {D}$.
\end{formal}

Das fundamentale magnetische Feld ist die magnetische Flussdichte $\vec {B}$.



% !TEX root = Theo_III.tex


\chapter{Grundgleichungen der Elektrodynamik}

\section{Die Maxwellschen Gleichungen: Zusammenstellung}

Die Maxwellgleichungen bilden die Basis der Elektrodynamik und fast alle Gleichungen zur Beschreibung elektromagnetische Phänomene können aus ihnen hergeleitet werden. Das Ziel ist die Beschreibung der Dynamik von elektrischen und magnetischen \textendash{} den elektromagnetischen \textendash{} Feldern.

Alle Maxwell-Gleichungen wurden in den vorangegangenen Kapiteln bereits in einer oder mehreren Formen behandelt.

Das Gaußsche Gesetz
\begin{equation*}
	\divg \vec {D}=\rho
\end{equation*}
besagt, dass die Quellen des dielektrischen Feldes makroskopische Ladungsdichten sind. Die magnetische Flussdichte hat keine Quellen, es gibt keine magnetischen Monopole und die Feldlinien sind geschlossen,
\begin{equation*}
	\divg \vec {B}=0.
\end{equation*}
Nach dem Faradayschen Induktionsgesetz
\begin{equation*}
	\rot \vec {E}=-\frac{\partial \vec {B}}{\partial t}
\end{equation*}
erzeugen veränderliche Magnetfelder rotierende elektrische Felder. Zuletzt beschreibt das Ampèresche Gesetz (mit Maxwellschem Verschiebungsstrom), dass rotierende Magnetfelder aus Stromdichten und veränderlichen dielektrischen Feldern entstehen,
\begin{equation*}
	\rot \vec {H}=\vec {j}+\frac{\partial \vec {D}}{\partial t}.
\end{equation*}
Diese Gleichungen gelten sowohl im Vakuum als auch in Materie. Dabei ist $\rho \left(\vec {r},t\right)$ die makroskopische Ladungsdichte und $\vec {j}\left(\vec {r},t\right)$ die makroskopische Stromdichte. Diese erfüllen die schon bekannte Kontinuitätsgleichung
\begin{equation*}
	\frac{\partial \rho }{\partial t}+\divg \vec {j}=0.
\end{equation*}
Im Vakuum nehmen die Maxwell-Gleichungen die folgende Form an:
\begin{equation*}
	\divg \vec {E}=\frac{1}{\varepsilon _{0}}\rho ,\quad \rot \vec {E}=-\frac{\partial \vec {B}}{\partial t},\quad \divg \vec {B}=0,\quad \rot \vec {B}=\mu _{0}\vec {j}+\frac{1}{c^{2}}\frac{\partial }{\partial t}\vec {E}
\end{equation*}
Auch die mikroskopischen Maxwellgleichungen haben wir kennengelernt. In Materie wird über stark fluktuierende Felder gemittelt, $\vec {E}=\left\langle \vec {e}\right\rangle $ und $\vec {B}=\left\langle \vec {b}\right\rangle $, sodass für die Ladungs- und Flussdichte gilt:
\begin{align*}
	\left\langle \tilde{\rho }\right\rangle    & =\rho -\nabla \cdot \vec {P}+\partial _{k}\partial _{l}Q_{kl}+\ldots  \\
	\left\langle \tilde{\vec {j}}\right\rangle & =\vec {j}+\frac{\partial }{\partial t}\vec {P}+\nabla \times \vec {M}
\end{align*}
Durch Einführung des dielektrischen Verschiebungsfeldes $\vec {D}$ und des Magnetfeldes $\vec {H}$ als Hilfsfelder erhält man dann die Materialgleichungen
\begin{equation*}
	\vec {D}=\varepsilon _{0}\vec {E}+\vec {P}-\nabla Q, \vec {B}=\mu _{0}\left(\vec {H}+\vec {M}\right)
\end{equation*}
bzw. in linearen Medien einfacher
\begin{align*}
	\vec {P} & =\varepsilon _{0}\chi _{e}\vec {E} & \Rightarrow \vec {D} & =\varepsilon \vec {E},\quad \varepsilon =\varepsilon _{0}\left(1+\chi _{e}\right)=\varepsilon _{0}\varepsilon _{r} \\
	\vec {M} & =\chi _{m}\vec {H}                 & \Rightarrow \vec {B} & =\mu \vec {H},\quad \mu =\mu _{0}\left(1+\chi _{m}\right)=\mu _{0}\mu _{r}.
\end{align*}
Die Kraftdichte auf Ladungsdichten und Stromdichten ist
\begin{equation*}
	f\left(\vec {r},t\right)=\rho \vec {E}+\vec {j}\times \vec {B},
\end{equation*}
also gerade die "`Lorentzkraft-Dichte`` als Kontinuumsform der Lorentzkraft.

\section{Elektromagnetische Potentiale}

Eine allgemeine Lösungsstrategie der Maxwellschen Gleichungen liefert die Bestimmung der elektromagnetischen Potentiale. Die Maxwellschen Gleichungen lassen sich als Paare von zwei homogenen Differentialgleichungen,
\begin{equation*}
	\divg \vec {B}=0, \quad\rot \vec {E}+\frac{\partial \vec {B}}{\partial t}=0
\end{equation*}
und zwei inhomogenen Differentialgleichungen,
\begin{equation*}
	\divg \vec {D}=\rho ,\quad \rot \vec {H}-\frac{\partial \vec {D}}{\partial t}=\vec {j}
\end{equation*}
schreiben.

Zuerst werden die homogenen Maxwellgleichungen gelöst. Die zweite Gleichung lässt sich umschreiben zu
\begin{equation*}
	0=\rot \left(\vec {E}+\frac{\partial \vec {A}}{\partial t}\right)\equiv -\rot \left(\nabla \varphi \right).
\end{equation*}
Dabei haben wir ein Vektorpotential $\vec {B}\equiv \nabla \times \vec {A}$ und ein skalares Potential $\vec {E}+\partial _{t}\vec {A}=-\nabla \varphi $ eingeführt. Das magnetische Vektorpotential $\vec {A}$ ist in dieser Form bereits bekannt und das skalare Potential $\varphi $ ist die Verallgemeinerung des elektrischen Potentials auf dynamische Felder.

Die so gewonnenen Gleichungen
\begin{equation}
	\label{eq:equations_for_B_and_E}
	\vec {B}=\rot \vec {A},\quad \vec {E}=-\nabla \varphi -\frac{\partial \vec {A}}{\partial t}
\end{equation}
können jetzt in die inhomogenen Maxwellgleichungen eingesetzt werden, wobei wir uns im Folgenden auf die Gleichungen im Vakuum beschränken. Einsetzen in das Gaußsche Gesetz liefert
\begin{equation*}
	\nabla ^{2}\varphi +\frac{\partial }{\partial t}\nabla \cdot \vec {A}=-\frac{\rho }{\varepsilon _{0}}
\end{equation*}
und aus dem Ampèreschen Gesetz folgt
\begin{equation*}
	\nabla \times \left(\nabla \times \vec {A}\right)=\mu _{0}\vec {j}-\frac{1}{c^{2}}\left(\nabla \frac{\partial \varphi }{\partial t}+\frac{\partial ^{2}\vec {A}}{\partial t^{2}}\right),
\end{equation*}
was man umstellen kann zu
\begin{equation}
	\label{eq:em_wellengleichung}
	\nabla ^{2}\vec {A}=\frac{1}{c^{2}}\frac{\partial ^{2}}{\partial t^{2}}\vec {A}-\nabla \left(\nabla \cdot \vec {A}+\frac{1}{c^{2}}\frac{\partial \varphi }{\partial t}\right)=-\mu _{0}\vec {j}.
\end{equation}
Wir erhalten also insgesamt vier gekoppelte partielle Differentialgleichungen für $\vec {A}$ und $\varphi $ statt acht für $\vec {E}$ und $\vec {B}$. In der Wellengleichung \eqref{eq:em_wellengleichung} können wir zur Bequemlichkeit den D’Alembert-Operator oder auch Wellenoperator
\begin{equation*}
	\Box\equiv\frac{1}{c^{2}}\frac{\partial ^{2}}{\partial t^{2}}-\nabla ^{2}
\end{equation*}
einführen.

In den Gleichungen \eqref{eq:equations_for_B_and_E} sind allerdings die Potentiale $\varphi $ und $\vec {A}$ nicht eindeutig, denn sie können mithilfe einer beliebigen skalaren Eichfunktion $\lambda \left(\vec {r},t\right)$ umgeeicht werden:
\begin{equation}
	\label{eq:umeichung}
	\tilde{\vec {A}}=\vec {A}+\nabla \lambda , \quad\tilde{\varphi }=\varphi -\frac{\partial \lambda }{\partial t}
\end{equation}
Alle auf diese Weise gewonnenen Potentiale $\tilde{\varphi }$ und $\tilde{\vec {A}}$ lassen die physikalischen Felder in \eqref{eq:equations_for_B_and_E} unverändert, was als Eichinvarianz bezeichnet wird. Es wird meist eine spezielle Wahl für die Eichung getroffen.

\subsection{Lorenz-Eichung}

Für die Lorenz-Eichung werden $\vec {A}$ und $\varphi $ so gewählt, dass
\begin{equation}
	\label{eq:lorenzeichung}
	\nabla \cdot \vec {A}+\frac{1}{c^{2}}\frac{\partial \varphi }{\partial t}=0,
\end{equation}
Diese Gleichung ist invariant unter den Lorentztransformation der speziellen Relativitätstheorie. Es ergeben sich damit die folgenden inhomogenen Wellengleichungen für $\varphi $ und $\vec {A}$:
\begin{align*}
	\nabla ^{2}\vec {A}-\frac{1}{c^{2}}\frac{\partial ^{2}}{\partial t^{2}}\vec {A} & =-\mu _{0}\vec {j}\equivalence \Box \vec {A}=\mu _{0}\vec {j}                              \\
	\nabla ^{2}\varphi -\frac{1}{c^{2}}\frac{\partial ^{2}}{\partial t^{2}}\varphi  & =-\frac{1}{\varepsilon _{0}}\rho \equivalence \Box \varphi =\frac{1}{\varepsilon _{0}}\rho
\end{align*}
Bemerkenswerterweise sind jetzt die verschiedenen Potentiale entkoppelt. Die Lorenz-Eichung ist immer durchführbar, denn man $\lambda $ nur nach der Differentialgleichung (setze \eqref{eq:umeichung} in \eqref{eq:lorenzeichung} ein)
\begin{equation*}
	\nabla ^{2}\lambda -\frac{1}{c^{2}}\frac{\partial ^{2}}{\partial t^{2}}\lambda =-\left(\nabla \cdot \vec {A}+\frac{1}{c^{2}}\frac{\partial \varphi }{\partial t}\right)
\end{equation*}
zu wählen, was stets möglich ist.




\end{document}
